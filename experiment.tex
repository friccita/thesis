\chapter{Experiment description\label{sec:experiment}}

\section{Large Hadron Collider\label{sec:lhc}}

The Large Hadron Collider (see~\cite{1748-0221-3-08-S08001} for a detailed description) is a circular accelerator and collider of high-energy particles, straddling the French-Swiss border near Geneva, Switzerland. Constructed between 1998 and 2008, the accelerator consists of 2 rings for counter-circulating proton or ion beams, 27 km in circumference and located at depths as low as 175 m underground in the tunnel previously occupied by the LEP collider ring. Superconducting magnets, composed of coils of NbTi Rutherford cable cooled to temperatures below 2 K with superfluid helium, generate magnetic fields above 8 T and serve to steer and focus the particle beams (either protons or lead ions) in their trajectory through the accelerator rings.

Protons, derived by ionizing hydrogen gas, are accelerated first via a linear accelerator to an energy of 50 MeV and injected into the Proton Synchrotron Booster, which accelerates them to 1.4 GeV. From there, they are injected into the Proton Synchrotron, which accelerates them further to 25 GeV, and then into the Super Proton Synchrotron, which brings them to an energy of 450 GeV before finally injecting them into the LHC ring, in which they are accelerated to the desired center-of-mass energy for collisions. The LHC has been designed to collide protons at a maximum center-of-mass energy of 14 TeV. During its first run, it operated at center-of-mass energy 7 TeV from 2010-2012, and then at 8 TeV from 2012-2013. After a long shutdown for planned repairs and maintenance, the LHC was restarted in 2015, operating at 13 TeV, with the intention of eventually increasing to 14 TeV.

The LHC is home to several experiments -- the general-purpose high-luminosity detectors ATLAS (A Toroidal LHC Apparatus) and CMS (Compact Muon Solenoid), low-luminosity detectors LHCb (focusing on B physics) and TOTEM (focusing on elastic proton scattering), and the heavy-ion detector ALICE. The rest of this chapter will focus on the CMS experiment, where the research described in this thesis was conducted.

\section{Compact Muon Solenoid\label{sec:cms}}

The Compact Muon Solenoid (CMS) detector is a hermetic general-purpose detector at the LHC, gathering data from the collision of proton-proton and heavy ion beams to study a wide range of physics processes. This experiment is characterized by a powerful superconducting solenoid magnet that produces a 4 T magnetic field; the paths of charged particles are bent by the magnetic field, allowing their momenta to be accurately reconstructed from their trajectories. Quadrupole magnets bend the two proton beams passing through the LHC ring to intersect and collide at the interaction point (IP) in the center of the CMS detector, and the particles produced in the collision pass through and are recorded by the layered system of cylindrical subdetectors that comprise the CMS detector. At the center of the ensemble is a silicon tracker subdetector for reconstructing accurately the momenta of charged particles. Encircling it is the electromagnetic calorimeter, a homogenous calorimeter that uses scintillating lead tungstate crystals to reconstruct electrons and photons with high energy resolution. Outside the electromagnetic calorimeter is the hadron calorimeter, a sampling calorimeter with alternating layers of brass and scintillator that measures the energy deposited in hadronic showers. The solenoid magnet is located outside the hadronic calorimeter, together with an iron yoke system that provides a return flux for the magnetic field. The outermost layer of the CMS detector is the muon tracking system for identifying and reconstructing the trajectories of muons. The solenoid magnet and the various subdetectors of CMS will be described in more detail in the rest of this chapter, followed by considerations regarding the mechanisms for data collection by CMS. A more complete description of the detector can be found at~\cite{1748-0221-3-08-S08004}; unless otherwise specified, all information in this chapter is derived from this source.

\subsection{Tracker\label{sec:cmstracker}}

\subsection{Electromagnetic calorimeter\label{sec:cms-ecal}}

\subsection{Hadronic calorimeter\label{sec:cms-hcal}}

\subsection{Magnet\label{sec:cms-magnet}}

\subsection{Muon system\label{sec:cms-muon}}

\section{Data acquisition and analysis at CMS\label{sec:cms-daq}}

\subsection{Triggers\label{sec:cms-triggers}}

\subsection{Event reconstruction\label{sec:cms-reco}}