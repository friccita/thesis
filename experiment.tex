\chapter{Experiment description\label{sec:experiment}}

\section{Large Hadron Collider\label{sec:lhc}}

The Large Hadron Collider (see~\cite{1748-0221-3-08-S08001} for a detailed description) is a circular accelerator and collider of high-energy particles, straddling the French-Swiss border near Geneva, Switzerland. Constructed between 1998 and 2008, the accelerator consists of 2 rings for counter-circulating proton or ion beams, 27 km in circumference and located at depths as low as 175 m underground in the tunnel previously occupied by the LEP collider ring. Superconducting magnets, composed of NbTi Rutherford conductors cooled to temperatures below 2 K with superfluid helium, generate magnetic fields above 8 T and serve to steer and focus the particle beams (either protons or lead ions) in their trajectory through the accelerator rings.

Protons, derived by ionizing hydrogen gas, are accelerated first via a linear accelerator to an energy of 50 MeV and injected into the Proton Synchrotron Booster, which accelerates them to 1.4 GeV. From there, they are injected into the Proton Synchrotron, which accelerates them further to 25 GeV, and then into the Super Proton Synchrotron, which brings them to an energy of 450 GeV before finally injecting them into the LHC ring, in which they are accelerated to the desired center-of-mass energy for collisions. The LHC has been designed to collide protons at a maximum center-of-mass energy of 14 TeV. During its first run, it operated at center-of-mass energy 7 TeV from 2010-2012, and then at 8 TeV from 2012-2013. After a long shutdown for planned repairs and maintenance, the LHC was restarted in 2015, operating at 13 TeV, with the intention of eventually increasing to 14 TeV.

The LHC is home to several experiments -- the general-purpose high-luminosity detectors ATLAS (A Toroidal LHC Apparatus) and CMS (Compact Muon Solenoid), low-luminosity detectors LHCb (focusing on B physics) and TOTEM (focusing on elastic proton scattering), and the heavy-ion detector ALICE. At each detector, the proton beams are focused together by quadrupole magnets at an interaction point, where they collide, and the particles produced in these collisions are detected and recorded by the detector system. The rest of this chapter will focus on the CMS experiment, where the research described in this thesis was conducted.

\section{Compact Muon Solenoid\label{sec:cms}}

The Compact Muon Solenoid (CMS) detector is a hermetic general-purpose detector at the LHC, gathering data from the collision of proton-proton and heavy ion beams to study a wide range of physics processes. This experiment is characterized by a powerful superconducting solenoid magnet that produces a 4 T magnetic field; the paths of charged particles are bent by the magnetic field, allowing their momenta to be accurately reconstructed from their trajectories. Quadrupole magnets bend the two proton beams passing through the LHC ring to intersect and collide at the interaction point (IP) in the center of the CMS detector, and the particles produced in the collision pass through and are recorded by the layered system of cylindrical subdetectors that comprise the CMS detector. At the center of the ensemble is a silicon tracker subdetector for reconstructing accurately the momenta of charged particles. Encircling it is the electromagnetic calorimeter, a homogenous calorimeter that uses scintillating lead tungstate crystals to reconstruct electrons and photons with high energy resolution. Outside the electromagnetic calorimeter is the hadron calorimeter, a sampling calorimeter with alternating layers of brass and scintillator that measures the energy deposited in hadronic showers. The solenoid magnet is located outside the hadronic calorimeter, together with an iron yoke system that provides a return flux for the magnetic field. The outermost layer of the CMS detector is the muon tracking system for identifying and reconstructing the trajectories of muons. The solenoid magnet and the various subdetectors of CMS will be described in more detail in the rest of this chapter, followed by considerations regarding the mechanisms for data collection by CMS. A more complete description of the detector can be found at~\cite{1748-0221-3-08-S08004}; unless otherwise specified, all information in this chapter is derived from this source.

\subsection{Tracker\label{sec:cms-tracker}}

5.8 m in length along the z axis and 2.5 m in diameter, the tracker detector is the innermost layer of the CMS detector system. Its purpose is to reconstruct the paths and momenta of charged particles with transverse momentum 1 GeV and upwards, and to provide good impact parameter resolution for accurately reconstructing the positions of secondary vertices, which are displaced from the point of collision and generally are a characteristic signature of long-lived particles such as those from heavy-flavour processes. Since the tracker detector receives higher irradiation than any other part of the detector due to its proximity to the beam line and the interaction point (IP), the tracker detector has been designed to perform in a high-flux environment, with thousands of particles passing through its volume every 25 ns when the LHC is running at its design luminosity. Thus, the tracker has been designed with these challenges in mind in order to yield good position and time resolution.

The tracker detector is composed of two subdetectors. The one closest to the beam line is the pixel detector; its sensors are 100 $\mu$m x 150 $\mu$m pixels, which receive an occupancy on the order of $10^{-4}$ per pixel per bunch crossing. The second and larger subdetector is the silicon strip tracker, whose sensors are silicon strips; since it is located at larger radii than the pixel detector, the fluence of particles that reach it is lower and thus the granularity of the silicon strips can be considerably lower than that of the pixel detector, with strip areas ranging from 10 cm x 80 $\mu$m at intermediate radii for the inner strip tracker (20 $<$ r $<$ 55 cm) and 25 cm x 180 $\mu$m for the outer strip tracker (55 $<$ r $<$ 110 cm), resulting in an occupancy of about 2-3\% for the inner tracker and 1\% for the outer tracker. In total, the CMS tracker detector is comprised of 200 $m^{2}$ of active silicon sensors, providing a coverage of up to $\abs{\eta}$ $<$ 2.5.

The support structures that hold the sensors in position are designed to minimize the amount of material used, since energy loss via multiple scattering and extraneous particles produced by nuclear interactions, gamma conversions, and bremsstrahlung in the supporting material can all interfere with tracking efficiency. Because of the high particle fluence passing through the tracker volume during operation, radiation-hard sensors and electronics are required to withstand the radiation dosage; also, an efficient system system of cooling tubes carrying chilled liquid $C_{6}F_{14}$ pervades the tracker volume, keeping it at or below -10 C during operation, thus minimizing radiation damage to the sensors caused either by direct irradiation or by the annealing of radiation-induced defects in the silicon crystal structure through thermal agitation.

\subsubsection{Pixel detector\label{sec:cms-pixel}}
% barrel and disks, design of sensors, electronics, readout

The pixel detector is the innermost layer of the tracker detector, with three concentric cylindrical barrel layers complemented by two endcap disks on either side of the interaction point. The barrel layers are located at radii 4.4, 7.3, and 10.2 cm from the beam line, extending out to 2.9 m from the interaction point in the $\pm$z directions. Two endcap disks are located at z $=$ $\pm$34.5 cm and $\pm$46.5 cm, with inner radius 6 cm and outer radius 15 cm. Both the barrel cylinders and endcap disks are split in halves along the y axis for ease of extraction and access. Each sensor has an area of 100 $\mu$m x 150 $\mu$m; the nearly square design is intended to provide good cluster size and hit resolution in both r$\phi$ and z for the barrel and r$\phi$ and r for the endcaps, as both of these coordinates are needed for measuring track impact parameters.

In the barrel, the sensors are arranged 2 x 8 on rectangular modules (or 1 x 8 along the edges of the half-cylinders). In each endcap half-disk, 12 trapezoidal support structures called blades hold the sensors, which come in arrangements of five different plaquette sizes (1 x 2, 2 x 3, 2 x 4, 1 x 5, and 2 x 5) in order to achieve full coverage of the wedge-shaped area of the blade. Altogether, the pixel detector covers a pseudorapidity range of $\abs{\eta}$ $<$ 2.5.

The pixel sensors consist of 52 x 80 high-dose $n$-type pixels implanted in a high-resistance $n$-type substrate of 320 $\mu$m thickness on a sensor plate. Each pixel is bump-bonded onto a readout chip (ROC) connected to the sensor plate. The ROC collects and amplifies the analog signals from the pixels, storing them in a buffer until the arrival of the appropriate readout control and clock signals causes it to pass the sensor signal on to the readout system.
 
When a charged particle passes through a silicon sensor, it induces charge carriers in the n-doped silicon substrate; traveling towards the pixels to be collected, electrons undergo a signficant Lorentz drift (roughly 32 degrees) due to the 4 T magnetic field along the z axis through the tracker volume, and thus the signal current ends up being spread over adjacent pixels. Interpolation between signals from multiple neighbouring pixels can be used to improve hit resolution. In the barrel, the normal direction of the pixel cells points along the radial direction, perpendicular to the magnetic field, so the Lorentz drift is along the r$\phi$ direction. To induce charge sharing in the pixel endcaps, whose normal points along the magnetic field axis, the blades are rotated at a 20 degree angle about their radial axis in a turbine-like geometry.

\subsubsection{Silicon strip detector\label{sec:cms-strips}}
% barrel and disks, design of sensors, electronics, readout

Outside the pixel detector, at radius 20 cm to 116 cm about the beam line and extending 118 cm in the +z and -z directions, lies the silicon strip detector. It is composed of 3 main subsystems: the tracker inner barrel and disks (TIB and TID), the tracker outer barrel (TOB), and the tracker endcaps (TEC). To optimize coverage, the silicon microstrip sensors in any given barrel layer or endcap disk are positioned to partially overlap with one another, thus resulting in a nonzero pitch with respect to the surface to which they are attached.

The TIB consists of 4 barrel layers 140.0 cm in length, with radii of 255.0 mm, 339.0 mm, 418.5 mm, and 498.0 mm, altogether providing up to 4 r-$\phi$ measurements per particle trajectory. The sensors are silicon microstrips with a thickness of 320 $\mu$m, lying parallel to the beam axis, with a pitch of 80 $\mu$m on the inner two layers and 120 $\mu$m on the outer two layers, yielding a single-point resolution of 23 $\mu$m for the inner two layers and 35 $\mu$m for the outer two layers. The TID is comprised of six endcap disks, three at each end of the TIB between $\pm$80.0 cm and $\pm$90.0 cm on the z axis; each disk consists of three support rings from radius 200 $\mu$m to 500 $\mu$m. Silicon microstrips, similar to the ones used in the barrel, lie radially on the disks, with a pitch varying from 100 $\mu$m to 141 $\mu$m. While the pixel detector is split down the y axis into half-cylinders for ease of installation, access, and independent testing, the TIB and TID are split into half-shells along the x axis for similar reasons. Two carbon-fibre service cylinders are coupled to the $\pm$z ends of the TIB, providing a route to the TIB shells for service cables originating from a service distribution disk called a margherita, and also housing the TID.

The TOB surrounds the TIB with 6 barrel layers, reaching to an outer radius of 116 cm and spanning 118 cm in the $\pm$z directions. The silicon microstrip sensors here are 500 $\mu$m thick and have a pitch of 183 $\mu$m in the inner four layers and 122 $\mu$m in the outer two layers, yielding a single-point resolution of 53 $\mu$m and 35 $\mu$m respectively in those layers. On either end of the TOB, the TEC extends radially from 220 mm to 1135 mm and in the $\pm$z direction from $\pm$1240 mm to $\pm$2800 mm; each side consists of an assembly of 9 disks with up to 7 rings bearing silicon microstrip sensors, plus 2 extra disks that serve as front-back termination. The microstrip sensors used here have a thickness of 320 $\mu$m in the four rings closest to the TOB and 500 $\mu$m in the remaining outer rings, with a pitch varying from 97 to 184 $\mu$m.

The silicon microstrip sensors used have a single-sided p-on-n design. Signal currents are amplified and stored by a custom integrated circuit called an APV25 before being transmitted via optical fibers through the readout system for digitization and storage.

\subsection{Electromagnetic calorimeter\label{sec:cms-ecal}}
% crystals, photodetectors, readout, preshower

\subsection{Hadronic calorimeter\label{sec:cms-hcal}}
% HB, HE, HO, HF -- scintillator, absorber, readout

\subsection{Magnet\label{sec:cms-magnet}}
% solenoid, yoke, cryogenics, vacuum system

The superconducting solenoid magnet of CMS encloses the calorimeters and tracker detector inside a cylindrical bore 6.3 m in diameter and 12.5 m long, weighing 220 tonnes. The solenoid is composed of a 4-layer winding of coils made of NbTi, which is mechanically reinforced by being mixed with an aluminium alloy, an innovative method that makes the coils serve both as conductors and as their own self-stabilizing structural support. At full current, the solenoid carries 19.14 kA, producing a nearly uniform 4 T magnetic field through the volume enclosed by the bore, containing 2.6 GJ of stored energy. The magnetic flux is returned through an iron yoke, consisting of a system of barrel wheels and endcap disks arranged outside the volume of the solenoid in a cylindrical pattern and interspersed with the muon tracking system, which will be described in the next section.

\subsection{Muon system\label{sec:cms-muon}}
% DT, CSC, RPC, readouts for all

\section{Triggers and data acquisition\label{sec:cms-triggerdaq}}
% rationale, L1, HLT; muon, calorimeter, global

\subsection{Event reconstruction\label{sec:cms-reco}}
% Particle Flow (list all types of particles, focus on muons, taus, jets, MET)