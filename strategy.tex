\chapter{Analysis strategy\label{sec:strategy}}

\section{Search signature\label{sec:signature}}
The signal studied in this analysis is the production of an SM-like Higgs boson H followed by its decay to a pair of lighter pseudoscalar Higgs bosons $a$, each of which decays to a boosted pair of taus.  Three production channels (Figure~\ref{fig:signatures}) for the H are considered: vector boson associated production (WH and ZH), in which the vector boson decays to a high-$p_T$ isolated muon that provides a convenient trigger, gluon fusion (ggH), and vector boson fusion (VBF).  The analysis was originally optimized for the WH mode but is sensitive to the ggH+VBF mode due to its large cross section. Since no forward jet tagging is done, the analysis is only sensitive to the sum of ggH and VBF, not each mode individually.  One of the $\tau\tau$ pairs is identified via the $\tau_{\mu}\tau_{\text{had}}$ decay topology, while no selection is made on the other $\tau\tau$ pair.  In order to reconstruct the boosted $\tau\tau$ pairs, a modified version of the standard hadron plus strips (HPS)~\cite{CMS:2011msa} tau reconstruction procedure was developed for this analysis. The most significant backgrounds to the signal are expected to be SM W and Drell-Yan production, where the W and Z decay to muons; t\={t} with one or two muons in the final state; and heavy flavor QCD. A sequence of optimized selections is determined to maximize the search sensitivity for the two signal production channels, and a counting experiment is then performed.

\section{Motivations\label{sec:motivations}}

\subsection{Light pseudoscalars\label{sec:lighta}}

\subsection{Semileptonic di-tau decays\label{sec:semileptonic}}