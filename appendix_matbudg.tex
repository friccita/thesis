\chapter{Pixel detector geometry description\label{sec:matbudg}}

The CMS detector simulation uses an XML schema called the Detector Description Language (DDL) to encode the description of the detector geometry and material composition~\cite{CMS_AN_2005-000}. Together with two other auxiliary packages, the Algorithm Description Language (ADL) and Configuration Description Language (CDL), DDL interfaces with GEANT4 to provide the volumes, positions, and material compositions of the simulated detector elements.

All the various components and subcomponents of the CMS detector form a geometrical hierarchy, with each individual object being a subcomponent of some larger whole. In a system of XML files, DDL defines the basic data structures for describing the dimensions and materials of these parts as well as their geometrical hierarchies.

Two different classes of material definitions exist in DDL. Elementary materials correspond to elements of the periodic table and are identified by name, periodic table symbol, density, atomic number, and atomic weight. Definitions of composite materials are built by specifying their fractional composition in terms of elementary materials, or even other composite materials; the data structure allows one to customize the density associated with a particular composite material. The radiation length and hadronic interaction length of composite materials are calculated from the material definitions in the XML files; these parameters are needed ifor the modelling of particle interactions with detector components.

Detector parts are defined by their material, their shape, and their position in the detector. The various types of 3-dimensional shapes (such as rectangular boxes, trapezoids, and cylinders) allowed in this package are based on the GEANT syntax. Various parameters such as angles and Cartesian coordinates encode a detector component's spatial position, often with respect to a larger structure of which it is a component. Algorithms from ADL, referred to as DDAlgorithms, can be used to position multiple copies of a detector component in a specific pattern, to represent symmetrical or repeating structures.

The rest of this chapter treats the projects involving the CMS pixel detector geometry description, in which I have participated.

\section{Pilot system simulation\label{sec:matbudg-pilot}}
%1. Pilot system simulation (mention of Phase I upgrade, LS1, installation of pilot system, how it was added to the description, figures)

From February 2013 to February 2015, the LHC was turned off. During this period of planned off-time, repairs and maintenance were done on the CMS detector. The pixel detector was extracted from the experimental cavern in order for problematic panels and electronics to be fixed or replaced. While it was out, one extra endcap disk was installed on the -z side of the forward pixel detector (FPIX); 8 prototype modules for the planned Phase I upgrade~\cite{Dominguez:1481838} were mounted on the blades of this extra disk. Circuit boards called portcards were also installed, for reading out the prototype modules. All together, the extra disk and its associated cables and electronics make up the pilot system.

The tracker geometry and material description at the time did not include a description of the pilot system. In order to have an accurate representation of the material distribution (often referred to as the material budget) in the detector for generating MC events with this new configuration of the pixel detector, the geometry description was updated accordingly. A disk on the -z endcap was cloned into the position where the new disk was installed; new objects were defined in DDL to represent the shape and material of the new modules and their support structures. The large uniform blocks roughly representing the FPIX portcard electronics in the service cylinder also were updated, since the material composition had changed due to the addition of copper-containing DC-DC converters and other new electronics for the pilot system's readout. A new composite material representing the new average material composition was defined.

\section{Phase I pixel simulation\label{sec:matbudg-phase1}}
%2. Updates to Phase I description (ongoing; mostly done, describe what is being corrected)

During the next long shutdown at the end of the LHC's Run II, the CMS detector will undergo another round of detector upgrades known as the Phase I upgrade. The pixel detector will acquire one new barrel layer and one new endcap on each side of the interaction point. The cooling system, which currently uses C$_{6}$F$_{14}$ as a coolant, will be replaced with a new cooling system that uses cold carbon dioxide instead. The geometry of the forward pixel endcap support structures will be radically changed, in order to decrease the material budget while still providing full coverage with no dead space.

The CMS geometry description currently has XML files that describe the Phase I pixel geometry and material. The barrel pixel part is mostly accurate aside from a few minor changes involving the dimensions of some support structures and the addition of aluminium cabling in some regions. However, the forward pixel part is very inaccurate and needs significant updating in order to represent the actual upgraded components that will be installed. The main challenges are the following:

\begin{itemize}
\item The current description of the FPIX endcap support rings are flat, uniform rings, whereas the actual support rings have a zigzagging shape. This is extremely difficult to render using the shapes allowed in DDL. A tentative solution is being developed and tested, in which a series of ``infinitesimally" thin blocks are positioned in the form of a zigzag-shaped ring using a DDAlgorithm that gives each block an appropriate displacement along z as a function of $\phi$.
\item The zigzag pattern of the blades in the rings is incorrect for some of the rings; comparisons with blueprints of the actual Phase I FPIX rings are currently underway to determine the correct pattern. Also, in simulation, the endcap disks on the +z side are obtained by rotating the -z disks about the y axis, whereas the actual Phase I disks have a mirror symmetry about the xy plane instead.
\item The shapes of the modules and their support structures need to be refined.
\item The composite material used to describe the Phase I portcard objects is the same material used to describe the portcard objects in the Run I geometry before the installation of the pilot system. This is clearly incorrect, and a new composite material needs to be declared that matches the average composition of the actual Phase I FPIX portcards and DC-DC converters.
\end{itemize}

I have been involved in this project since spring 2015, and it is still ongoing.