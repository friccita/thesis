\documentclass[12pt,cls]{ucdavisthesis}

% PLEASE READ THE MANUAL - ucdavisthesis.pdf (in the package installation directory)

%%%%%%%%%%%%%%%%%%%%%%%%%%%%%%%%%%%%%%%%%%%%%%%%%%%%%%%%%%%%%%%%%%%%%%%%
%                                                                      %
%               LATEX COMMANDS FOR DOCUMENT SETUP                      %
%                                                                      %
%%%%%%%%%%%%%%%%%%%%%%%%%%%%%%%%%%%%%%%%%%%%%%%%%%%%%%%%%%%%%%%%%%%%%%%%

\usepackage[
  % Some remarks:
  % * drivers like 'pdftex' that can be detected automatically
  %   are not necessary
  % * breaklinks is rather an internal option.
  %   If a driver does not support it, then forcing the option
  %   let the text break across lines, but also the link
  %   areas are "broken". If the driver supports the option,
  %   then the option is enabled anyway.
  % * Information entries should be set outside,
  %   because LaTeX expands the package options,
  %   hyperref does not like them, if they are
  %   prematurely expanded.
  % * Hyperref has a new option for hiding links: hidelinks
  hidelinks,
  letterpaper,
  pagebackref,
  bookmarksopen,
  bookmarksnumbered,
]{hyperref}
\usepackage{bookmark}
\usepackage[toc,page]{appendix}
\usepackage[us,nodayofweek,12hr]{datetime}
\usepackage{graphicx}
\usepackage{siunitx}
\usepackage{rotating}
\usepackage{supertabular}
\usepackage{multirow}
%\usepackage[square,comma,numbers,sort&compress]{natbib}
\usepackage{hypernat}
% Other useful packages to try
\usepackage{amsmath}
\usepackage{amssymb}
\usepackage{doi}
\usepackage{lineno}
%
% Different fonts to try (uncomment only fontenc and one font at a time)
% (you may need to install these first)
%\usepackage[T1]{fontenc} %enable fontenc package if using one of the fonts below
%\usepackage[adobe-utopia]{mathdesign}
%\usepackage{tgschola}
%\usepackage{tgbonum}
%\usepackage{tgpagella}
%\usepackage{tgtermes}
%\usepackage{fourier}
%\usepackage{fouriernc}
%\usepackage{kmath,kerkis}
%\usepackage{kpfonts}
%\usepackage[urw-garamond]{mathdesign}
%\usepackage[bitstream-charter]{mathdesign}
%\usepackage[sc]{mathpazo}
%\usepackage{mathptmx}
%\usepackage[varg]{txfonts}
%packages for CMS symbols
\usepackage{xspace}
\usepackage{ptdr-definitions}
\hyphenation{dis-ser-ta-tion blue-print man-u-script pre-par-ing calo-ri-me-ter so-le-noid brems-strahl-ung} %add hyphenation rules for words TeX doesn't know
\newlength\cmsFigWidth
\setlength\cmsFigWidth{0.85\columnwidth}
\setlength\cmsFigWidth{0.4\textwidth}
\providecommand{\cmsLeft}{Left}
\providecommand{\cmsRight}{Right}


%\renewcommand{\rightmark}{\scriptsize A University of California Davis\ldots \hfill Rev.~\#1.0 \quad Compiled: \currenttime, \today}
% a fancier running header that can be used with draftcls options

%%%%%%%%%%%%%%%%%%%%%%%%%%%%%%%%%%%%%%%%%%%%%%%%%%%%%%%%%%%%%%%%%%%%%%%%
%                                                                      %
%        DOCUMENT SETUP AND INFORMATION FOR PRELIMINARY PAGES          %
%                                                                      %
%%%%%%%%%%%%%%%%%%%%%%%%%%%%%%%%%%%%%%%%%%%%%%%%%%%%%%%%%%%%%%%%%%%%%%%%

%\title          {A University of California Davis\\
%                 Dissertation/Thesis LaTeX Class File}
\title          {Search for New Light Higgs Bosons in Boosted\\
                 Tau Final States with the CMS Experiment}
%Exact title of your thesis. Indicate italics where necessary by underlining or using italics. Please capitalize the first letter of each word that would normally be capitalized in a title.

\author         {Francesca Shun-Ning Annarosa Ricci-Tam}
%Your full name as it appears on University records. Do not use initials.

\authordegrees  {B.S. (University of California, Davis) 2010 \\
                 M.S. (University of California, Davis) 2012}
%Indicate your previous degrees conferred.

\officialmajor  {Physics}
%This is your official major as it appears on your University records.

\graduateprogram{Physics}
%This is your official graduate program name. Used for UMI abstract.

\degreeyear     {2016}
% Indicate the year in which your degree will be officially conferred.

\degreemonth    {March}
% Indicate the month in which your degree will be officially conferred. Used for UMI abstract.

\committee{Maxwell Chertok}{Robin Erbacher}{Michael Mulhearn}{}{}
% These are your committee members. The command accepts up to five committee members so be sure to have five sets of braces, even if there are empties.

%%%%%%%%%%%%%%%%%%%%%%%%%%%%%%%%%%%%%%%%%%%%%%%%%%%%%%%%%%%%%%%%%%%%%%%%

\copyrightyear{2016}
%\nocopyright

%%%%%%%%%%%%%%%%%%%%%%%%%%%%%%%%%%%%%%%%%%%%%%%%%%%%%%%%%%%%%%%%%%%%%%%%

%\dedication{\textsl{To my family.}}

%%%%%%%%%%%%%%%%%%%%%%%%%%%%%%%%%%%%%%%%%%%%%%%%%%%%%%%%%%%%%%%%%%%%%%%%

\abstract{In this dissertation, I present a search for non-standard decays of a Standard Model-like Higgs boson to pairs of light bosons, as predicted in models with extended Higgs sectors. In two Higgs doublet models, including the next-to-minimal supersymmetric standard model, the Higgs boson can decay into a pair of light pseudoscalars $a$.  In this search, the gluon fusion, $W$ and $Z$ associated Higgs, and vector boson fusion production channels for the Higgs are all considered, and the decay $H\rightarrow$$aa$ with $a\rightarrow\tau\tau$ is reconstructed from the tau decay products. The final state is characterized by one isolated high $p_T$ muon plus at least one highly boosted pair of taus, of which one of the taus is required to decay to a muon. Using 19.7 fb$^{-1}$ of 8 TeV center of mass $pp$ collision data recorded by the Compact Muon Solenoid experiment at the Large Hadron Collider, a counting experiment is performed in a region of high di-tau invariant mass. We have found no excess of events above the Standard Model backgrounds, and the observed data is used to set upper limits on the branching ratio Br($H\rightarrow$$aa$)Br$^{2}(a\rightarrow\tau\tau)$. These results are equally applicable to decays of the SM-like Higgs boson to a pair of light scalars $h$.}

%%%%%%%%%%%%%%%%%%%%%%%%%%%%%%%%%%%%%%%%%%%%%%%%%%%%%%%%%%%%%%%%%%%%%%%%

\acknowledgments{I would like to thank my graduate advisor, Maxwell Chertok, for his patient help, advice, and friendship during my development as a high-energy physics researcher; and also Robin Erbacher and Michael Mulhearn for serving on my thesis committee. My appreciation extends to the rest of the UC Davis CMS group as well, for their welcoming and supportive working atmosphere.

During my years at CERN, I have learned a lot from working with my colleagues. There was Rachel Yohay, whose mentorship taught me much of what I know about data analysis techniques and CMS software as we worked together on what was to be my thesis research. Mauro Dinardo, Martina Malberti, and Gino Bolla all patiently gave me their expert advice and answered my questions during my hardware work with the CMS forward pixel detector. I am grateful to Dan Duggan and Gaelle Boudoul for teaching me about pixel offline software, Petra Merkel and Yana Osborne for getting me started in my first pixel geometry simulation project, and Viktor Veszpr\'{e}mi for his guidance during my tracker material budget group convenership.

I want to express my gratitude and appreciation to the friends that I have met along my path, who are too many to name here. And finally, I dedicate this thesis to my family. -- To my father, who has always taught me to have confidence in my own intellect and logic, and who proved to me that the mind of a physicist can tackle anything. To my mother, who has encouraged my interest in reading and the study of nature. To my sisters, who have constantly given me inspiration with their boldness, creativity, and curiosity as we grew up together.}
%I would like to thank my graduate advisor, Maxwell Chertok, for all his patient help, advice, and friendship during my development as a high-energy physics researcher; and also Robin Erbacher and Michael Mulhearn for serving on my thesis committee. My appreciation extends to the rest of the UC Davis CMS group as well, for their welcoming and supportive working atmosphere.
%During my years at CERN, I have learned a lot from working with my colleagues. There was Rachel Yohay, whose mentorship taught me so much about physics analysis techniques and CMS software as we worked together on what was to be my thesis research. Mauro Dinardo, Martina Malberti, and Gino Bolla all patiently gave me their expert advice and answered my questions as we worked with the CMS forward pixel detector. I am grateful to Dan Duggan and Gaelle Boudoul for teaching me about pixel offline software, Petra Merkel and Yana Osborne for getting me started in my first pixel geometry simulation project, and Viktor Veszpr\'{e}mi for his guidance during my tracker material budget group convenership.
%I want to express my gratitude and appreciation to many of the friends that I have met along my path -- to name a few: Valentina and Yuliya Prilepina, Roxanne Moran, C\'{e}cile Caillol, Andrzej \.{Z}ura\'{n}ski, Xiaohang Quan, Mahboobeh Yarmohammadi Satri, Nataliia Kondrashova, Ping Yang, Juliana Froggatt, Aurelijus Rinkevicius, Judita Mamuzic, Christine McLean, and Fanbo Meng.
%Most of all, my family has shaped me personally and intellectually into who I am now, and the appreciation that I express to them can never be adequate. I dedicate this thesis to all of them. To Pap\`{a}, who has always taught me to have confidence in my own intellect and logic, and who proved to me that the mind of a physicist can tackle anything. To Mammina, who has encouraged my fascination with nature and my interest in reading and art. To my sisters Daniela and Chiara, who have constantly given me inspiration with their boldness, creativity, and curiosity as we grew up together.

%%%%%%%%%%%%%%%%%%%%%%%%%%%%%%%%%%%%%%%%%%%%%%%%%%%%%%%%%%%%%%%%%%%%%%%%
\preface{Elementary particle physics is the study of the most fundamental components of matter: determining what these components are, and how they behave. The word ``atom" is derived from the ancient Greek word for ``indivisible"; the discovery that atoms are not in fact indivisible but are themselves made up of smaller components -- electrons, protons, and neutrons -- set the stage for the ongoing search for ever-smaller and more fundamental building blocks of matter.

Decades of research have culminated in the present day with the Standard Model of fundamental physics. The search for the Higgs boson, the last major puzzle piece of the Standard model, carried on for many years,  The most recent validation of the Standard Model was the discovery of the Higgs boson in 2012, which resolved the longstanding paradox of electroweak gauge boson masses. I feel fortunate to have started my work at CERN just at that exciting time when the Higgs boson discovery was announced, and when much work still lay ahead to uncover what lies beyond the Standard Model.

Various theories of physics beyond the Standard Model (BSM) have been developed, and they involve many unknown and interdependent parameters. Just as the Higgs boson mass was once unknown and had to be narrowed down by years of experimental research and data analysis, the parameters of BSM physics theories are still in the process of being increasingly constrained by experimental data, but there is a wide possible range of values open to them. This means that many different models of BSM physics, based on different combinations of assumptions for these parameters, still remain to be experimentally tested.

The collaboration of thousands of researchers at the European Organization for Nuclear Research (CERN) are currently engaged in analyzing data from the Large Hadron Collider for signals of processes predicted by the many models of BSM physics that there are. The research done for this dissertation has been part of that effort -- my little contribution to our collective chipping-away at the unknown, inching towards a clearer picture of fundamental physics.}

%%%%%%%%%%%%%%%%%%%%%%%%%%%%%%%%%%%%%%%%%%%%%%%%%%%%%%%%%%%%%%%%%%%%%%%%

% Each chapter can be in its own file for easier editing and brought in with the \include command.
% Then use the \includeonly command to speed compilation when working on a particular chapter.
%%% \includeonly{ucdavisthesis_example_Chap1}
\begin{document}

\newcommand{\bibfont}{\singlespacing}
% need this command to keep single spacing in the bibliography when using natbib
\bibliographystyle{unsrturl}
%\bibliographystyle{plainnat}
%\bibliographystyle{fdm}
%many other bibliography styles are available (IEEEtran, mla, etc.). Use one appropriate for your field.

\makeintropages %Processes/produces the preliminary pages
\linenumbers
\chapter{Introduction\label{sec:intro}}

%Standard Model
% - What does it consist of (particles, interactions, gauge fields)
% - Description of Higgs mechanism and its significance
% - Examples of SM limitations (DM, gravity, Higgs loop corrections...)

%Supersymmetry
% - What is supersymmetry, what SM problems does it address
% - 2HDM, NMSSM: Motivations
% - 

\subsection{The Standard Model\label{sec:SM}}

The Standard Model is a quantum field theory that provides the most successful description to date of all experimentally observed fundamental particles and their interactions. In this framework, the fundamental particles are all treated as excitations of quantum fields, and interactions between particles occur via the exchange of mediating particles~\cite{BettiniPhysics}.
Fundamental particles can be classified into three main categories: leptons, quarks, and gauge bosons. The leptons and quarks are all fermions -- particles with half-integer spin whose dynamics obey the Dirac equation. Leptons have integer electric charge and fall into three categories called ``flavours", each of which can be regarded as a representation of the SU(2)xU(1) group. The negatively charged leptons are the electron, the muon, and the tau (in increasing order of mass), and each is associated with a massless neutrino; for each lepton, there is also an associated antiparticle. Quarks have fractional electric charge and fall into three categories known as ``colour", which can be regarded as representations of the SU(3) group.
Particles interact via four fundamental forces: the strong force, electromagnetism, the weak force, and gravity. These interactions are mediated by the third category of fundamental particle, the gauge bosons. The strong force is mediated by gluons, which are colorless, electrically neutral, and massless; only quarks and gluons can participate in strong interactions. Any charged particle can participate in electromagnetic interactions, which are mediated by colorless, electrically, neutral, massless photons. The weak force, which is responsible for nuclear decays, is mediated by the W bosons, which can have positive or negative charge, and Z bosons, which are electrically neutral.
An illustration of the classification of Standard Model particles is shown in Fig.~\ref{fig:StandardModelTable}.

\begin{figure}
   \begin{center}
      \includegraphics[width=0.4\textwidth]{figures/StandardModelTable}
      \caption{Standard Model particles.}
      \label{fig:StandardModelTable}
   \end{center}
\end{figure}

\subsection{Deficiencies of the Standard Model\label{sec:SMdeficiencies}}


Although the Standard Model has been successful in describing a wide range of experimental results, it also falls short of providing a complete description of nature in many respects. It describes only three out of the four fundamental forces; the way in which gravity, which is $10^{-32}$ times weaker than the weak force, factors into the Standard Model is still unknown. It fails to provide a satisfactory description of neutrino oscillations, an explanation of the matter-antimatter asymmetry in the universe, or a suitable candidate (or candidates) for dark matter.

The Standard Model also predicts that the mass of the Higgs boson receives loop corrections -- corrections of a magnitude close to the Planck scale at which the three fundamental forces are expected to be unified. The fact that the experimentally measured Higgs boson mass is 125 GeV -- orders of magnitude smaller than the corrections -- means that these loop corrections must somehow be cancelled, but it is not known how. This constitutes what is known as the hierarchy problem of particle physics.

\subsection{Supersymmetry}

One theory that has been proposed to address the hierarchy problem is that there exists a certain symmetry, referred to a supersymmetry, that relates fermions to bosons. For each fermion, there would exist a corresponding bosonic parter particle, and likewise for each boson there would exist a fermionic partner; under a supersymmetric transformation, a fermion would turn into its bosonic partner and vice versa. Thus, for each loop correction to the Higgs mass, there would be another loop correction from the supersymmetric partner particle; since fermionic loops have a sign opposite that of bosonic loops, the loop corrections would cancel neatly.

\subsubsection{MSSM}

\subsubsection{NMSSM}

\chapter{Experiment description\label{sec:experiment}}

\section{Large Hadron Collider\label{sec:lhc}}
% - What is the LHC: roperties and purposes

\section{Compact Muon Solenoid\label{sec:cms}}
% - What is CMS

\subsection{Tracker\label{sec:cmstracker}}

\subsection{Electromagnetic calorimeter\label{sec:cmsecal}}

\subsection{Hadronic calorimeter\label{sec:cmshcal}}

\subsection{Magnet\label{sec:cmsmagnet}}

\subsection{Muon system\label{sec:cmsmuon}}

\section{Data acquisition and analysis at CMS\label{sec:cmsdaq}}

\subsection{Triggers\label{sec:cmstriggers}}

\subsection{Event reconstruction\label{sec:cmsreco}}
\chapter{Event reconstruction and simulation\label{sec:recosim}}

\section{Particle reconstruction\label{sec:cms-reco}}
% Particle Flow (list all types of particles, focus on muons, taus, jets, MET)
Before reaching a relatively stable final state, the immediate products of the proton-proton collisions in the LHC tunnel undergo various interactions such as radiation, decays, and/or hadronization. The resulting particles travel through and interact with the material of the CMS detector, which records their passage and reconstructs their paths and energies, which are then used to identify the particles and deduce the interactions that they underwent. The set of algorithms used predominantly for particle reconstruction and identification in the CMS experiment is called Particle Flow~\cite{CMS-PAS-PFT-09-001}, often abbreviated as PF.

The basic building blocks passed to the PF algorithm are tracks and clusters. Charged particle tracks are reconstructed from hits in the silicon tracker by an iterative tracking algorithm; a similar process is done to reconstruct muon tracks in the muon detector. Track resolution is crucial for accurately determining the trajectory as each track, as inaccuracies can lead to large discrepancies in reconstructed energies. Clustering algorithms search for patterns in energy deposits in the ECAL and HCAL, to reconstruct the energies and trajectories of photons, neutral hadrons, and charged hadrons.

Once tracks and clusters are reconstructed, linking algorithms make tentative associations between these elements, based on criteria such as their closeness in distance and in energy. Track paths are extrapolated from the tracker into the ECAL and HCAL and matched to clusters if their reconstructed momenta and positions are compatible; clusters in the ECAL and HCAL are linked if the extrapolated trajectory of the ECAL cluster lies within the HCAL cluster envelope; tracks from the silicon and muon trackers are linked by performing a global $\chi^2$ fit between the two types of tracks.

Particle Flow combines all this track and cluster information from all subdetectors to build reconstructed particle objects. The abbreviation "reco" will often be used in this thesis to refer to reconstructed objects. Charged particle tracks linked to ECAL clusters are used to seed electron and charged hadron candidates, while ECAL and HCAL clusters that cannot be matched to any track are used to seed photon and neutral hadron candidates respectively. If the momentum from the combination of a linked muon track and silicon tracker track is compatible with the silicon tracker track momentum alone, the linked object is a PF muon candidate. Jet clustering algorithms group electron, muon, photon, and charged or neutral hadron candidates together into jets; iterative cone techniques~\cite{1126-6708-2008-04-063} use the hardest particle or calorimeter tower in the event as a seed and builds a jet from the PF candidates in a cone around it, removes all of the jet candidates from consideration, and then moves on to find the next jet seed from the remaining candidates in the event, proceeding thus until no seeds are left. Hadronic tau decays are reconstructed from PF jets; currently the approved algorithm used by CMS is the Hadrons Plus Strips (HPS) algorithm~\cite{CMS:2011msa}, which reconstructs the decay mode of a tau based on the number of charged hadrons, neutral hadrons, and photons among the tau decay candidates. Finally, since the initial-state colliding protons have zero transverse energy, the missing transverse energy in the event is reconstructed by calculating the vector sum of the transverse energies of all reconstructed particles and taking the negative, based on the conservation of momentum.

\section{Event simulation with Monte Carlo\label{sec:cms-sim}}

Simulating particle collisions is an essential part of high-energy collider experiments. For instance, in order to determine whether actual data indicate the detection of new physics, one must know what data to expect if nothing new is occurring, so as to be able to compare collected data with predictions from the old theory; a reliable program for simulating physics processes based on known theory can provide a convenient means of obtaining a prediction for expected backgrounds. Other examples of uses for physics event simulation include calibrating the detector and testing the efficiency of its hardware and software.

The evolution of a simulated event in a collider can be broken down into two main steps: 1) modeling the particle collision and subsequent particle production, and 2) modeling the interaction of the final-state particles with the various parts of the detector, decays in flight, and the detector response. The principles behind these two aspects of event simulation will now be discussed. The physics event generation package PYTHIA~\cite{Sjostrand:2006za} and the detector simulation package GEANT4~\cite{documents:998155}, both used in the Compact Muon Solenoid (CMS) experiment, will be described as concrete examples of event simulation software. Many other packages exist that have somewhat different mechanisms, but the general principles are essentially the same.

For any particle interaction, there exists a spectrum of final states, each with its own particular amplitude for occurring. The phase space describes all the possible final states that the system can achieve; the relative probabilities for these final states are a function of the momenta and relative trajectories of the particles. The evolution of the system involves an element of randomness due to quantum mechanics; the most common technique for simulating this is is the Monte Carlo (MC) method, which uses a random number generator to sample the phase space for each simulated physics process and thus evolve of the event in a probabilistic fashion.

PYTHIA generates physics events in a series of steps. The first step is the hard scattering of the colliding initial-state particles (protons, in LHC event simulations). To generate events with the relative frequencies with which one would expect them to occur in actual colliders, the various possible reaction channels need to be weighted according to their cross-sections, which PYTHIA calculates. The initial-state particles are characterized by parton distribution functions (PDFs); even leptons, which are not partons, are described with a PDF that reflects the likelihood of photon emission by the lepton before it enters the initial hard process with a fraction $x$ of its original momentum.

Photon or gluon radiation can occur before and after the hard scattering process. PYTHIA is optimized to model $2 \rightarrow 1$ and $2 \rightarrow 2$ processes (where the numbers indicate the number of particles in the initial and final states), for which it can compute the cross-sections. However, a challenge arises in simulating radiation processes, which begin with one particle and end with two or more. Gluon radiation is especially problematic because it is governed by QCD, and for soft radiative processes -- where the radiated gluons are roughly collimated with the final-state parton -- the momentum transfer values involved are so low that strong coupling constant is large enough for processes higher than tree level to be significant, and thus amplitudes for these processes cannot be calculated perturbatively. This makes the computation of amplitudes for radiation processes extremely complicated, and PYTHIA does not perform such calculations. Instead, it uses the parton showering method to estimate the higher-order matrix elements for initial and final state radiation. Parton showering simulates the branching of one randomly-chosen ``shower initiation" parton (not necessarily the parton involved in the hard process) into a number of other partons and combining the results. PYTHIA estimates the branching probabilities for quarks, leptons, and gluons with a simplified kinematic model that is a function of the 4-momentum fraction $z$ for the branching; $1 \rightarrow 2$ decays are simulated with this model until the final-state particles in the shower reach a certain cutoff energy, below which no further radiation is simulated. The formation of jets from the beam remnants (the partons not involved directly in the hard scattering) is modeled similarly. Other event generation packages used by CMS, such as MADGRAPH~\cite{springerlink:10.1007/JHEP06(2011)128}, BlackHat, SHERPA~\cite{Berger:1177972}, and POWHEG~\cite{Alioli:2010ab}, do calculate some of the matrix elements for higher-order processes. However, due to the inherent non-perturbativeness of QCD, these calculations necessarily involve their own approximations and inaccuracies.

After the hard scattering and final-state showering, the resultant quarks and gluons must hadronize; in PYTHIA, this process is referred to as fragmentation. Often, the hadrons produced are unstable and will radiate and decay further; the series of fragmentation and decays that occur before the final state is reached are collectively termed hadronization. Again because of the non-perturbative QCD diagrams involved in the matrix elements, PYTHIA relies instead on simplified models of fragmentation based on the Lund string scheme to approximate the process.

The next step in event generation is to model the response of the detector to the particles produced in simulated collisions. At CMS, GEANT4 is used to model the trajectories of final-state particles and their interactions with the parts of the detector in their path, and the way in which the detector elements register and record the particles that pass through them.

The geometry of the detector -- the material composition, positions, dimensions, etc. of its components (both sensitive elements and structural supports) -- must be specified in the simulation program. Monte Carlo methods are used to model the interaction of particles with the detector components. The rate of energy loss per unit distance is determined by the medium's chemical composition and by the type of interaction involved in the energy deposition, which depends on on the particle's energy; for different particle energies, processes such as ionization (governed by the Bethe-Bloch equation) and bremsstrahlung occur with different relative probabilities, characterized by a mean free path (radiation length for bremsstrahlung, and hadronic interaction length for strong interactions)~\cite{Tavernier:1172614}.

When final-state particles emerge from a simulated collision, GEANT4 simulates each particle's trajectory step-by-step through the detector volume, evolving it under the influence of electromagnetic fields and also via its interactions with the materials that it passes through. For each step that a particle makes, all possible types of interactions with the detector material are considered, and cross-sections are computed for all of them; the probability and spacetime step-lengths for each interaction are then calculated, and the smallest step-length is selected; the spacetime position and kinematic properties of the particle are then updated and the particle is ready for the next step to be simulated. Simulated hits in the detector are interpreted by algorithms and clustered together to reconstruct the kinematics and paths of particles in the event.

The detector's efficiency is the frequency with which it correctly records and reconstructs events. To have a realistic picture of the detector's performance, its finite resolution and inefficiencies must be included in the simulation. For any given physics search, the detector efficiency can be considered as a combination of two main factors. The first is the acceptance -- the probability that a simulated particle passing through the simulated detector will be reconstructed. This depends on the geometry of the detector -- where its components and dead space are located -- and on the physics that produced that particle, which determines the probability that it will be produced with the right kinematic properties (e.g., momentum and scattering angle) to pass through the active part of the detector and be recorded. The other factor -- the reconstruction efficiency -- is the probability that the track will be reconstructed in a way that accurately represents the actual particle and its path. This depends on whether the particle satisfies the triggering criteria, and on the selection criteria used to filter out background events. The total efficiency is the product of the acceptance and the reconstruction efficiency~\cite{CMS:2010mua}~\cite{Chatrchyan:2012rga}.

Detector acceptance can be modelled in the GEANT simulation by passing it a database of the calibration constants and detector element efficiencies measured at CMS in calibration studies; these detector conditions are used to correct the representation of particle interactions with individual detector components and better simulate their efficiencies or inefficiencies.

Modelling reconstruction efficiencies is best illustrated by an example. The efficiency for simulated muons to pass a particular HLT path may differ from that of actual muons in data, due to the imperfection of modelling the detector acceptance. The discrepancy may vary with the trigger muon's $p_T$, $\eta$, and other kinematic parameters. Thus, in order to make the simulated HLT efficiency agree better with actual data, official studies are done at CMS to measure data/MC scale factors as a function of trigger muon $p_T$, $\eta$, and other important kinematic parameters that these discrepancies may depend on. Then, when an HLT filter is applied to a dataset of MC events, each event is weighted by the appropriate scale factor, depending on the $p_T$, $\eta$, etc. of the trigger muon. Scale factors are calculated for basic ID selections on reconstructed objects at CMS and are thus used for weighting the events used in MC datasets in order to represent the effect of reconstruction inefficiencies.

\chapter{Physics signature and search strategy\label{sec:strategy}}

This chapter discusses the physics signature being sought and the theoretical motivations for this search, followed by a description of the datasets used in this search.

\section{Target signature\label{sec:signature}}

In this dissertation, I present a search for an NMSSM physics signal in a boosted four-tau final state (cf. Section~\ref{sec:SUSY} for a brief overview of the NMSSM Higgs sector). The signature process sought is the production of an SM-like Higgs boson $H$ followed by its decay to a pair of lighter pseudoscalar Higgs bosons $a$, each of which decays to a pair of taus. Due to the large mass difference between $H$ and $a$, the $a$'s are produced with a large boost.

Four production channels (Figure~\ref{fig:signatures}) for the $H$ are considered: $W$ and $Z$ associated production (WH and ZH), where a high-$p_T$ isolated muon from the vector boson decay provides a convenient trigger, gluon fusion (ggH), and vector boson fusion (VBF). The search was originally optimized for the WH mode but is sensitive to the ggH+VBF mode due to its large cross section. Since no forward jet tagging is done, the search is only sensitive to the sum of ggH and VBF, not each mode individually.

One of the $\tau\tau$ pairs is identified via the $\tau_{\mu}\tau_{\text{had}}$ decay topology, while no selection is made on the other $\tau\tau$ pair. The most significant backgrounds to the signal are expected to be SM $W$ and Drell-Yan production, where the $W$ and $Z$ decay to muons; $t\bar{t}$ with one or two muons in the final state; and production of heavy flavor ($c$ and $b$) mesons. In all of these backgrounds, a jet is misidentified as a boosted $\tau_{\mu}\tau_{\text{had}}$ pair.

\begin{figure}[hbtp]
  \begin{center}
    \includegraphics[width=\cmsFigWidth]{figures/FeynWH_ellnu4tau}
    \includegraphics[width=\cmsFigWidth]{figures/FeynggH_aa_4tau}
    \caption{Feynman diagrams of signal processes. (\cmsLeft) $W$ associated production channel. (\cmsRight) Gluon fusion production channel.}
    \label{fig:signatures}
  \end{center}
\end{figure}

\section{Motivations\label{sec:motivations}}

\subsection{Light pseudoscalars\label{sec:lighta}}
% Explain how some models allow light (pseudo)scalars, under what conditions this can occur, and what the expected cross-sections are --> ask Jack.

Following the discovery by the CMS and ATLAS experiments at the LHC~\cite{Aad:2012tfa,Chatrchyan:2012ufa} of a Higgs-like particle $H$, additional measurements of its properties using the full data sets at $\sqrt{s}$ = 7 and 8 TeV reveal that the observed state with a mass near 125.5 GeV is quite consistent with the standard model (SM)~\cite{ATLASnew,CMS:new,Aad:2013wqa}; this agreement is illustrated well by the results of the most recent measurements of the Higgs production cross-sections and branching ratios shown in Figure~\ref{fig:Higgsproperties}.

\begin{figure}[hbtp]
  \begin{center}
    \includegraphics[width=\cmsFigWidth]{figures/Higgsproperties-xsec}
    \includegraphics[width=\cmsFigWidth]{figures/Higgsproperties-br}
    \caption{Best-fit results for signal strengths for the Higgs production cross-sections (\cmsLeft) and branching ratios (\cmsRight) by the ATLAS and CMS experiments, normalized to Standard Model predictions~\cite{ATLAS-CONF-2015-044}.}
    \label{fig:Higgsproperties}
  \end{center}
\end{figure}

It is thus clear that models with an extended Higgs sector are significantly constrained by the data.  Consequently, it is interesting to explore the important possibility~\cite{Dermisek:2005ar,Dermisek:2006wr} that decays of the type $H$$\rightarrow$$aa$ (where $a$ is a lighter pseudoscalar) or $H$$\rightarrow$$hh$ (where $h$ is a lighter scalar) are present (for reviews, see~\cite{Chang:2008cw,Curtin:2013fra}). Such decays are certainly possible in the context of various extensions of the SM, including two Higgs doublet models (2HDM), the next-to-minimal supersymmetric standard model (NMSSM), and purely Higgs-sector models containing additional singlet Higgs fields, but notably are not possible in the (CP-conserving) minimal supersymmetric standard model (MSSM) because of the tightly constrained nature of its Higgs sector. 2HDM studies that consider, at least briefly, the possible decays of the observed SM-like Higgs to a pair of lighter Higgs bosons include~\cite{Curtin:2013fra,Celis:2013rcs,Grinstein:2013npa,Coleppa:2013dya,Chen:2013rba,Craig:2013hca,Wang:2013sha,Baglio:2014nea}. Studies in the NMSSM or NMSSM-like context include~\cite{Curtin:2013fra,King:2012tr,Cao:2013gba,Christensen:2013dra,Cerdeno:2013cz} and studies in the general case of adding a singlet field to the SM or the 2HDM can be found in~\cite{Curtin:2013fra,Chalons:2012qe,Ahriche:2013vqa}.

%The branching ratio for the $H$ to decay to two lighter Higgs bosons is limited by the apparently SM-like nature of the $H$. An often-studied option is that of $H$ decays to invisible states. However, branching ratio limits obtained under the assumption of invisibility do not apply to light Higgs pair states, which should rather be thought of as simply unseen, $U$, decay modes. The most thorough study for this case is that of~\cite{Belanger:2013xza} which combines CMS and ATLAS data. There it is found at 95\% C.L. that: Br($H$$\rightarrow$$U$) $\le$ 0.21 for a Higgs with completely SM-like couplings; Br($H$$\rightarrow$$U$) $\le$ 0.31 for a SM Higgs but allowing for extra loop contributions to its $\gamma\gamma$ and gg couplings; and Br($H$$\rightarrow$$U$) $\le$ 0.39 if the couplings to up quarks, down quarks and vector bosons are allowed to vary within a general model with only doublets and singlets in the Higgs sector (and no extra loop contributions to the gg and $\gamma\gamma$ couplings). If the up, down and vector boson couplings are allowed to vary completely freely, then all LHC rates can be reproduced if all the couplings-squared are increased by a factor of 1/(1-Br($H$$\rightarrow$$U$)). The only limit in this latter case arises from direct limits on the observed Higgs total width.  At the moment, this is at the level of $\Gamma_{\text{tot}} \le 4\Gamma_{\text{tot}}^{\text{SM}}$~\cite{CMSHwidth}. If the couplings-squared are all increased by 1/(1-Br($H$$\rightarrow$$U$)), the rates for $gg$$\rightarrow$$H$$\rightarrow$$U$ and other production mechanisms are all increased by a factor of Br($H$$\rightarrow$$U$)/(1-Br($H$$\rightarrow$$U$)), making such modes even more accessible. However, even if one adopts the more conservative approach of only considering doublets+singlets models, there is still an excellent prospect for seeing Higgs pair modes if Br($H$$\rightarrow$$U$) $\la$ 0.39.
The branching ratio for the $H$ to decay to two lighter Higgs bosons is limited by the apparently SM-like nature of the $H$. An often-studied option is that of $H$ decays to invisible states. However, branching ratio limits obtained under the assumption of invisibility do not apply to light Higgs pair states, which should rather be thought of as simply unseen, $U$, decay modes. The most thorough study for this case is that of~\cite{Bernon:2014vta} which combines CMS and ATLAS data. There it is found at 95.4\% C.L. that: Br($H$$\rightarrow$$U$) $\le$ 0.09 for a Higgs with completely SM-like couplings; Br($H$$\rightarrow$$U$) $\le$ 0.23 for a SM Higgs but allowing for extra loop contributions to its $\gamma\gamma$ and gg couplings; and Br($H$$\rightarrow$$U$) $\le$ 0.22 if the couplings to up quarks, down quarks and vector bosons are allowed to vary within a general model with only doublets and singlets in the Higgs sector (and no extra loop contributions to the gg and $\gamma\gamma$ couplings). If the up, down and vector boson couplings are allowed to vary completely freely, then all LHC rates can be reproduced if all the couplings-squared are increased by a factor of 1/(1-Br($H$$\rightarrow$$U$)). The only limit in this latter case arises from direct limits on the observed Higgs total width.  At the moment, this is at the level of $\Gamma_{\text{tot}} \le 4\Gamma_{\text{tot}}^{\text{SM}}$~\cite{CMSHwidth}. If the couplings-squared are all increased by 1/(1-Br($H$$\rightarrow$$U$)), the rates for $gg$$\rightarrow$$H$$\rightarrow$$U$ and other production mechanisms are all increased by a factor of Br($H$$\rightarrow$$U$)/(1-Br($H$$\rightarrow$$U$)), making such modes even more accessible. However, even if one adopts the more conservative approach of only considering doublets+singlets models, there is still an excellent prospect for seeing Higgs pair modes if Br($H$$\rightarrow$$U$) $\la$ 0.22.

Imposing Br($H$$\rightarrow$$U$) $\la$ 0.22 significantly constrains the 2HDM+singlets theory predictions for the values that Br($H$$\rightarrow$$aa$) or Br($H$$\rightarrow$$hh$) can have.  This is because the required $H$$\rightarrow$$aa$ or $H$$\rightarrow$$hh$ couplings are inevitably present and are generically sizeable, and are sufficiently large that to avoid $\Gamma$($H$$\rightarrow$$aa$,$hh$) $\gg$ $\Gamma$($H$$\rightarrow$$b\bar{b}$) requires significant parameter tuning (assuming $m_a$, $m_h$ $<$ $m_H$/2). For example, in the NMSSM (where the pseudoscalar mass eigenstate is defined by $a$ = $\cos\theta_A$ $a_{\text{MSSM}}$ + $\sin\theta_A$ $a_S$, with $a_{\text{MSSM}}$ being the MSSM-like pseudoscalar and $a_S$ the singlet pseudoscalar of the NMSSM). $\abs{\cos\theta_A}$ $\ll$ 1 is generically needed to keep the Haa coupling small by suppressing the doublet content of the $a$.  In the 2HDM, fine-tuned relations among the parameters of the model are required for acceptably small ($\la 0.2$) Br($H$$\rightarrow$$aa$) or Br($H$$\rightarrow$$hh$).

Direct constraints on the $a$ or $h$ play a role in assessing the possibilities. A previous CMS result~\cite{Chatrchyan:2012am} (based on~\cite{Dermisek:2009fd}) places limits on $\sigma$($pp$$\rightarrow$$aa$$\rightarrow$$\mu\mu$) on the order of 2\mdash6 pb in the mass range from 5\mdash14 GeV, excluding the upsilon resonance region. These limits, despite being based on just 1.3 fb$^{-1}$ of 7 TeV data, can impact models. For example,~\cite{Chatrchyan:2012am} shows that they significantly constrain the $\cos\theta_A$ mixing angle factor defining the NMSSM.  The constraints on $\abs{\cos\theta_A}$ are especially strong at large $\tan\beta$, where $\tan\beta$ is defined as the ratio of the neutral Higgs vacuum expectation values in the MSSM. In the case of the single $a$ = $a_{\text{MSSM}}$ state of the CP-conserving 2HDM, points in the parameter space that are consistent with $m_H \sim$ 125.5 GeV fits at 95\% C.L. and other LHC and pre-LHC constraints violate this limit in the case of Type~II models (but not in the case of Type~I)~\cite{Dumont:2014wha}.  However, in general such constraints do not significantly restrict Br($H$$\rightarrow$$aa$) or Br($H$$\rightarrow$$hh$).  

The techniques appropriate for detecting a Higgs-pair decay mode depend crucially on the mass of the lighter Higgs boson. One important possibility, particularly prominent in the NMSSM, is that the lightest CP-odd state $a$ has mass below or not far above $2m_b$. Small $m_a$ arises naturally in the limit of a so-called $U(1)_R$ symmetry of the model. However, a small mass for the light Higgs states is generically possible in all the models listed earlier. In addition, even if the light Higgs boson has mass above $2m_b$ (but, of course, below $m_H$/2) the $\tau\tau$ mode will still have a branching ratio of order 0.045 and will have smaller backgrounds than a purely 4$b$ final state or the $bb\tau\tau$ final state. Thus, a generic exploration of the sensitivity in the $4\tau$ final state is of considerable interest, especially as more and more integrated luminosity is accumulated in future running.

\subsection{Semileptonic di-tau decays\label{sec:semileptonic}}

This search explores the current level of sensitivity to the $4\tau$ final state, and techniques are developed for isolating this final state from backgrounds. In particular, at least one of the tau pairs produced in the decays of the light Higgs bosons is required to decay semileptonically as $\tau_{\mu}\tau_{\text{had}}$. Requiring at least one hadronic tau decay is intended to maximize statistics, due to the higher branching ratio for hadronic tau decays -- 64.76\% compared to 17.41\% and 17.83\% for decays to muons or electrons respectively~\cite{Agashe:2014kda}. However, the choice for the other tau not to decay hadronically was motivated by issues with the reconstruction of boosted tau pairs, as well as the relative difficulty in discriminating fully hadronic tau pair decays from background processes.

Because of the large mass difference between the $H$ and $a$, the final-state tau pairs are highly collimated, resulting in the overlap of their decay products. This spoils the isolation of the individual taus and renders their reconstruction difficult or impossible by standard means. In order to reconstruct boosted $\tau\tau$ pairs, a modified version of the standard hadron plus strips (HPS)~\cite{CMS:2011msa} tau reconstruction procedure has been developed for this search. This method, described fully in Section~\ref{sec:evtsel-tauID}, involves identifying and removing a leptonic tau decay candidate (muon or electron) from the PF jet used to seed the hadronic tau decay reconstruction. As this method has been successful in recovering hadronic tau ID efficiency, this search has thus focused on the reconstruction of semileptonic boosted tau decays -- in particular, only $\tau_{\mu}\tau_{\text{had}}$ decays, due to the relative ease and cleanness with which low-$p_T$ muons are reconstructed at CMS compared to electrons.

\section{Datasets\label{sec:datasets}}
%%%%%%%%%%%%%%%%%%%

\subsection{Data samples and trigger\label{sec:datasets-data}}
The datasets used in this search are the most recent \texttt{SingleMu} primary datasets collected by CMS in 2012 at $\sqrt{s}$ = 8 TeV.  The high level trigger (HLT) path \texttt{HLT\_IsoMu24\_eta2p1} requires the presence of at least one isolated muon with $p_T >$ 24 GeV found within the CMS muon coverage of $\abs{\eta} <$ 2.1.  More details about the HLT muon reconstruction and isolation requirement can be found in Ref.~\cite{HLTMenus}.  These datasets, listed in Table~\ref{tab:Data}, correspond to an integrated luminosity of 19.7 fb$^{-1}$.%The JSON file used in the search for bad data masking can be found at \texttt{/afs/cern.ch/cms/CAF/CMSCOMM/COMM\_DQM/certification/Collisions12/8TeV/\\Reprocessing/Cert\_190456-208686\_8TeV\_22Jan2013ReReco\_Collisions12\_JSON.\\txt}.

\begin{table}
\begin{center}
\singlespacing
\begin{tabular}{ll}
	\hline
	Dataset name & Run range \\
	\hline
	\texttt{/SingleMu/Run2012A-22Jan2013-v1/AOD} & 190456-193621 \\
	\texttt{/SingleMu/Run2012B-22Jan2013-v1/AOD} & 193833-196531 \\
	\texttt{/SingleMu/Run2012C-22Jan2013-v1/AOD} & 198022-203742 \\
	\texttt{/SingleMu/Run2012D-22Jan2013-v1/AOD} & 203777-208686 \\
	\hline
\end{tabular}
\caption{Data samples.}
\label{tab:Data}
\end{center}
\end{table}

\subsection{Monte Carlo samples\label{sec:datasets-MC}}
The Monte Carlo (MC) samples used for the backgrounds outlined in Section~\ref{sec:signature} are listed in Table~\ref{tab:MCBkg}.

\begin{sidewaystable}
\begin{center}
\caption{Monte Carlo background samples. Cross sections from \cite{PREP} and \cite{8TeVTwiki}.\label{tab:MCBkg}}
\singlespacing
\resizebox{\textwidth}{!}{\begin{tabular}{| l | l |}
        \hline
	Dataset name & \begin{tabular}[c]{@{}l@{}}Cross \\section (pb)\end{tabular} \\
	\hline
	\hline
	\texttt{/W1JetsToLNu$\_$TuneZ2Star$\_$8TeV-madgraph/Summer12$\_$DR53X-PU$\_$S10$\_$START53$\_$V7A-v1/AODSIM} & 6601.5 \\
	\texttt{/W2JetsToLNu$\_$TuneZ2Star$\_$8TeV-madgraph/Summer12$\_$DR53X-PU$\_$S10$\_$START53$\_$V7A-v1/AODSIM} & 2110.3 \\
	\texttt{/W3JetsToLNu$\_$TuneZ2Star$\_$8TeV-madgraph/Summer12$\_$DR53X-PU$\_$S10$\_$START53$\_$V7A-v1/AODSIM} & 633.6 \\
	\texttt{/W4JetsToLNu$\_$TuneZ2Star$\_$8TeV-madgraph/Summer12$\_$DR53X-PU$\_$S10$\_$START53$\_$V7A-v1/AODSIM} & 214 \\
	\texttt{/DYJetsToLL$\_$M-10To50$\_$TuneZ2Star$\_$8TeV-madgraph/Summer12$\_$DR53X-PU$\_$S10$\_$START53$\_$V7A-v1/AODSIM} & 14702 \\
	\texttt{/DYJetsToLL$\_$M-50$\_$TuneZ2Star$\_$8TeV-madgraph-tarball/Summer12$\_$DR53X-PU$\_$S10$\_$START53$\_$V7A-v1/AODSIM} & 3503.71 \\
	\texttt{/TTJets$\_$MassiveBinDECAY$\_$TuneZ2star$\_$8TeV-madgraph-tauola/Summer12$\_$DR53X-PU$\_$S10$\_$START53$\_$V7A-v2/AODSIM} & 245.8 \\
	\texttt{/T$\_$s-channel$\_$TuneZ2star$\_$8TeV-powheg-tauola/Summer12$\_$DR53X-PU$\_$S10$\_$START53$\_$V7A-v1/AODSIM} & 3.79 \\
	\texttt{/T$\_$t-channel$\_$TuneZ2star$\_$8TeV-powheg-tauola/Summer12$\_$DR53X-PU$\_$S10$\_$START53$\_$V7A-v1/AODSIM} & 56.4 \\
	\texttt{/Tbar$\_$s-channel$\_$TuneZ2star$\_$8TeV-powheg-tauola/Summer12$\_$DR53X-PU$\_$S10$\_$START53$\_$V7A-v1/AODSIM} & 1.76 \\
	\texttt{/Tbar$\_$t-channel$\_$TuneZ2star$\_$8TeV-powheg-tauola/Summer12$\_$DR53X-PU$\_$S10$\_$START53$\_$V7A-v1/AODSIM} & 30.7 \\
	\texttt{/WW$\_$TuneZ2star$\_$8TeV$\_$pythia6$\_$tauola/Summer12$\_$DR53X-PU$\_$S10$\_$START53$\_$V7A-v1/AODSIM} & 54.838 \\
	\texttt{/WZ$\_$TuneZ2star$\_$8TeV$\_$pythia6$\_$tauola/Summer12$\_$DR53X-PU$\_$S10$\_$START53$\_$V7A-v1/AODSIM} & 33.21 \\
	\texttt{/ZZ$\_$TuneZ2star$\_$8TeV$\_$pythia6$\_$tauola/Summer12$\_$DR53X-PU$\_$S10$\_$START53$\_$V7A-v1/AODSIM} & 17.654 \\
	\hline
\end{tabular}}
\end{center}
\end{sidewaystable}

The Monte Carlo signal samples for associated WH production and gluon fusion production were generated with PYTHIA~\cite{1126-6708-2006-05-026} and reconstructed with CMSSW version 5.3 using the \texttt{S10} pileup scenario. The $W$ in the associated $W$ production sample is constrained to decay only leptonically. Since PYTHIA does not model NMSSM processes, the production and decay of the NMSSM scalar and pseudoscalar Higgs particles were approximated using PYTHIA's two Higgs doublet model instead. A separate sample of signal events was generated using MADGRAPH~\cite{springerlink:10.1007/JHEP06(2011)128}, which does contain methods for modeling NMSSM processes directly, and the kinematics of the PYTHIA and MADGRAPH event samples were shown to be compatible and equivalent. The benchmark for this search takes the masses of the NMSSM $a$, $h_1$, $h_2$, and $h_3$ to be 9, 125, 500, and 500 GeV respectively; a mass scan is performed over $m_{a}$ from 5 to 15 GeV in increments of 2 GeV.  The assumed cross sections for each signal sample are given in Table~\ref{tab:MC-sig}.  These are the cross sections for SM 125 GeV Higgs production at 8 TeV~\cite{LHCHXSWG} multiplied by BR($H\rightarrow$$aa\rightarrow4\tau$) = 100\%, which is why the cross sections are constant with pseudoscalar mass.  The $W\rightarrow$ leptons branching ratio is included in the quoted cross sections for the WH signals.

\begin{table*}[htbh]
\begin{center}
\caption{Assumed signal MC cross sections.\label{tab:MC-sig}}
\singlespacing
\begin{tabular}{|c|c|c|}
\hline
\multicolumn{2}{|c|}{} & Cross section (pb) \\
\hline
\multirow{5}{*}{WH} & $m_{a}$ = 5 GeV & 0.2296\\
& $m_{a}$ = 7 GeV & 0.2296\\
& $m_{a}$ = 9 GeV & 0.2296\\
& $m_{a}$ = 11 GeV & 0.2296\\
& $m_{a}$ = 13 GeV & 0.2296\\
& $m_{a}$ = 15 GeV & 0.2296\\
\hline
\multirow{5}{*}{ggH} & $m_{a}$ = 5 GeV & 19.27\\
& $m_{a}$ = 7 GeV & 19.27\\
& $m_{a}$ = 9 GeV & 19.27\\
& $m_{a}$ = 11 GeV & 19.27\\
& $m_{a}$ = 13 GeV & 19.27\\
& $m_{a}$ = 15 GeV & 19.27\\
\hline
\multirow{5}{*}{ZH} & $m_{a}$ = 5 GeV & 0.4153\\
& $m_{a}$ = 7 GeV & 0.4153\\
& $m_{a}$ = 9 GeV & 0.4153\\
& $m_{a}$ = 11 GeV & 0.4153\\
& $m_{a}$ = 13 GeV & 0.4153\\
& $m_{a}$ = 15 GeV & 0.4153\\
\hline
\multirow{5}{*}{VBF} & $m_{a}$ = 5 GeV & 1.578\\
& $m_{a}$ = 7 GeV & 1.578\\
& $m_{a}$ = 9 GeV & 1.578\\
& $m_{a}$ = 11 GeV & 1.578\\
& $m_{a}$ = 13 GeV & 1.578\\
& $m_{a}$ = 15 GeV & 1.578\\
\hline
\end{tabular}
\end{center}
\end{table*}

\subsection{Higgs transverse momentum reweighting for ggH\label{sec:datasets-higgsptreweight}}

In gluon fusion Higgs production, the Higgs $p_T$ spectrum can be significantly affected by next-to-leading logarithmic (NLL) and next-to-next-to-leading logarithmic (NNLL) corrections, especially in the low-$p_T$ range~\cite{Bozzi:2003jy}. Thus, a set of weights binned in $p_T$, calculated with the Higgs $p_T$-reweighting HqT software \cite{HQTDocumentation}, was applied to the Higgs $p_T$ spectrum of signal MC events in the ggH production channel. The effect of this reweighting was observed to be quite small, as it produced a change of less than 2\% in signal-to-background ratio for each signal and a change of less than 2.5\% in the number of each type of signal event passing the final selection.

\subsection{ZH and VBF production channels\label{sec:datasets-zhvbf}}

A study was done to assess the contribution of ZH and VBF production channels to the expected signal significance. ZH and VBF signal samples were generated for pseudoscalar mass point $m_{a}$ = 9 GeV and the numbers of events passing the full selection sequence were normalized to 19.7 fb$^{-1}$ using the official SM production cross sections for ZH and VBF. Then, expected exclusion limits for the WH and ggH signal channels were calculated after the total expected yield of ZH and VBF events was distributed among the WH and ggH expected yields proportionally to their sizes, and these expected exclusion limits were compared to the nominal expected exclusion limits for WH and ggH without the added events. The combined presence of ZH and VBF signals changed the WH and ggH expected exclusion limits by at most 10\%, and the change was always well within the $1\sigma$ error band of the nominal expected limits. Yet, in one of the search regions, the contribution of VBF was larger than that of WH, so ultimately it was concluded that the contributions of VBF and ZH should be considered too.

However, due to a shortage of time, ZH and VBF MC samples could not be generated for the other pseudoscalar mass points. Instead, the contribution of VBF for each pseudoscalar mass point was estimated by taking the number of surviving ggH events in the counting experiment bin after the full selection and normalizing this number to the expected SM VBF production cross-section, since the selection efficiency is expected to be the same for ggH and VBF topologies; effects due to the different $H$ $p_T$ spectra for the two topologies were found not to be significant. Also, since ZH and WH are expected to have similar selection efficiencies (with the ZH trigger efficiency being 1.1 times higher than for WH due to the decay of $Z$ to two high-$p_T$ muons rather than one), the contribution of ZH at each pseudoscalar mass point was estimated similarly by rescaling the WH contribution to the expected SM ZH production cross-section. %motivations
\chapter{Event selection\label{sec:evtsel}}

For the events passing the high-level trigger \texttt{HLT\_IsoMu24\_eta2p1}, a series of selection cuts has been developed to identify the most important physics objects in the signal -- the high-$p_T$ trigger muon, the tau decay muon $\tau_{\mu}$ from one leg of the $a$($h$) decay, and the tau $\tau_{\text{had}}$ from the other leg of the $a$($h$) decay -- and optimize the signal sensitivity. This set of cuts will be referred to as the preselection, and plays a role in the estimation of the background. The physics objects to which the selections are applied are reconstructed via the CMS particle flow (PF) algorithm.
%PF described fully in \cite{CMS:2010eua}

A brief list of the preselection cuts is as follows:

\begin{itemize}
	\item Trigger $\mu$ $p_T$ selection
	\item Trigger $\mu$ ID
	\item Trigger $\mu$ PF relative isolation selection
	\item $\tau_{\mu}$ $p_T$ selection
	\item $\tau_{\mu}$ ID
	\item $\tau_{\text{had}}$ $p_T$ selection
	\item $\tau_{\text{had}}$ HPS decay mode finding discriminator
	\item $\tau_{\text{had}}$ HPS isolation discriminator
	\item Charge requirement: $q(\text{Trigger} \mu) \cdot q(\tau_{\mu}) >$ 0
	\item Charge requirement: $q(\tau_{\text{had}}) \cdot q(\tau_{\mu}) <$ 0
        \item b-jet veto
        \item Neighbouring lepton veto around trigger muon
	\item Requirement of consistency with the primary vertex
\end{itemize}

Finally, events are classified into one of two bins: low transverse mass $M_{\text{T}} \le$ 50 GeV or high transverse mass $M_{\text{T}} >$ 50 GeV, where $M_{\text{T}} = \sqrt{2p_{T}^{\text{Trig}\mu}\ETslash(1 - \cos{\Delta\phi(\text{Trig}\mu, \ETslash)})}$.  The low-$M_{\text{T}}$ bin is sensitive to gluon fusion and VBF signal production, where there is no real $W$, while the high-$M_{\text{T}}$ bin is optimized for WH production.

\section{Trigger muon ID\label{sec:evtsel-triggermu}}

Events are required to have at least one reco muon that satisfies the following criteria:
\begin{itemize}
	\item $p_T >$ 25 GeV (this is at the turn-on point for the HLT used in this analysis, as shown in \cite{CMS:muonhlttwiki})
	\item $\abs{\eta} <$ 2.1
	\item Tight muon ID:
	\begin{itemize}
		\item The reco muon is reconstructed as a global muon and as a PF muon
		\item The global muon track fit has $\chi^{2}/\text{ndof} <$ 10 and at least one muon chamber hit 
		\item The reco muon has segments in at least 2 muon stations
		\item The reco muon's tracker track has $d_{\text{xy}} <$ 2 mm and $d_{\text{z}} <$ 5 mm
		\item Number of pixel hits $>$ 0
		\item More than 5 tracker layers with hits
	\end{itemize}
	\item Relative isolation $I_{\text{rel}} <$ 0.12, where the $I_{\text{rel}}$ of a muon is defined as the pileup-corrected sum of the transverse energy of the photons and charged and neutral hadrons in a cone of radius $\Delta$R = $\sqrt{\Delta\eta^{2} + \Delta\phi^{2}} =$ 0.4 about the muon divided by the $p_T$ of the muon. This is the tight isolation working point recommended by the CMS Muon POG \cite{CMS:muonidtwiki}.
        \item Isolation from nearby leptons located within a cone of $\Delta$R $=$ 0.4 around the trigger muon, where nearby lepton ID criteria are as follows:
          \begin{itemize}
          \item \textbf{Electrons:} \texttt{reco::GsfElectron} passing PF reconstruction with $p_T >$ 7 GeV and $\abs{\eta} <$ 2.5 (same as~\cite{Chatrchyan:2013mxa})
          \item \textbf{Muons:} PF muon with $p_T >$ 5 GeV and $\abs{\eta} <$ 2.4 passing the soft muon ID described in Section~\ref{sec:evtsel-softmu} and~\cite{CMS:2010uta}
          \item \textbf{Taus:} HPS PF tau with $p_T >$ 10 GeV, $\abs{\eta} <$ 2.3, and passing \texttt{DecayModeFinding} and \texttt{MediumCombinedIsolationDBSumPtCorr} discriminators reconstructed from an AK5 PF jet that has been cleaned of the trigger muon with the same jet-cleaning algorithm described in Section~\ref{sec:evtsel-tauID} (the $p_T$ cut at 10 GeV rather than 20 GeV was chosen to make the veto more stringent)
          \end{itemize}
\end{itemize}

The reco muon passing the above criteria (or, if more than one reco muon passed, the one with the highest $p_T$) is then matched to the object that fired the \texttt{HLT\_IsoMu24\_eta2p1} trigger. This is done by requiring $\Delta$R $<$ 0.1 between the reco muon and the trigger object.

%leptonveto subsection
\subsection{Neighbouring lepton veto for trigger muon\label{sec:evtsel-leptonveto}}

The nearby lepton isolation requirement is motivated by the desire to have a well-understood trigger and PF relative isolation efficiency for the ggH and VBF production modes to which this analysis is sensitive.  Unlike in the WH and ZH production channels, in which the particle firing the isolated muon trigger is an isolated muon from $W$ or $Z$ decay, the particle firing the isolated muon trigger in the ggH and VBF production channels is a muon coming from a tau decay.  The difference is illustrated in Figure~\ref{fig:WHZH-vs-ggHVBF-trigger}.

\begin{figure}[hbtp]
  \begin{center}
    \includegraphics[width=\cmsFigWidth]{figures/WH_trigger}
    \includegraphics[width=\cmsFigWidth]{figures/ZH_trigger}
    \includegraphics[width=\cmsFigWidth]{figures/ggH_trigger}
    \includegraphics[width=\cmsFigWidth]{figures/VBF_trigger}
    \caption{Diagrams of the four Higgs production modes considered in this analysis, with the triggering particle circled in red.  (Top \cmsLeft) WH.  (Top \cmsRight) ZH.  (Bottom \cmsLeft) ggH.  (Bottom \cmsRight) VBF.}
    \label{fig:WHZH-vs-ggHVBF-trigger}
  \end{center}
\end{figure}

Due to the boost and low $p_T$ of the pseudoscalar tau decay pairs in ggH and VBF events, most are rejected by \texttt{HLT\_IsoMu24\_eta2p1}.  Those that are accepted fall into two categories:

\begin{enumerate}
\item $a\rightarrow\tau\tau$, one tau decays to a 24 GeV muon and the other tau decays to particles with $p_T$ low enough to pass the HLT muon isolation cut
\item $a\rightarrow\tau\tau$, one tau decays to a 24 GeV muon and the other tau decays far enough away to not be counted in the HLT muon isolation sum
\end{enumerate}

To avoid likely systematic effects in the MC description of ggH and VBF trigger and PF relative isolation efficiency due to the presence of low $p_T$ particles around the trigger muon, events in category 1 are rejected by the nearby lepton isolation requirement.  With this requirement, tau decay muons from the accepted category 2 events appear very similar to muons from $Z$'s or $W$'s, for which \texttt{HLT\_IsoMu24\_eta2p1} and PF relative isolation $<$ 0.12 efficiency measurements and standard scale factors for data-simulation differences are well understood.  The following sections demonstrate that once the nearby lepton isolation requirement is imposed, the efficiency of \texttt{HLT\_IsoMu24\_eta2p1} and PF relative isolation $<$ 0.12 for ggH and VBF tau decay muons in MC is very similar to that of $W$ decay muons in MC.

%HLT efficiency subsection

\subsection{Study of the HLT efficiency for signal events produced via gluon fusion\label{sec:lep-id-eff-ggH-HLT}}
The efficiency for ggH $a\rightarrow\tau\rightarrow\mu$ decay muons to fire \texttt{HLT\_IsoMu24\_eta2p1} is calculated for two reconstructed muon selections.  The first selection is  criteria described in Sec.~\ref{sec:evtsel-triggermu}, except that the nearby lepton isolation requirement is removed.  The second selection is identical to the criteria described in Sec.~\ref{sec:evtsel-triggermu}.  Trigger efficiency is compared for the two selections.  Since the signature of pseudoscalar decays in the detector is similar between the ggH and VBF production modes, the results obtained for ggH simulation can be applied to VBF simulation as well.

The trigger efficiencies of the two selections are given by 

\begin{equation}
\epsilon_{\text{HLT}}^{\text{no }l\text{ iso}} = \frac{\text{No. gen-matched reco'd muons passing no-lepton-isolation ID and HLT}}{\text{No. gen-matched reco'd muons passing trigger muon ID}} \\
\label{eq:muonHLTeff-noliso}
\end{equation}

\begin{equation}
\epsilon_{\text{HLT}} = \frac{\text{No. gen-matched reco'd muons passing trigger muon ID and HLT}}{\text{No. gen-matched reco'd muons passing trigger muon ID}} \\
\label{eq:muonHLTeff}
\end{equation}

where

\begin{itemize}
\item the gen-matching criteria is $\Delta$$p_T$(reco muon, gen $a\rightarrow\tau\rightarrow\mu$ muon) $<$ 0.1 GeV and the gen muon is from the decay of a tau that is itself from the decay of a pseudoscalar;
\item the trigger muon ID for $\epsilon_{\text{HLT}}^{\text{no }l\text{ iso}}$ is as described in Sec.~\ref{sec:evtsel-triggermu} but with the nearby lepton isolation requirement removed;
\item the trigger muon ID for $\epsilon_{\text{HLT}}$ is as described in Sec.~\ref{sec:evtsel-triggermu}; and
\item ``HLT'' refers to firing \texttt{HLT\_IsoMu24\_eta2p1}.
\end{itemize}

Figure~\ref{fig:HLTEffVsDR} shows $\epsilon_{\text{HLT}}^{\text{no }l\text{ iso}}$ as a function of $\Delta$R(gen $a\rightarrow\tau\rightarrow\mu$ muon, gen $\tau_{\text{2}}$), where the gen $a\rightarrow\tau\rightarrow\mu$ muon is matched to the reco'd muon as described in Eq.~\ref{eq:muonHLTeff-noliso} above and the gen $\tau_{\text{2}}$ is the other tau from the $a\rightarrow\tau\tau$ decay.  The efficiency is calculated separately for each decay mode of the gen $\tau_{\text{2}}$ (electronic, muonic, or hadronic).  $\epsilon_{\text{HLT}}^{\text{no }l\text{ iso}}$ is $\sim$90\% for $\Delta$R $>$ 0.4, when the two taus from pseudoscalar decay are separated enough that the tau decay muon appears isolated. This is similar to the efficiency of \texttt{HLT\_IsoMu24\_eta2p1} for $W$ decay muons~\cite{1748-0221-7-10-P10002}.  When the two taus are closer than $\Delta$R $\sim$ 0.4, the efficiency decreases because the non-triggering tau spoils the isolation of the tau decay muon that fires the trigger.  The effect is worst in the $\tau_{\mu}\tau_{e}$ and $\tau_{\mu}\tau_{\mu}$ modes because electrons and muons contribute to isolation at the trigger level, but are not counted in the offline PF relative isolation.

\begin{figure}[hbtp]
  \begin{center}
    \includegraphics[width=0.8\cmsFigWidth]{figures/dREfficiency_WmuIDIso_muEOnly}
    \includegraphics[width=0.8\cmsFigWidth]{figures/dREfficiency_WmuIDIso_muMuOnly}
    \includegraphics[width=0.8\cmsFigWidth]{figures/dREfficiency_WmuIDIso_muHadOnly}
    \caption{$\epsilon_{\text{HLT}}^{\text{no }l\text{ iso}}$ for the ggH signal as a function of the separation $\Delta$R(gen $a\rightarrow\tau\rightarrow\mu$ muon, gen $\tau_{\text{2}}$), where the gen $\tau_{\text{2}}$ is a decay product of the same pseudoscalar as in the $a\rightarrow\tau\rightarrow\mu$.  The $a\rightarrow\tau\rightarrow\mu$ muon is matched to the reco'd muon as described in the text.  The reco'd muon is required to pass the trigger muon ID of Sec.~\ref{sec:evtsel-triggermu}, but with the nearby lepton isolation requirement removed.  (\cmsLeft) Gen $\tau_{\text{2}}$ decays to an electron.  (middle) Gen $\tau_{\text{2}}$ decays to a muon.  (\cmsRight) Gen $\tau_{\text{2}}$ decays to hadrons.}
    \label{fig:HLTEffVsDR}
  \end{center}
\end{figure}

In contrast, Figure~\ref{fig:HLTEffVsDR_withFilters} shows $\epsilon_{\text{HLT}}$ as a function of $\Delta$R(gen $a\rightarrow\tau\rightarrow\mu$ muon, gen $\tau_{\text{2}}$), where the gen $a\rightarrow\tau\rightarrow\mu$ muon is matched to the reco'd muon as described in Eq.~\ref{eq:muonHLTeff} above and the gen $\tau_{\text{2}}$ is the other tau from the $a\rightarrow\tau\rightarrow\mu$ decay.  The efficiency is calculated separately for each decay mode of the gen $\tau_{\text{2}}$ (electronic, muonic, or hadronic).  The efficiencies are much flatter in $\Delta$R when the nearby lepton isolation requirement is applied to the reconstructed trigger muon, because it insures that only events in which the two reconstructed taus from pseudoscalar decay are well separated may pass the analysis selection.  The trigger efficiency for $a\rightarrow\tau\rightarrow\mu$ muons in these events is similar to that of $W$ decay muons and is in the regime where the trigger muon is isolated and MC describes the data well.

\begin{figure}[hbtp]
  \begin{center}
    \includegraphics[width=0.8\cmsFigWidth]{figures/dRHLTEfficiency_WmuIDIso_withFilters_muEOnly}
    \includegraphics[width=0.8\cmsFigWidth]{figures/dRHLTEfficiency_WmuIDIso_withFilters_muMuOnly}
    \includegraphics[width=0.8\cmsFigWidth]{figures/dRHLTEfficiency_WmuIDIso_withFilters_muHadOnly}
    \caption{$\epsilon_{\text{HLT}}$ for the ggH signal as a function of the separation $\Delta$R(gen $a\rightarrow\tau\rightarrow\mu$ muon, gen $\tau_{\text{2}}$), where the gen $\tau_{\text{2}}$ is a decay product of the same pseudoscalar as in the $a\rightarrow\tau\rightarrow\mu$.  The $a\rightarrow\tau\rightarrow\mu$ muon is matched to the reco'd muon as described in the text.  The reco'd muon is required to pass the trigger muon ID of Sec.~\ref{sec:evtsel-triggermu}.  (\cmsLeft) Gen $\tau_{\text{2}}$ decays to an electron.  (middle) Gen $\tau_{\text{2}}$ decays to a muon.  (\cmsRight) Gen $\tau_{\text{2}}$ decays to hadrons.}
    \label{fig:HLTEffVsDR_withFilters}
  \end{center}
\end{figure}

Figure~\ref{fig:HLT-eff-W-mu} shows the HLT efficiency for muons passing the trigger muon ID in both the WH and gluon fusion production modes.  In both modes, the trigger muon ID includes the nearby lepton non-overlap requirement.  The efficiencies are very similar for the reasons discussed above.

\begin{figure}[hbtp]
  \begin{center}
    \includegraphics[width=1.24\cmsFigWidth]{figures/HLT_eff_Wmu_WH}
    \includegraphics[width=1.24\cmsFigWidth]{figures/HLT_eff_Wmu_ggH}
    \caption{MC simulation prediction of efficiency for reconstructed muons passing the trigger muon ID to fire \texttt{HLT\_IsoMu24\_eta2p1}. Efficiencies were measured in MC events where the Higgs is produced via the (\cmsLeft) WH and (\cmsRight) gluon fusion channels.}
    \label{fig:HLT-eff-W-mu}
  \end{center}
\end{figure}

%Iso efficiency subsection
\subsection{Study of the particle flow relative isolation efficiency for signal events produced via gluon fusion\label{sec:muon-id-eff-iso}}

The efficiency for ggH $a\rightarrow\tau\rightarrow\mu$ decay muons that pass the tight muon ID criteria to pass the PF relative isolation cut is calculated for two reconstructed muon selections.  As a reminder, the PF relative isolation is defined as the $p_T$ sum of all PF hadrons and photons within a cone of $\Delta$R = 0.4 around the trigger muon divided by the trigger muon $p_T$.  Both selections have the PF relative isolation requirement of Sec.~\ref{sec:evtsel-triggermu} removed, since it is the efficiency of that requirement being studied here.   Barring that, the first selection is identical to the criteria described in Sec.~\ref{sec:evtsel-triggermu}, except that in addition the nearby lepton isolation requirement is removed.  Similarly, barring the PF relative isolation requirement, the second selection is identical to the criteria described in Sec.~\ref{sec:evtsel-triggermu}.  PF relative isolation efficiency is compared for the two selections.  Since the signature of pseudoscalar decays in the detector is similar between the ggH and VBF production modes, the results obtained for ggH simulation can be applied to VBF simulation as well.

The PF relative isolation efficiencies of the two selections are given by 

\begin{equation}
\epsilon_{\text{rel. iso}}^{\text{no }l\text{ iso}} = \frac{\text{No. gen-matched reco'd muons passing no-lepton-isolation ID and PF rel. iso.}}{\text{No. gen-matched reco'd muons passing trigger muon ID excl. PF rel. iso.}} \\
\label{eq:muonPFRelIsoeff-noliso}
\end{equation}

\begin{equation}
\epsilon_{\text{rel. iso.}} = \frac{\text{No. gen-matched reco'd muons passing trigger muon ID incl. PF. rel. iso.}}{\text{No. gen-matched reco'd muons passing trigger muon ID excl. PF rel. iso.}} \\
\label{eq:muonPFRelIsoeff}
\end{equation}

where

\begin{itemize}
\item the gen-matching criteria is $\Delta$$p_T$(reco muon, gen $a\rightarrow\tau\rightarrow\mu$ muon) $<$ 0.1 GeV and the gen muon is from the decay of a tau that is itself from the decay of a pseudoscalar;
\item the trigger muon ID for $\epsilon_{\text{rel. iso.}}^{\text{no }l\text{ iso}}$ is as described in Sec.~\ref{sec:evtsel-triggermu} but with the PF relative isolation and nearby lepton isolation requirements removed;
\item the trigger muon ID for $\epsilon_{\text{rel. iso.}}$ is as described in Sec.~\ref{sec:evtsel-triggermu} but with the PF relative isolation requirement removed; and
\item ``rel. iso.'' refers to passing the cut PF relative isolation $<$ 0.12.
\end{itemize}

Figure~\ref{fig:IsoEffVsDR} shows $\epsilon_{\text{rel. iso.}}^{\text{no }l\text{ iso}}$ as a function of $\Delta$R(gen $a\rightarrow\tau\rightarrow\mu$ muon, gen $\tau_{\text{2}}$), where the gen $a\rightarrow\tau\rightarrow\mu$ muon is matched to the reco'd muon as described in Eq.~\ref{eq:muonPFRelIsoeff-noliso} above and the gen $\tau_{\text{2}}$ is the other tau from the $a\rightarrow\tau\tau$ decay.  The efficiency is calculated separately for each decay mode of the gen $\tau_{\text{2}}$ (electronic, muonic, or hadronic).  $\epsilon_{\text{rel. iso.}}^{\text{no }l\text{ iso}}$ is $\sim$80\%, independent of $\Delta$R, for the $\tau_{\mu}\tau_{\text{e}}$ the $\tau_{\mu}\tau_{\mu}$ channels.  Because PF electrons and muons are not counted in the PF relative isolation sum, the presence of a nearby $\tau_{\text{e}}$ or $\tau_{\mu}$ does not significantly spoil the relative isolation of the main $a\rightarrow\tau\rightarrow\mu$ muon.  The overall efficiency is lower than the efficiency for $Z$ decay muons~\cite{1748-0221-7-10-P10002} by $\sim$15\% due to the different kinematics of di-tau objects vs. isolated $Z$ decay muons.  Conversely, in the $\tau_{\mu}\tau_{\text{had}}$ channel, $\epsilon_{\text{rel. iso.}}^{\text{no }l\text{ iso}}$ is $\sim$80\% only for $\Delta$R $>$ 0.4, when the two taus from pseudoscalar decay are separated enough that the tau decay muon appears isolated.  When the two taus are closer than $\Delta$R $\sim$ 0.4, the efficiency decreases because the hadronically decaying non-identified partner tau spoils the relative isolation of the tau decay muon that is identified as a trigger muon.

\begin{figure}[hbtp]
  \begin{center}
    \includegraphics[width=0.8\cmsFigWidth]{figures/dRIsoEfficiency_muEOnly}
    \includegraphics[width=0.8\cmsFigWidth]{figures/dRIsoEfficiency_muMuOnly}
    \includegraphics[width=0.8\cmsFigWidth]{figures/dRIsoEfficiency_muHadOnly}
    \caption{$\epsilon_{\text{rel. iso.}}^{\text{no }l\text{ iso}}$ for the ggH signal as a function of the separation $\Delta$R(gen $a\rightarrow\tau\rightarrow\mu$ muon, gen $\tau_{\text{2}}$), where the gen $\tau_{\text{2}}$ is a decay product of the same pseudoscalar as in the $a\rightarrow\tau\rightarrow\mu$.  The $a\rightarrow\tau\rightarrow\mu$ muon is matched to the reco'd muon as described in the text.  The reco'd muon is required to pass the trigger muon ID of Sec.~\ref{sec:evtsel-triggermu}, but with the PF relative isolation (because this is the cut under study) and nearby lepton isolation requirements removed.  (\cmsLeft) Gen $\tau_{\text{2}}$ decays to an electron.  (middle) Gen $\tau_{\text{2}}$ decays to a muon.  (\cmsRight) Gen $\tau_{\text{2}}$ decays to hadrons.}
    \label{fig:IsoEffVsDR}
  \end{center}
\end{figure}

In contrast, Figure~\ref{fig:IsoEffVsDR_withFilters} shows $\epsilon_{\text{rel. iso.}}$ as a function of $\Delta$R(gen $a\rightarrow\tau\rightarrow\mu$ muon, gen $\tau_{\text{2}}$), where the gen $a\rightarrow\tau\rightarrow\mu$ muon is matched to the reco'd muon as described in Eq.~\ref{eq:muonPFRelIsoeff} above and the gen $\tau_{\text{2}}$ is the other tau from the $\cmsSymbolFace{a}\rightarrow\tau\tau$ decay.  The efficiency is calculated separately for each decay mode of the gen $\tau_{\text{2}}$ (electronic, muonic, or hadronic).  The efficiencies are much flatter in $\Delta$R\ when the nearby lepton isolation requirement is applied to the reconstructed trigger muon, because it insures that only events in which the two reconstructed taus from pseudoscalar decay are well separated may pass the analysis selection.  The PF relative isolation efficiency for $a\rightarrow\tau\rightarrow\mu$ muons in these events is now similar to that of $Z$ decay muons and is in the regime where the trigger muon is isolated and MC describes the data well.

\begin{figure}[hbtp]
  \begin{center}
    \includegraphics[width=0.8\cmsFigWidth]{figures/dRIsoEfficiency_withFilters_muEOnly}
    \includegraphics[width=0.8\cmsFigWidth]{figures/dRIsoEfficiency_withFilters_muMuOnly}
    \includegraphics[width=0.8\cmsFigWidth]{figures/dRIsoEfficiency_withFilters_muHadOnly}
    \caption{$\epsilon_{\text{rel. iso.}}$ for the ggH signal as a function of the separation $\Delta$R(gen $a\rightarrow\tau\rightarrow\mu$ muon, gen $\tau_{\text{2}}$), where the gen $\tau_{\text{2}}$ is a decay product of the same pseudoscalar as in the $a\rightarrow\tau\rightarrow\mu$.  The $a\rightarrow\tau\rightarrow\mu$ muon is matched to the reco'd muon as described in the text.  The reco'd muon is required to pass the trigger muon ID of Sec.~\ref{sec:evtsel-triggermu}, but with the PF relative isolation requirement removed (because this is the cut under study).  (\cmsLeft) Gen $\tau_{\text{2}}$ decays to an electron.  (middle) Gen $\tau_{\text{2}}$ decays to a muon.  (\cmsRight) Gen $\tau_{\text{2}}$ decays to hadrons.}
    \label{fig:IsoEffVsDR_withFilters}
  \end{center}
\end{figure}

After all other selection cuts, the acceptance of the nearby lepton isolation requirement ranges from 87\% to 95\% for ggH pseudoscalar masses 7, 9, 11, 13, and 15 GeV.

%boosted tau ID
\section{Boosted tau ID\label{sec:evtsel-ditau}}

The signature being sought in this analysis is one in which one of the $a$ decays results in a $\tau_{\mu}\tau_{\text{had}}$ final state, while no constraints are placed on the decay of the taus coming from the other $\cmsSymbolFace{a}$. The hadronic tau identification algorithm employed in this analysis is the HPS algorithm.

As described in Section~\ref{sec:semileptonic}, the tau pairs produced in the pseudoscalar decays will be highly collimated, and their decay products will invade one another's isolation cones. In particular, for the $\tau_{\mu}\tau_{\text{had}}$ pair, the muon from $\tau_{\mu}$ has been found to end up frequently among the constituents of the jet seeded by the $\tau_{\text{had}}$ decay and therefore among the isolation constituents of the $\tau_{\text{had}}$ reconstructed with HPS.  Figure~\ref{fig:evt-sel-HPS-iso-eff-standard-vs-boosted-ID} shows the $\tau_{\text{had}}$ isolation efficiency for the standard HPS ID versus the boosted $\tau_{\mu}\tau_{\text{had}}$ ID described below.  The isolation efficiency is about four times higher for the boosted $\tau_{\mu}\tau_{\text{had}}$ ID.

\begin{figure}[hbtp]
  \begin{center}
    \includegraphics[width=\cmsFigWidth]{figures/effVsPT_MCTruthMuonRemoval_LooseCombinedIsolationDBSumPtCorr_canvas}
    \caption{Hadronic tau isolation efficiency for the WH signal using the standard tau identification algorithm (black) and the boosted ID developed for this analysis (red).}
    \label{fig:evt-sel-HPS-iso-eff-standard-vs-boosted-ID}
  \end{center}
\end{figure}

To recover the correct reconstruction of the $\tau_{\text{had}}$, a method has been developed to identify soft muon candidates for the $\tau_{\mu}$ and remove them from the constituents of any jet that contained them, while the remaining jet constituents are then reconstructed into a jet and passed to the HPS algorithm to reconstruct a $\tau_{\text{had}}$ decay. The MUO POG soft muon ID is used to identify $\tau_{\mu}$ candidates.

%soft mu
\subsection{Soft muon ID\label{sec:evtsel-softmu}}

In addition to the trigger $\mu$ requirement, events are selected that have at least one reco muon passing the following cuts:
\begin{itemize}
	\item $p_T >$ 5 GeV
	\item $\abs{\eta} <$ 2.4
%	\item required to be \DR \textgreater 0.3 from the reco muon identified earlier as the $\PW_{\mu}$ (thus excluding the possibility that the $\tau_{\mu}$ and $\PW_{\mu}$ candidates are the same object)
	\item Distinct from the trigger muon
	\item Soft muon ID~\cite{CMS:2010uta}:
	\begin{itemize}
		\item Tracker muon track is matched with at least one muon segment in both \unit{x} and y coordinates
		\item More than 5 tracker layers with hits
		\item Number of pixel layers $>$ 1
		\item The tracker muon track fit has $\chi^{2}/\text{ndof} <$ 1.8
		\item The reco muon's tracker track has $d_{\text{xy}} <$ 3 mm and $d_{\text{z}} <$ 30 mm
	\end{itemize}
\end{itemize}
After all soft muons passing these requirements are collected, they are used as described in Section~\ref{sec:evtsel-tauID} to reconstruct $\tau_{\text{had}}$ objects.

%Jet cleaning and boosted tau ID
\subsection{Jet cleaning and hadronic tau ID\label{sec:evtsel-tauID}}

In this analysis, tau decays are reconstructed from the anti-$k_{\text{T}}$ R = 0.5~\cite{1126-6708-2008-04-063} PF jets (``ak5PFJets'') using the HPS algorithm. Before running HPS, jet constituents passing the soft muon ID (Sec.~\ref{sec:evtsel-softmu}) are removed. In the majority of cases, only one soft muon is removed from a jet, but if more than one muon is removed, the highest $p_T$ removed muon is identified as the $\tau_{\mu}$. A new ak5PFJet (henceforth referred to as the cleaned jet after the removal of the muon) is then reconstructed from the remaining PF constituents. These cleaned jets are submitted to the HPS algorithm and reconstructed as $\tau_{\text{had}}$ candidates.

The event is then selected to have at least one $\tau_{\text{had}}$ candidate reconstructed as above with $p_T >$ 20 GeV, $\abs{\eta} <$ 2.3, and passing the HPS \texttt{DecayModeFinding} and \texttt{\\MediumCombinedIsolationDBSumPtCorr} discriminators. HPS is currently capable of reconstructing the following hadronic tau decay modes: single hadron (one prong decay, or one prong plus one low-energy $\pi^0$), single hadron plus one ECAL strip (one prong plus one $\pi^0$), single hadron plus two strips (one prong plus one $\pi^0$ in which the photons from the $\pi^0$ decay are well separated in the ECAL), and three hadrons (three prong decay). In addition, because no anti-muon or anti-electron discriminators are applied to the HPS object, some leptonic tau decays get counted as single hadron decays.  The DecayModeFinding discriminator requires that the reconstructed HPS $\tau_{\text{had}}$ object have one of these four decay modes. Further selection on the isolation of the $\tau_{\text{had}}$ candidate helps to discriminate against fake taus reconstructed from quark or gluon jets, which tend to involve more hadronic activity and soft radiation and thus are less isolated.  The HPS isolation energy, described in \cite{CMS_AN_2010-082}, is defined by the $\Delta\beta$ pileup-corrected $E_T$ sum of the PF charged and neutral hadron and PF gamma candidates found within a $\Delta$R = 0.5 cone around the $\tau_{\text{had}}$ axis.  The \texttt{MediumCombinedIsolationDBSumPtCorr} discriminator~\cite{CMS:tauidtwiki} imposes an upper limit of 1.0 GeV on the HPS isolation energy.

In the majority of events passing all these selection cuts, there is only one $\tau_{\text{had}}$ passing both HPS discriminators after the $\tau_{\mu}$ is removed from the jet used to reconstruct it. If more than one $\tau_{\text{had}}$ object passes, then the one with the highest $p_T$ is taken to be the $\tau_{\text{had}}$. If more than one $\tau_{\mu}\tau_{\text{had}}$ object is found in the event, the one with the highest $\tau_{\mu}\tau_{\text{had}}$ invariant mass is chosen.

Only one $\tau_{\mu}\tau_{\text{had}}$ object is selected in this analysis. Requiring two such objects was tested, to see whether this could increase sensitivity to the WH signal, but this was observed to kill all MC background events, while only a handful of signal events and QCD control sample events survived.  It would have been impossible to model the background meaningfully.

%charge requirements
\section{Opposite charge muon veto\label{sec:evtsel-OSSF}}

The presence of two muons in the event, one of which is isolated and energetic, makes Drell-Yan di-muon production a large background.  To combat this, the trigger $\mu$ and $\tau_{\mu}$ are required to have the same electric charge.  Figures~\ref{fig:muHadMass_OSSFVeto_vs_none_lowMT} and~\ref{fig:muHadMass_OSSFVeto_vs_none_highMT} show the effect of this requirement on the Drell-Yan background, which is reduced by almost a factor of 20 in the low-$M_{T}$ bin and 13 in the high-$M_{T}$ bin, at a cost of reducing the signal by at most 50\%.

\begin{figure}[hbtp]
  \begin{center}
    \includegraphics[width=1.2\cmsFigWidth]{figures/muHadMass_lowMT_beforeOSSF}
    \includegraphics[width=1.2\cmsFigWidth]{figures/muHadMass_lowMT_afterOSSF}
    \caption{Invariant mass of the $\tau_{\mu}\tau_{\text{had}}$ pair for signals in the low-$M_{T}$ bin with $m_a$ = 9 GeV and all backgrounds before (\cmsLeft) and after (\cmsRight) the (trigger $\mu$)-$\tau_{\mu}$ same charge requirement.  All backgrounds except QCD are estimated from MC simulation. }
    \label{fig:muHadMass_OSSFVeto_vs_none_lowMT}
  \end{center}
\end{figure}

\begin{figure}[hbtp]
  \begin{center}
    \includegraphics[width=1.2\cmsFigWidth]{figures/muHadMass_highMT_beforeOSSF}
    \includegraphics[width=1.2\cmsFigWidth]{figures/muHadMass_highMT_afterOSSF}
    \caption{Invariant mass of the $\tau_{\mu}\tau_{\text{had}}$ pair in the high-$M_{T}$ bin for signals with $m_a$ = 9 GeV and all backgrounds before (\cmsLeft) and after (\cmsRight) the (trigger $\mu$)-$\tau_{\mu}$ same charge requirement.  All backgrounds except QCD are estimated from MC simulation. }
    \label{fig:muHadMass_OSSFVeto_vs_none_highMT}
  \end{center}
\end{figure}

\section{Same charge tau veto\label{sec:evtsel-SSSF}}

$\tau_{\mu}\tau_{\text{had}}$ pairs are reconstructed in background events when there is a poorly isolated real muon, either promptly produced or coming from a heavy flavor jet, or when one track in a light jet fakes a soft muon and the others fake an HPS tau.  Fake $\tau_{\mu}\tau_{\text{had}}$ pairs rarely come from a real boosted di-tau decay, and therefore no correlation between the $\tau_{\mu}$ and $\tau_{\text{had}}$ charge is expected.  We therefore impose an opposite charge requirement on the $\tau_{\mu}$ and $\tau_{\text{had}}$, which reduces the background by about 20\% while leaving the signal virtually unchanged (Figure~\ref{fig:muHadMass_SSSF}).

\begin{figure}[hbtp]
  \begin{center}
    \includegraphics[width=1.2\cmsFigWidth]{figures/muHadCharge_lowMT_beforeSSSF}
    \includegraphics[width=1.2\cmsFigWidth]{figures/muHadCharge_highMT_beforeSSSF}
    \caption{Sum of the $\tau_{\mu}$ charge and $\tau_{\text{had}}$ charge for signals with $m_a$ = 9 GeV and all backgrounds.  All backgrounds except QCD are estimated from MC simulation. (\cmsLeft) Low-$M_{T}$ bin. (\cmsRight) High-$M_{T}$ bin. }
    \label{fig:muHadMass_SSSF}
  \end{center}
\end{figure}

%b-veto
\section{B-veto on tau jet\label{sec:evtsel-bveto}}

Heavy flavour jets, such as those from B meson or top decays, often contain a muon among the decay products which gets reconstructed as the $\tau_{\mu}$. Thus, the identification of b-jets can serve as a means to reject background from heavy flavour jets.

To optimize the identification of b-jets, b-tagging algorithms take advantage of the unique properties that distinguish b-jets from other kinds of jets produced at the LHC. One important property is the long lifetime of B mesons; when they decay, they will have travelled a significant distance (on the order of millimeters) from the primary vertex, resulting in displaced secondary vertices.  Thus, the impact parameter and secondary vertex associated with such jets can be used as discriminating variables. The combined secondary vertex (CSV) algorithm~\cite{Weiser:927399}, which is used in this analysis, employs a likelihood ratio that takes as input information about the primary vertex, secondary vertex, 2D and 3D impact parameters, track multiplicity, and track pseudorapidities of jets.

Figure~\ref{fig:sigVsBkg_csv_regA} shows the distribution of the CSV discriminator for Monte Carlo signals and all backgrounds except QCD in the low-$M_{\text{T}}$ and high-$M_{\text{T}}$ bins. As can be seen from this figure, the $\tau_{\text{had}}$ jet in the signal final states generally does not have large values of CSV; by vetoing b-tagged jets, the single top and $t\bar{t}$ backgrounds can be cut down significantly. A b-tag veto was thus implemented by rejecting events in which the cleaned jet associated with the $\tau_{\text{had}}$ object had a CSV value greater than the medium CSV working point of 0.679 recommended by the BTV POG for \texttt{22Jan2013} re-reco data at 8 \TeV~\cite{CMS:btvpogtwiki}.

\begin{figure}[hbtp]
  \begin{center}
    \includegraphics[width=\cmsFigWidth]{figures/sigVsBkg_csv_regA_lowMT_v61}
    \includegraphics[width=\cmsFigWidth]{figures/sigVsBkg_csv_regA_highMT_v61}
    \caption{Distribution of the CSV discriminator for four signal models and all backgrounds, including data-driven QCD, after all the preselection cuts except the b veto have been applied. Normalized to 19.7 \fbinv. (\cmsLeft) Low-$M_{\text{T}}$ bin. (\cmsRight) High-$M_{\text{T}}$ bin.}
    \label{fig:sigVsBkg_csv_regA}
  \end{center}
\end{figure}

As shown in Figure~\ref{fig:sigVsBkg_csv_regA}, the distribution of the CSV discriminator for signal events tends to peak at low values of the discriminator, while single top and $t\bar{t}$ events with real b-jets peak at high values; thus, since no data/MC b-veto efficiency scale factors exist for tau jets, we make the approximation that the tau jets in our signal behave more like light jets with regard to the CSV discriminator, and so the expected signal is corrected for differences between data and MC b-veto efficiency using the BTV-provided scale factors~\cite{CMS:btaguncertaintytwiki} for light jets.  The scale factors are binned in HPS tau parent (cleaned) jet $\eta$ and $p_T$.

\begin{figure}[hbtp]
  \begin{center}
    \includegraphics[width=\cmsFigWidth]{figures/PassNN_mvis}
    \includegraphics[width=\cmsFigWidth]{figures/FailNN_mvis}
    \caption{$\tau_{\mu}\tau_{\text{had}}$ invariant mass plots showing Z peak in Run I data and MC, for events passing all Z peak selections including the medium combined isolation discriminator for $\tau_{\text{had}}$, and passing or failing the medium CSV b-tag applied to the jet that seeded the $\tau_{\text{had}}$. (\cmsLeft) Events passing the medium CSV b-tag. (\cmsRight) Events failing the medium CSV b-tag.}
    \label{fig:BTagZPeakPassFail}
  \end{center}
\end{figure}

Two methods are used to assess the error on the expected signal due to the uncertainty on the scale factors. Firstly, the expected signal is recalculated for a coherent +1$\sigma$ shift in the scale factors in all simulated events, and then again for a -1$\sigma$ shift; the difference between the nominal and $\pm1\sigma$-shifted expected signal is taken as the $\pm1\sigma$ error due to b-tag scale factors for light jets. Secondly, another systematic is calculated to account for the uncertainty of using light-jet scale factors for tau jets; the signal yields after the final selection are calculated using light-jet scale factors (the nominal method) and using b-jet scale factors (the logic being that the phase space for tau jets should be somewhere between the two extremes of light jets and b jets), and the percent difference in the yield is taken as a conservative uncertainty on the yield due to the usage of light-jet scale factors.

Since a b-veto is applied to a tau jet, the following cross-check has been performed to verify that the assigned systematic uncertainty for a potential data/MC discrepancy is adequate. First, a clean sample of of tau lepton candidates was obtained using $Z\rightarrow\tau_{\mu}\tau_{\text{had}}$ selections in Run I data and MC and the Z peak was reconstructed as per the methods in~\cite{Tau14001}, requiring that the $\tau_{\text{had}}$ object pass the medium combined isolation discriminator. Then, additionally, a b-tag at the medium CSV working point was applied to the jets that seeded the $\tau_{\text{had}}$, and two Z peaks were plotted -- one for events passing the b-tag and one for events failing the b-tag. The Z peak plots are shown in Figure~\ref{fig:BTagZPeakPassFail}; these results suggest that the data/MC agreement is unaffected by whether the tau jets pass or fail the CSV b-tag, and also it can be seen that the percentage of events in data and MC that pass the medium CSV b-tag is in the neighbourhood of 10\%, which is similar to the proportion of signal MC events observed to pass the medium CSV b-tag. Thus, these results lend confidence to the assumption that the requirement for the tau jet to pass the medium CSV b-veto does not significantly affect the known data/MC discrepancy covered by the present systematic uncertainty.

%PV
\section{Primary vertex compatibility requirement\label{sec:evtsel-dz}}
To reduce the background from $\tau_{\mu}\tau_{\text{had}}$ pairs in which the $\tau_{\mu}$ and $\tau_{\text{had}}$ come from different $pp$ interactions, we require $d_{\text{z}}(\tau_{\mu},\text{PV}) <$ 0.5 \cm and $d_{\text{z}}(\tau_{\text{had}},\text{PV}) <$ 0.2 cm, where PV refers to the hardest (primary) interaction in the event.  These cuts are recommended by the MUO and TAU POGs.  The $d_{\text{z}}(\tau_{\mu},\text{PV})$ and $d_{\text{z}}(\tau_{\text{had}},\text{PV})$ distributions for events passing all preselection cuts except those plotted are shown in Figures~\ref{fig:sigVsBkg_dz_regA_lowMT} and~\ref{fig:sigVsBkg_dz_regA_highMT}.

\begin{figure}[hbtp]
  \begin{center}
    \includegraphics[width=1.2\cmsFigWidth]{figures/sigVsBkg_dztaumu_regA_lowMT_v60}
    \includegraphics[width=1.2\cmsFigWidth]{figures/sigVsBkg_dztauhad_regA_lowMT_v60}
    \caption{Distribution of (\cmsLeft) dz($\tau_{\mu}$,PV) and (\cmsRight) dz($\tau_{\text{had}}$,PV) in the low-$M_{T}$ bin for four signal models and all backgrounds including data-driven QCD, after all the preselection cuts except the dz cuts have been applied. Normalized to 19.7 \fbinv.}
    \label{fig:sigVsBkg_dz_regA_lowMT}
  \end{center}
\end{figure}

\begin{figure}[hbtp]
  \begin{center}
    \includegraphics[width=1.2\cmsFigWidth]{figures/sigVsBkg_dztaumu_regA_highMT_v60}
    \includegraphics[width=1.2\cmsFigWidth]{figures/sigVsBkg_dztauhad_regA_highMT_v60}
    \caption{Distribution of (\cmsLeft) dz($\tau_{\mu}$,PV) and (\cmsRight) dz($\tau_{\text{had}}$,PV) in the high-$M_{T}$ bin for four signal models and all backgrounds including data-driven QCD, after all the preselection cuts except the dz cuts have been applied. Normalized to 19.7 \fbinv.}
    \label{fig:sigVsBkg_dz_regA_highMT}
  \end{center}
\end{figure}

%MT
\section{Transverse mass regions \label{sec:evtsel-mt}}

Figure~\ref{fig:sigVsBkg_MET_MT_regA} shows the $M_{\text{T}}$ distribution for four signal models and all backgrounds.  The gluon fusion signal is clustered at low $M_{\text{T}}$, where Drell-Yan and QCD are the most important backgrounds.  The WH signal can be found in the high $M_{\text{T}}$ bin, where $W$+jets and $t\bar{t}$ dominate the background.  We define $M_{\text{T}} \le$ 50 GeV as the low-$M_{\text{T}}$ bin of the analysis, sensitive to ggH and VBF, and $M_{\text{T}} >$ 50 GeV as the high-$M_{\text{T}}$ bin, sensitive to WH.  The cut was chosen to optimize S/$\sqrt{\text{S} + \text{B}}$ for the WH signal in the high-$M_{\text{T}}$ bin.  The $M_{\text{T}}$ in this analysis is calculated using Type I-corrected~\cite{1748-0221-6-11-P11002} PAT \ETslash, which is also used in the calculation of \ETslash systematics~\cite{METuncertainty}.

\begin{figure}[hbtp]
  \begin{center}
    \includegraphics[width=1.2\cmsFigWidth]{figures/sigVsBkg_MT_regA_v62}
    \caption{$M_{\text{T}}$ distribution after the preselection (excluding the $M_{\text{T}}$ cut) has been applied for four signal models and all backgrounds. The term ``$W$ muon'' in the label refers to the trigger muon, not necessarily a muon from a $W$ decay (as in the case of the ggH signal, for instance).}
    \label{fig:sigVsBkg_MET_MT_regA}
  \end{center}
\end{figure}

Following the JME approved procedure~\cite{METuncertainty}, uncertainty on the expected signal in each $M_{\text{T}}$ bin due to \ETslash scale is assessed by independently varying the e/$\gamma$, muon, tau, jet, and unclustered energy scales up and down by their approved 1$\sigma$ errors for each event in the signal sample.  \ETslash and $M_{\text{T}}$ are recalculated in each event, yielding an expected signal estimate in each of the +1$\sigma$ and -1$\sigma$ scenarios.  For each energy scale variation, the larger of the $\pm1\sigma$ deviations from nominal is taken as the error due to the uncertainty on that energy scale.  The quadrature sum of these individual errors is taken as the total $\pm1\sigma$ error due to \ETslash scale.

For technical reasons, the \ETslash definition for the $\pm1\sigma$ varied \ETslash collections and the nominal from which deviations are measured is slightly different from the \ETslash definition used when quoting the expected signal.  However, the deviations for the \ETslash uncertainty calculation are measured in a consistent way (same \ETslash definition for varied and nominal collections), and it is only the percent difference which is quoted as the \ETslash scale error.

%Final selection
\section{Search region \label{sec:evtsel-search}}
Table~\ref{tab:cut-flow-MC} shows the number of events surviving each successive cut in the selection sequence for the $m_a$ = 9 GeV WH and ggH signal samples and all background Monte Carlo samples (except for QCD Monte Carlo, due to poor statistics) used in the analysis optimization.  Table~\ref{tab:cut-flow-WHSignal} displays the selection efficiencies for the WH signal samples for each selection cut, expressed as the fraction of triggered signal events (i.e., events passing the HLT) surviving after each cut.  Table~\ref{tab:cut-flow-ggHSignal} shows the analogous selection efficiencies for the ggH signal samples.  The number of events is scaled to 19.7 fb$^{-1}$ using the cross sections given in Tables~\ref{tab:MCBkg} and~\ref{tab:MC-sig}.

%\begin{table*}[htbh]
\begin{sidewaystable}
\begin{center}
\caption{Number of events in MC signal and background datasets remaining after each cut in the selection sequence.  The signal samples have a pseudoscalar mass of 9 GeV.  The number of events is scaled to 19.7 \fbinv using the cross sections given in Tables~\ref{tab:MCBkg} and~\ref{tab:MC-sig}.  For the rows labeled ``$d_{\text{Z}}$ to PV'' and ``$m_{\mu+\text{had}} >$ 4 GeV'', pileup reweighting has been applied, while for the other rows, no pileup reweighting has been applied.\label{tab:cut-flow-MC}}
\resizebox{\columnwidth}{!}{
\begin{tabular}{|m{2.5cm}|c|c|c|c|c|c|c|c|c|c|c|}
  \hline
  \multicolumn{2}{|l|}{Cut} & WH & ggH & W+$\ge$1 jet & Drell-Yan + jets & $t\bar{t}$ + jets & Single top & WZ & ZZ & WW \\
  \hline
  \hline
  \multicolumn{2}{|l|}{} & 4.522\ten{3} & 6.721\ten{4} & 1.883\ten{8} & 3.587\ten{8} & 4.842\ten{6} & 1.825\ten{6} & 6.542\ten{5} & 3.478\ten{5} & 1.080\ten{6} \\
  \hline
  \multicolumn{2}{|l|}{\texttt{HLT\_IsoMu24\_eta2p1}} & 1.024\ten{3} & 6.616\ten{3} & 2.723\ten{7} & 1.507\ten{7} & 6.359\ten{5} & 1.124\ten{5} & 5.203\ten{4} & 1.937\ten{4} & 1.167\ten{5} \\
  \hline
  \multicolumn{2}{|l|}{Trigger $\mu p_T$} & 1.001\ten{3} & 6.083\ten{3} & 2.633\ten{7} & 1.459\ten{7} & 6.238\ten{5} & 1.091\ten{5} & 5.090\ten{4} & 1.913\ten{4} & 1.138\ten{5} \\
  \hline
  \multicolumn{2}{|l|}{Trigger $\mu \eta$, ID, and isolation} & 9.065\ten{2} & 4.417\ten{3} & 2.410\ten{7} & 1.381\ten{7} & 5.543\ten{5} & 9.936\ten{4} & 4.732\ten{4} & 1.816\ten{4} & 1.043\ten{5} \\
  \hline
  \multicolumn{2}{|l|}{$\tau_{\mu} p_T$, $\eta$, and ID} & 2.702\ten{2} & 2.600\ten{3} & 2.950\ten{5} & 3.705\ten{6} & 1.449\ten{5} & 1.430\ten{4} & 7.300\ten{3} & 6.842\ten{3} & 6.138\ten{3} \\
  \hline
  \multicolumn{2}{|l|}{HPS $\eta$ and ID} & 1.270\ten{2} & 9.838\ten{2} & 7.518\ten{4} & 9.074\ten{4} & 6.805\ten{4} & 6.508\ten{3} & 7.541\ten{2} & 6.257\ten{2} & 7.861\ten{2} \\
  \hline
  \multicolumn{2}{|l|}{HPS $p_T$ and isolation}                          & 39.06 & 216.0 & 1145  & 4338   & 798.3 & 85.56  & 24.53  & 24.35  & 13.18 \\
  \hline
  \multicolumn{2}{|l|}{b-veto}                              & 34.28 & 194.8 & 903.6 & 4170   & 273.2 & 20.40  & 20.41  & 21.93  & 10.26 \\
  \hline
  \multicolumn{2}{|l|}{HLT matching}                                   & 33.99 & 194.0 & 903.6 & 4088   & 266.1 & 20.40  & 19.95  & 21.33  & 10.05 \\
  \hline
  \multicolumn{2}{|l|}{$q(\tau_{\mu}) \times q(\text{trigger }\mu)$} & 16.70 & 92.56 & 171.0 & 93.21  & 81.60 & 8.365  & 3.075  & 1.455  & 1.404 \\
  \hline
  \multicolumn{2}{|l|}{$q(\tau_{\mu}) \times q(\tau_{\text{had}})$}     & 16.55 & 91.80 & 154.3 & 68.84  & 67.41 & 6.901  & 2.159  & 1.171  & 1.296 \\
  \hline
  \multicolumn{2}{|l|}{Trigger $\mu$ nearby lepton filter}       & 16.37 & 79.80 & 153.1 & 68.84  & 63.86 & 6.901  & 2.159  & 1.171  & 1.296 \\
  \hline
  \hline
  \multicolumn{2}{|l|}{\textbf{$M_{\text{T}} >$ 50 GeV}}              & 12.04 & 14.91 & 95.40 & 22.66  & 39.03 & 5.389  & 1.701  & 0.4968 & 0.8642 \\
  \hline
  \multicolumn{2}{|r|}{$d_{\text{Z}}$ to PV}                           & 11.77 & 13.52 & 84.15 & 20.85  & 25.10 & 4.306  & 1.783  & 0.4766 & 0.7122 \\
  \hline
  \multicolumn{2}{|r|}{$m_{\mu+\text{had}} >$ 4 GeV} & 6.972 & 7.598 & 1.761 & 0.4630 & 0     & 0      & 0.4038 & 0.1447 & 0 \\
  \hline
  \hline
  \multicolumn{2}{|l|}{\textbf{$M_{\text{T}} \le$ 50 GeV}}            & 4.326 & 64.89 & 57.67 & 46.18  & 24.84 & 1.512  & 0.4580 & 0.6743 & 0.4321 \\
  \hline
  \multicolumn{2}{|r|}{$d_{\text{Z}}$ to PV}                           & 4.225 & 60.79 & 50.30 & 32.03  & 17.62 & 1.839  & 0.5187 & 0.5394 & 0.3159 \\
  \hline
  \multicolumn{2}{|r|}{$m_{\mu+\text{had}} >$ 4 GeV} & 2.723 & 46.35 & 0     & 5.248  & 4.104 & 0.2781 & 0      & 0.1082 & 0 \\
  \hline
\end{tabular}
}
\end{center}
\end{sidewaystable}

\begin{sidewaystable}
\begin{center}
\caption{Selection efficiencies for MC WH signal samples, expressed as the fraction of triggered events surviving each selection cut.\label{tab:cut-flow-WHSignal}}
\resizebox{\columnwidth}{!}{
\begin{tabular}{|m{2.5cm}|c|c|c|c|c|c|c|}
  \hline
  \multicolumn{2}{|l|}{Cut} & $m_a$ = 5 GeV & $m_a$ = 7 GeV & $m_a$ = 9 GeV & $m_a$ = 11 GeV & $m_a$ = 13 GeV & $m_a$ = 15 GeV \\
  \hline
  \hline
  \multicolumn{2}{|l|}{Trigger $\mu p_T$} & 0.9780 & 0.9779 & 0.9769 & 0.9765 & 0.9754 & 0.9739 \\
  \hline
  \multicolumn{2}{|l|}{Trigger $\mu \eta$, ID, and isolation} & 0.8841 & 0.8841 & 0.8850 & 0.8801 & 0.8795 & 0.8802 \\
  \hline
  \multicolumn{2}{|l|}{$\tau_{\mu} p_T$, $\eta$, and ID} & 0.2898 & 0.2814 & 0.2638 & 0.2643 & 0.2620 & 0.2570 \\
  \hline
  \multicolumn{2}{|l|}{HPS $\eta$ and ID} & 0.1688 & 0.1476 & 0.1240 & 0.1189 & 0.1051 & 0.0914 \\
  \hline
  \multicolumn{2}{|l|}{HPS $p_T$ and isolation} & 0.0485 & 0.0439 & 0.0381 & 0.0379 & 0.0338 & 0.0282 \\
  \hline
  \multicolumn{2}{|l|}{b-veto} & 0.0407 & 0.0382 & 0.0335 & 0.0334 & 0.0305 & 0.0256 \\
  \hline
  \multicolumn{2}{|l|}{HLT matching} & 0.0404 & 0.0380 & 0.0332 & 0.0332 & 0.0303 & 0.0254 \\
  \hline
  \multicolumn{2}{|l|}{$q(\tau_{\mu}) \times q(\text{trigger }\mu)$} & 0.0204 & 0.0192 & 0.0163 & 0.0167 &  0.0152 & 0.0125 \\
  \hline
  \multicolumn{2}{|l|}{$q(\tau_{\mu}) \times q(\tau_{\text{had}})$} & 0.0202 & 0.0190 & 0.0162 & 0.0166 & 0.0150 & 0.0124 \\
  \hline
  \multicolumn{2}{|l|}{Trigger $\mu$ nearby lepton filter} & 0.0200 & 0.0188 & 0.0160 & 0.0165 & 0.0148 & 0.0124 \\
  \hline
  \multirow{2}{*}{\textbf{$M_{\text{T}} \le$  50 GeV}} & $d_{\text{Z}}$ to PV & 0.00575 & 0.00494 & 0.00413 & 0.00480 & 0.00364 & 0.00342 \\
  & $m_{\mu+\text{had}} >$ 4 GeV & 0.00114 & 0.00148 & 0.00266 & 0.00403 & 0.00329 & 0.00335 \\
  \hline
  \multirow{2}{*}{\textbf{$M_{\text{T}} >$  50 GeV}} & $d_{\text{Z}}$ to PV & 0.0140 & 0.0136 & 0.0115 & 0.0111 & 0.0106 & 0.00835 \\
  & $m_{\mu+\text{had}} >$ 4 GeV & 0.000101 & 0.00379 & 0.00681 & 0.00837 & 0.00932 & 0.00774 \\
  \hline
\end{tabular}
}
\end{center}
\end{sidewaystable}

\begin{sidewaystable}
\begin{center}
\caption{Selection efficiencies for MC ggH signal samples, expressed as the fraction of triggered events surviving each selection cut.\label{tab:cut-flow-ggHSignal}}
\resizebox{\columnwidth}{!}{
\begin{tabular}{|m{2.5cm}|c|c|c|c|c|c|c|}
  \hline
  \multicolumn{2}{|l|}{Cut} & $m_a$ = 5 GeV & $m_a$ = 7 GeV & $m_a$ = 9 GeV & $m_a$ = 11 GeV & $m_a$ = 13 GeV & $m_a$ = 15 GeV \\
  \hline
  \hline
  \multicolumn{2}{|l|}{Trigger $\mu p_T$} & 0.9147 & 0.9226 & 0.9194 & 0.9169 & 0.9156 & 0.9155 \\
  \hline
  \multicolumn{2}{|l|}{Trigger $\mu \eta$, ID, and isolation} & 0.6093 & 0.6491 & 0.6675 & 0.6979 & 0.7328 & 0.7623 \\
  \hline
  \multicolumn{2}{|l|}{$\tau_{\mu} p_T$, $\eta$, and ID} & 0.4308 & 0.4082 & 0.3931 & 0.3973 & 0.4091 & 0.4173 \\
  \hline
  \multicolumn{2}{|l|}{HPS $\eta$ and ID} & 0.1846 & 0.1679 & 0.1487 & 0.1319 & 0.1151 & 0.0918 \\
  \hline
  \multicolumn{2}{|l|}{HPS $p_T$ and isolation} & 0.0364 & 0.0345 & 0.0327 & 0.0302 & 0.0250 & 0.0166 \\
  \hline
  \multicolumn{2}{|l|}{b-veto} & 0.0321 & 0.0303 & 0.0294 & 0.0280 & 0.0232 & 0.0153 \\
  \hline
  \multicolumn{2}{|l|}{HLT matching} & 0.0316 & 0.0301 & 0.0293 & 0.0279 & 0.0230 & 0.0152 \\
  \hline
  \multicolumn{2}{|l|}{$q(\tau_{\mu}) \times q(\text{trigger }\mu)$} & 0.0154 & 0.0144 & 0.0140 & 0.0133 & 0.0117 & 0.0073 \\
  \hline
  \multicolumn{2}{|l|}{$q(\tau_{\mu}) \times q(\tau_{\text{had}})$} & 0.0153 & 0.0143 & 0.0139 & 0.0131 & 0.0116 & 0.0072 \\
  \hline
  \multicolumn{2}{|l|}{Trigger $\mu$ nearby lepton filter} & 0.0102 & 0.0117 & 0.0121 & 0.0117 & 0.0110 & 0.0068 \\
  \hline
  \multirow{2}{*}{\textbf{$M_{\text{T}} \le$  50 GeV}} & $d_{\text{Z}}$ to PV & 0.00816 & 0.00926 & 0.00914 & 0.00923 & 0.00787 & 0.00488 \\
  & $m_{\mu+\text{had}} >$ 4 GeV & 0.000089 & 0.00419 & 0.00701 & 0.00853 & 0.00763 & 0.00469 \\
  \hline
  \multirow{2}{*}{\textbf{$M_{\text{T}} >$  50 GeV}} & $d_{\text{Z}}$ to PV & 0.00112 & 0.00142 & 0.00204 & 0.00185 & 0.00234 & 0.00147 \\
  & $m_{\mu+\text{had}} >$ 4 GeV & 0 & 0.000381 & 0.00115 & 0.00139 & 0.00212 & 0.00130 \\
  \hline
\end{tabular}
}
\end{center}
\end{sidewaystable}

Compared to WH, \texttt{HLT\_IsoMu24\_eta2p1} rejects a large fraction of ggH events (factor 10 \vs factor 4).  However, the ggH cross section is 80 times larger than the WH cross section, making ggH an important signal in this analysis.  Once the HLT selection has been applied, the acceptance of the trigger muon selection, $\tau_{\mu}\tau_{\text{had}}$ selection, b-veto, and event-level cuts $q(\tau_{\mu}) \times q(\text{trigger }\mu) >$ 0 and $q(\tau_{\mu}) \times q(\tau_{\text{had}}) <$ 0 is larger for the WH samples than for the ggH samples by factors 1.3-2 depending on pseudoscalar mass.  A large portion of that difference is explained by the better acceptance of the trigger muon ID for $W$ decay muons than for ggH $a\rightarrow\tau\rightarrow\mu$ muons.

In both the WH and ggH samples, the trigger muon + $\tau_{\mu}\tau_{\text{had}}$ ID selects 1.7-4.9\% of triggered events depending on pseudoscalar mass.  The main contributors to this acceptance are the $\tau_{\mu}\tau_{\text{had}}$ decay mode requirement, high tau $p_T$ threshold of 20 GeV, and HPS isolation efficiency of $\sim$60\%.  For events with an identified trigger muon, the $\tau_{\mu}\tau_{\text{had}}$ ID accepts 4-5\% of signal events but only 1 in $10^{5} W$ + jets events, 1 in $10^{4}$ Drell-Yan + jets events, and 1 in 1000 $t\bar{t}$ events.  A drastic reduction in the Drell-Yan background comes from the requirement $q(\tau_{\mu}) \times q(\text{trigger }\mu) >$ 0, and about 65\% of the $t\bar{t}$ background is removed by the b-veto.

Signal versus background studies with Monte Carlo have shown that $m_{\mu+\text{had}}$, the invariant mass of the $\tau_{\mu}$ and $\tau_{\text{had}}$, provides good separation between the signal and the various backgrounds. The region $m_{\mu+\text{had}} <$ 2 GeV is primarily background-dominated, while most of the signal distribution is found in the region $m_{\mu+\text{had}} >$ 4 GeV.

The final selection consists of the preselection sequence followed by the requirement $m_{\mu+\text{had}} >$ 4 GeV. Events passing the final selection constitute the signal region, where a counting experiment will be performed. The background $m_{\mu+\text{had}}$ distribution for events passing the preselection has been shown -- using both Monte Carlo and QCD -- to be modelled well by events that pass all preselection cuts up to and failing the HPS $\tau_{\text{had}}$ isolation cut. Thus, the expected background for events in the signal region will be estimated by normalizing the signal-poor $m_{\mu+\text{had}} <$ 2 GeV sideband to match the background prediction and applying this normalization factor to the signal region.
 %event selection and its rationale
\chapter{Muon and tau selection efficiency validation\label{sec:lepideff}}
\sloppy

\section{Uncertainties on trigger muon data/MC scale factors\label{sec:lepideff-triggermu}}

%tight id uncertainty
\subsection{Tight muon ID\label{lepideff-tightID}}

The tight muon ID used in the trigger muon selection is one of the standard ones used by the CMS experiment, so the officially-provided data/MC efficiency scale factor error of 0.5\% is used in the exclusion limit calculation (cf. Secs.~\ref{sec:results-systematics} and~\ref{sec:results-interpretation}).

%HLT uncertainty
\subsection{HLT\label{lepideff-HLT}}

For the WH and ZH signals, the error on the \texttt{HLT\_IsoMu24\_eta2p1} efficiency data/MC scale factor is taken as 0.2\%, the standard value officially approved by CMS.  As the standard scale factors were computed for isolated $Z$ decay muons, they are applicable to the isolated $W$($Z$) decay muons present in the WH(ZH) signal samples.

The trigger efficiency for the ggH and VBF samples is discussed in Sec.~\ref{sec:lep-id-eff-ggH-HLT}.  With the nearby lepton isolation requirement on the reconstructed trigger muon, the trigger efficiency for ggH and VBF $a\rightarrow\tau\to\mu$ muons is in the regime where the trigger muon is isolated and MC describes the data well.  A data/MC scale factor of 1 is applied to the ggH and VBF HLT efficiencies, with systematic uncertainty due to the small remaining inefficiency in the $\tau_{\mu}\tau_{e}$ mode taken as the difference ($\epsilon_{\text{HLT}}$($\Delta$R $>$ 0.4) - $\epsilon_{\text{HLT}}$($\Delta$R $>$ 0))$/$100 ($\epsilon_{\text{HLT}}$ and $\Delta$R are defined in Eq.~\ref{eq:muonHLTeff} and Sec.~\ref{sec:lep-id-eff-ggH-HLT}, respectively).  In the calculation of the error, all gen $\tau_{\text{2}}$ (see Sec.~\ref{sec:lep-id-eff-ggH-HLT}) decay modes are integrated over.  The uncertainty obtained is 4.2\%.

%uncertainty from nearby lepton filter
\subsection{Nearby lepton isolation\label{lepideff-leptonveto}}

The nearby lepton isolation requirement is a veto on the presence of any reconstructed electron, muon, or tau within $\Delta$R = 0.4 of the trigger muon, where the electron, muon, and tau selection criteria are summarized in Sec.~\ref{sec:evtsel-triggermu}.  The selection criteria are standard within CMS, so three additional uncertainties are used in the exclusion limit setting to cover the standard data-MC scale factor errors for the three lepton selections.  They are 1\% (electrons, cf.~\cite{CMS:egammauncertaintytwiki}), 1.5\% (muons, cf.~\cite{CMS:muonuncertaintytwiki}), and 10\% (taus).  For the tau ID, Ref.~\cite{CMS:tauuncertaintytwiki} recommends an uncertainty of 6\% for reconstructed taus with $p_T >$ 20 GeV, which was increased to a conservative 10\% to cover the difference in MC decay mode finding efficiency for 10 GeV $< p_T <$ 20 GeV between isolated hadronic taus from $Z\rightarrow\tau\tau$ and hadronic taus in $\tau_{\mu}\tau_{\text{had}}$ objects in the WH signal sample.

%isolation uncertainty
\subsection{Particle flow relative isolation\label{lepideff-iso}}

For the WH and ZH signals, the error on the PF relative isolation efficiency data/MC scale factor is taken as 0.2\%, the standard value officially approved by CMS.  As the standard scale factors were computed for isolated $Z$ decay muons, they are applicable to the isolated $W$($Z$) decay muons present in the WH(ZH) signal samples.

The PF relative isolation efficiency for the ggH and VBF samples is discussed in Sec.~\ref{sec:muon-id-eff-iso}. With the nearby lepton isolation requirement on the reconstructed trigger muon, the PF relative isolation efficiency for ggH and VBF $a\rightarrow\tau\rightarrow\mu$ muons is in the regime where the trigger muon is isolated and MC describes the data well.  A data/MC scale factor of 1 is applied to the ggH and VBF HLT efficiencies, with systematic uncertainty due to the small remaining inefficiency in the $\tau_{\mu}\tau_{\text{had}}$ mode taken as the difference ($\epsilon_{\text{rel. iso.}}$($\Delta$R $>$ 0.4) - $\epsilon_{\text{rel. iso.}}$($\Delta$R $>$ 0))$/$100 ($\epsilon_{\text{rel. iso.}}$ and $\Delta$R are defined in Eq.~\ref{eq:muonPFRelIsoeff} and Sec.~\ref{sec:muon-id-eff-iso}, respectively).  In the calculation of the error, all gen $\tau_{\text{2}}$ (see Sec.~\ref{sec:muon-id-eff-iso}) decay modes are integrated over.  The uncertainty obtained is 3.8\%.

%section on tau_mu tau_had id efficiency studies
\section{$\tau_{\mu}\tau_{\text{had}}$\label{lepid-eff-muPlusX}}

The soft muon and HPS tau IDs used in this search are standard within CMS.  However, they are used here in a nonstandard way, in particular for the special case where the soft muon and HPS tau are nearly overlapping. %In order to understand how these IDs perform in the signal environment, soft muon and HPS tau efficiencies for the boosted tau signal are compared to the efficiencies for the same IDs in the physics processes for which they were developed: $Z\rightarrow\tau\tau$ for the HPS ID and $\cPJgy\to\mu\mu$ for the soft muon ID.  As shown below, the comparisons yield little difference, so the data/MC ID efficiency scale factors and their errors are taken straight from the POG recommendations.
The soft muon and HPS tau efficiencies for the boosted tau signal have been studied in order to understand how these IDs perform in the signal environment. The HPS tau ID efficiency in the signal process is compared to the efficiency in the $Z\rightarrow\tau\tau$ process for which it was developed.  As shown below, the soft muon ID efficiency is quite high, so the data/MC ID efficiency scale factors and their errors are taken straight from official CMS recommendations. The HPS tau ID efficiency for the signal is generally similar to the efficiency measured in $Z\rightarrow\tau\tau$ events, with some discrepancy in the lower-$p_T$ region, an increased uncertainty on the data/MC efficiency scale factor is used.

All signal efficiency studies are performed with a Monte Carlo sample of WH signal events, with $m_{a}$ = 9 GeV, generated as in Sec.~\ref{sec:datasets}.  Signal events are required to pass the isolated muon trigger and have at least one reconstructed trigger muon according to the criteria in Sec.~\ref{sec:evtsel-triggermu}.

%soft muon id efficiency
\subsection{Soft muon ID efficiency\label{sec:soft-mu-id}}

The soft muon efficiency $\epsilon_{\text{soft}}$ = (number of gen-matched muons with $p_T >$ 5 GeV, $\abs{\eta} <$ 2.1, and passing the soft ID)/(number of gen-matched muons with $p_T >$ 5 GeV, $\abs{\eta} <$ 2.1) is shown in Figure~\ref{fig:soft_muon} for WH signal events.  Gen-matching is done within a cone of $\Delta$R = 0.3 around the reconstructed soft muon.  The soft muon ID includes the requirement that the soft muon be distinct from the trigger muon, as described in Sec.~\ref{sec:evtsel-softmu}.

\begin{figure}[hbtp]
  \begin{center}
    \includegraphics[width=\cmsFigWidth]{figures/soft_eta_eff}
    \includegraphics[width=\cmsFigWidth]{figures/soft_pt_eff}
    \caption{Soft muon efficiency as a function of $\eta$ (\cmsLeft) and $p_T$ (\cmsRight) in WH signal events. Errors are statistical only.}
    \label{fig:soft_muon}
  \end{center}
\end{figure}

The WH soft muon efficiency is $\sim$95\% across a range of $\eta$ and $p_T$ and is in good qualitative agreement with soft muon efficiencies measured in CMS data $J\slash\psi\rightarrow\mu\mu$ events~\cite{1748-0221-7-10-P10002}.  As the officially-measured soft muon efficiency agrees with the value from $J\slash\psi\rightarrow\mu\mu$ simulation~\cite{CMS:muonrefeffstwiki} within the quoted error, the signal WH and ggH MC is not corrected for differences from data.  Instead, recommended error of 1.5\% is propagated to the error on the expected signal.

%HPS tau id efficiency (Minsoo)
\subsection{HPS tau\label{sec:HPS-id}}

The MC sample \texttt{/DYJetsToLL\_M-50\_TuneZ2star\_8TeV-madgraph-tarball/\\Summer12\_DR53X-PU\_S10\_START53\_V7A-v1/AODSIM} is used to calculate HPS tau efficiency on $Z\rightarrow\tau_{\mu}\tau_{\text{had}}$ events.  The $\tau_{\mu}$ leg of the $Z$ decay is required to fire \texttt{HLT\_IsoMu24\_eta2p1} and pass the trigger muon ID.  HPS decay mode finding and isolation efficiency are measured on the $\tau_{\text{had}}$ leg.  For WH signal events, in addition to the trigger muon ID described above, the HPS tau is required to be built from a jet cleaned of a soft muon (cf. Sec.~\ref{sec:evtsel-ditau}).

The $p_T$ distributions of gen-level taus matched to reconstructed HPS taus are shown in Figure~\ref{fig:gen_whtt} for the WH sample and Figure~\ref{fig:gen_ztt} for the $Z\rightarrow\tau\tau$ sample.  Gen-matching is performed in a cone of $\Delta$R = 0.3 around the reconstructed HPS tau.  Signal WH taus tend to be softer than $Z$ decay taus, yet their ID and isolation efficiencies are similar as shown below.

\begin{figure}[hbtp]
  \begin{center}
    \includegraphics[width=\cmsFigWidth]{figures/pT_whtt_DMF}
    \caption{$p_T$ of gen taus from WH $\tau_{\mu}\tau_{\text{had}}$ pairs matched to reconstructed HPS taus with associated soft muons (cf. Sec.~\ref{sec:evtsel-ditau}).  Errors are statistical only.  (\cmsLeft) No discriminator requirement.  (\cmsRight) DecayModeFinding requirement.}
    \label{fig:gen_whtt}
  \end{center}
\end{figure}

\begin{figure}[hbtp]
  \begin{center}
    \includegraphics[width=\cmsFigWidth]{figures/gentau_pt_ztt_dmf}
    \caption{$p_T$ of gen taus from $Z\rightarrow\tau\tau$ decay matched to standard reconstructed HPS taus.  Errors are statistical only.  (\cmsLeft) No discriminator requirement.  (\cmsRight) DecayModeFinding requirement.}
    \label{fig:gen_ztt}
  \end{center}
\end{figure}

The decay mode finding efficiency $\epsilon_{\text{DMF}}$ = (number of gen-matched HPS taus in $\abs{\eta} <$ 2.4 passing the DecayModeFinding discriminator)/(number of gen-matched HPS taus in \abs{\eta} \textless\xspace 2.4) is shown for WHand $Z\rightarrow\tau\tau$ events in Figure~\ref{fig:eff_dmf} (\cmsLeft).  There is good agreement across a range of $\eta$ and $p_T$ between the simulated efficiency for signal boosted tau pair events reconstructed with the cleaning procedure described in Sec.~\ref{sec:evtsel-tauID} and $Z\rightarrow\tau_{\mu}\tau_{\text{had}}$ events.  In particular, the agreement is good even for relatively low tau $p_T$ ($<$ 20 GeV).  Similarly, the decay mode finding and isolation efficiency $\epsilon_{\text{DMF+iso}}$ = (number of gen-matched HPS taus in $\abs{\eta} <$ 2.4 passing the DecayModeFinding and MediumCombinedIsolationDBSumPtCorr discriminators)/(number of gen-matched HPS taus in $\abs{\eta} <$ 2.4) is shown in Figure~\ref{fig:eff_dmf_mi}. There is qualitative agreement with publicly approved efficiencies for simulated $Z\rightarrow\tau\tau$ events~\cite{CMS:approvedTAUResults}.

\begin{figure}[hbtp]
  \begin{center}
    \includegraphics[width=\cmsFigWidth]{figures/gentau_dmf_pt}
    \includegraphics[width=\cmsFigWidth]{figures/gentau_dmf_eta}
    \caption{(\cmsLeft) HPS decay mode finding efficiency as a function of matched gen tau $p_T$. (\cmsRight) HPS decay mode finding efficiency as a function of matched gen tau $\eta$.  Signal HPS taus (blue) are reconstructed using the soft muon cleaning procedure described in this document, while taus from $Z$ decay (red) are reconstructed with standard HPS. Errors are statistical only.}
    \label{fig:eff_dmf}
  \end{center}
\end{figure}

\begin{figure}[hbtp]
  \begin{center}
    \includegraphics[width=\cmsFigWidth]{figures/gentau_dmfmi_pt}
    \includegraphics[width=\cmsFigWidth]{figures/gentau_dmfmi_eta}
    \caption{(\cmsLeft) HPS decay mode finding + medium combined isolation efficiency as a function of matched gen tau $p_T$. (\cmsRight) HPS decay mode finding efficiency as a function of matched gen tau $\eta$.  Signal HPS taus (blue) are reconstructed using the soft muon cleaning procedure described in this document, while taus from $Z$ decay (red) are reconstructed with standard HPS. Errors are statistical only.}
    \label{fig:eff_dmf_mi}
  \end{center}
\end{figure}

%As in the case of the soft muon, differences in the TAU POG approved HPS decay mode finding and isolation efficiencies between data and MC are within errors~\cite{CMS:tauuncertaintytwiki}, so instead of correcting the signal MC we simply propagate the recommended error of 6\% to the expected signal.
To cover discrepancies of up to about 10\% between the efficiencies in the signal and in $Z\rightarrow\tau_{\mu}\tau_{\text{had}}$, a conservative systematic of 10\% is applied to the HPS tau ID efficiency data-MC scale factor when a $p_T$ cut of 10 GeV is used. The $p_T$ cut used for the $\tau_{\text{had}}$ from the $\tau_{\mu}\tau_{\text{had}}$ object is 20 GeV; however, this efficiency study is still important because a $p_T$ cut of 10 GeV is applied to taus in the neighbouring lepton veto for the trigger muon (in order to have a better efficiency for the veto), and because future iterations of this search may explore the possibility of lowering the $p_T$ cut on the $\tau_{\text{had}}$ from $\tau_{\mu}\tau_{\text{had}}$ as well, as was originally intended.

Since the HPS tau ID efficiencies and scale factors have been officially validated only down to 20 GeV, a study was done to reconstruct the \Z peak using HPS taus with $p_T$ between 10 and 20 GeV and to compare it to the $Z$ peak reconstructed from HPS taus with $p_T >$ 20 GeV, to assess the reliability of using taus with $p_T$ between 10 and 20 GeV. $Z\rightarrow\tau_{\mu}\tau_{\text{had}}$ events were selected in the MC sample \texttt{/DYJetsToLL\_M-50\_TuneZ2star\_8TeV-madgraph-tarball/\\Summer12\_DR53X-PU\_S10\_START53\_V7A-v1/AODSIM}; the $\tau_{\mu}$ leg of the \Z decay wass required to fire \texttt{HLT\_IsoMu24\_eta2p1}, pass the trigger muon ID, and be gen-matched to the $Z$ decay, while the gen-matched HPS tau was required to pass DecayModeFinding and MediumCombinedIsolationDBSumPtCorr discriminators and a $p_T$ cut of either $>$ 20 GeV for the standard case, or between 10 and 20 GeV for the low-$p_T$ case of interest. As shown in Figure~\ref{fig:Zpeakstudy}, the Z peak looks normal for HPS tau $p_T$ \textgreater\xspace 20 GeV, and the Z peak shape for the low-$p_T$ range looks normal aside from being biased to a lower mean due to the lower HPS tau $p_T$ cut.

\begin{figure}[hbtp]
  \begin{center}
    \includegraphics[width=\cmsFigWidth]{figures/gen_muhad_mass_1020}
    \includegraphics[width=\cmsFigWidth]{figures/gen_muhad_mass_20}
    \caption{Z peak reconstructed in a sample of Drell-Yan MC events. (\cmsLeft) HPS tau $p_T$ between 10 and 20 GeV. (\cmsRight) HPS tau $p_T <$ 20 GeV.}
    \label{fig:Zpeakstudy}
  \end{center}
\end{figure}
\chapter{Background modelling\label{sec:bkg}}
 % bkg modelling, rationale, and validation
\chapter{Results and interpretation\label{sec:results}}

\section{Systematic uncertainties\label{sec:results-systematics}}

The following is a list of the sources of systematic uncertainty used in the calculation of the total uncertainty in this search, some of which have been described in previous sections. In the limit calculation, these systematics are all treated as nuisance parameters affecting only the scale of the expected signal or background yields, and they are modelled with log-normal distributions.
%A comprehensive list of uncertainties can be found in Appendix~\ref{sec:errors}.

\begin{itemize}
\item \textbf{Luminosity: } As assessed in summer 2013~\cite{CMS-PAS-LUM-13-001}, the uncertainty on the integrated luminosity is taken to be 2.6\%.
\item \textbf{Muon trigger efficiency: } According to the CMS MUO POG~\cite{CMS:muonuncertaintytwiki}, the systematic uncertainty from the single-muon trigger \texttt{HLT\_IsoMu24\_eta2p1} is 0.2\% for the WH and ZH signals. For the ggH and VBF signals, because of the effect of the nearby lepton filter applied to the trigger muon, a larger systematic uncertainty of 4.2\% is applied (see Sec.~\ref{lepideff-HLT} for details).
\item \textbf{Tight muon ID efficiency: } According to the CMS MUO POG~\cite{CMS:muonuncertaintytwiki}, the systematic uncertainty on the trigger muon tight ID is 0.5\%.
\item \textbf{Muon isolation efficiency: } According to the CMS MUO POG~\cite{CMS:muonuncertaintytwiki}, the systematic uncertainty on the trigger muon isolation is 0.2\%. This is applied to signal events in the WH and ZH channels. For the ggH and VBF channels, an uncertainty of 10\% is used instead, to account for the fact that the muon which fires the trigger comes from a boosted $\tau_{\mu}\tau_{\text{had}}$ topology, and that the isolation efficiency for the trigger muon is largely recovered if the nearby reconstructed tau is subtracted from its isolation cone; the 10\% figure is taken from the Tau POG recommendation for the HPS tau ID efficiency for this boosted configuration.
\item \textbf{Soft muon ID efficiency: } According to the CMS MUO POG~\cite{CMS:muonuncertaintytwiki}, the systematic uncertainty on the $\tau_{\mu}$ ID is 1.5\%.
\item \textbf{HPS ID efficiency: } The accepted value of 6\% from the CMS TAU POG is used~\cite{CMS:tauuncertaintytwiki}.
\item \textbf{Tau charge misidentification rate: } The accepted value of -1\%/+2\% from the CMS TAU POG is used~\cite{CMS:tauuncertaintytwiki}.
\item \textbf{b-veto efficiency: } Two systematic uncertainties are considered for the b-veto efficiency. The first uncertainty stems from the fact that b-veto data/MC scale factors for light jets are applied to the tau jets on which the b-veto is applied; since the actual data/MC scale factors are expected to be somewhere between light jets and b-jets, the percent difference in signal yields when using light jet scale factors and when using b-jet scale factors is taken as a systematic uncertainty, and the magnitude ranges between 1.8-8.5\% depending on the $M_{T}$ bin and signal process. The second source of systematic uncertainty comes from the uncertainty on the light-jet scale factors used; following the BTV recommendations, the scale factors are shifted coherently by $\pm$1$\sigma$, and the difference between the nominal and shifted expected signal yields is taken as the systematic uncertainty; errors range up to 5.2\% depending on signal sample. Because the VBF signal is expected to have a similar selection efficiency as the ggH channel, the errors calculated for each mass point in the ggH channel are applied to the VBF prediction for each analogous mass point; for similar reasons, the errors calculated for the WH channel are applied to the ZH channel.
\item \textbf{ID efficiencies for nearby lepton filter around trigger muon: } Systematics are assigned to the ID efficiency data/MC scale factors of the PF electrons, muons, and taus used for the neighbouring lepton veto around the trigger muon. For the PF electrons, since no ID is applied beyond the requirement that they pass PF reconstruction, have $p_T >$ 7 GeV, and $\abs{\eta} <$ 2.5, we apply a conservative error of 1.1\%, based on the highest uncertainty for the low-$p_T$ electrons passing Loose ID requirements~\cite{CMS:egammauncertaintytwiki}. For the PF muons, since the same soft ID is used as for the reconstructed $\tau_{\mu}$, the same systematics uncertainty of 1.5\% is applied~\cite{CMS:muonuncertaintytwiki}. For the PF taus, a conservative uncertainty of 10\% is used. This came from the suggestion of the CMS TAU POG for the ID efficiency of HPS taus reconstructed from jets via the jet-cleaning method (cf. Section~\ref{sec:evtsel-tauID}) with $p_T >$ 10 GeV; this value was estimated by taking the accepted uncertainty of 6\% from the TAU POG~\cite{CMS:tauuncertaintytwiki}, adding in quadrature the discrepancy of at least 1\% observed between the HPS tau ID efficiencies for our signal and Drell-Yan MC events in the studies described in Chapter~\ref{sec:lepideff}, and rounding upwards.
\item \textbf{Background: } To obtain the final jet-faking-tau background prediction in the $m_{\mu+\text{had}}$ $>$ 4 GeV bin in Region A, we take the unweighted average of the nominal background prediction from Region B and the predictions from the alternative non-QCD (from MC) or all-QCD (from region D data) background shapes. For the systematic uncertainty on this background prediction, we look at the central value $+$($-$)1$\sigma$ of the alternative background shapes and compare it to the final background prediction; the greatest positive (negative) difference between the final background prediction and one of these values is then taken to be the positive (negative) systematic error on the final background prediction. This results in asymmetric systematic errors of +77.6\%/-85.0\% for the low-$M_{T}$ bin and +60.9\%/-59.2\% for the high-$M_{T}$ bin.
\item \textbf{$M_{\text{T}}$: } Following the JME recommendations, errors range up to 12.2\% depending on signal sample. Just as for the b-veto errors, the MET errors calculated for each mass point in the ggH channel are applied to the VBF prediction for each analogous mass point, while the errors calculated for the WH channel are applied to the ZH channel.
\item \textbf{VBF and ZH predictions: } The expected signal yields from the VBF and ZH channels were calculated by scaling the ggH and WH expected yields (from MC) respectively to the appropriate SM cross-sections. To account for the extra element of uncertainty introduced by this indirect method of estimation, the percent difference between the number of VBF (or ZH) events from MC and from the indirect estimation method after the full selection at the 9 GeV pseudoscalar mass point (the only mass point for which MC samples were generated for the VBF and ZH channels) was taken as an estimate of the error on the VBF and ZH predictions for all pseudoscalar mass points. For the low-$M_{T}$ bin, the errors were 23.2\% for VBF and 19.1\% for ZH; for the high-$M_{T}$ bin, the errors were 25.3\% for VBF and 24.3\% for ZH.
\end{itemize}

\section{Observed and expected results\label{sec:results-obsexp}}

Table~\ref{tab:results} shows the expected numbers of signal events from each generated pseudoscalar mass point in each of the $M_{\text{T}}$ bins, followed by the background prediction (obtained from Region B) and the actual number of events observed in Region A for $m_{\mu+\text{had}}$ $>$ 4 GeV.

\begin{table*}[htbh]
\begin{center}
\caption{Observed data, estimated background, and expected signal from each generated pseudoscalar mass point in each of the $M_{\text{T}}$ bins assuming SM cross sections and 100\% Br($H\rightarrow$$aa$$\rightarrow4\tau$)  Only statistical error is shown for the signal, while the full error is shown for the background.\label{tab:results}}
\singlespacing
\begin{tabular}{|c|c|c|c|}
\hline
\multicolumn{2}{|c|}{} & $M_{\text{T}} \le$ 50 GeV & $M_{\text{T}} >$ 50 GeV \\
\hline
\multirow{5}{*}{WH} & $m_{a}$ = 5 GeV & 0.11 $\pm$ 0.05 & 0.10 $\pm$ 0.04 \\
& $m_{a}$ = 7 GeV & 1.5 $\pm$ 0.2 & 3.8 $\pm$ 0.3 \\
& $m_{a}$ = 9 GeV & 2.7 $\pm$ 0.2 & 7.0 $\pm$ 0.3 \\
& $m_{a}$ = 11 GeV & 4.2 $\pm$ 0.3 & 8.8 $\pm$ 0.4 \\
& $m_{a}$ = 13 GeV & 3.5 $\pm$ 0.2 & 9.9 $\pm$ 0.4 \\
& $m_{a}$ = 15 GeV & 3.6 $\pm$ 0.2 & 8.4 $\pm$ 0.4 \\
\hline
\multirow{5}{*}{ggH} & $m_{a}$ = 5 GeV & 0.31 $\pm$ 0.22 & 0 \\
& $m_{a}$ = 7 GeV & 21 $\pm$ 2 & 1.9 $\pm$ 0.6 \\
& $m_{a}$ = 9 GeV & 46 $\pm$ 3 & 7.6 $\pm$ 1.1 \\
& $m_{a}$ = 11 GeV & 64 $\pm$ 3 & 11 $\pm$ 1 \\
& $m_{a}$ = 13 GeV & 63 $\pm$ 3 & 18 $\pm$ 2 \\
& $m_{a}$ = 15 GeV & 41 $\pm$ 3 & 11 $\pm$ 1 \\
\hline
\multirow{5}{*}{ZH} & $m_{a}$ = 5 GeV & 0.03 $\pm$ 0.01 & 0.03 $\pm$ 0.01 \\
& $m_{a}$ = 7 GeV & 0.38 $\pm$ 0.04 & 1.0 $\pm$ 0.1 \\
& $m_{a}$ = 9 GeV & 0.68 $\pm$ 0.05 & 1.9 $\pm$ 0.1 \\
& $m_{a}$ = 11 GeV & 1.1 $\pm$ 0.05 & 2.3 $\pm$ 0.1 \\
& $m_{a}$ = 13 GeV & 0.88 $\pm$ 0.06 & 2.7 $\pm$ 0.1 \\
& $m_{a}$ = 15 GeV & 0.91 $\pm$ 0.06 & 2.3 $\pm$ 0.1 \\
\hline
\multirow{5}{*}{VBF} & $m_{a}$ = 5 GeV & 0.03 $\pm$ 0.02 & 0 \\
& $m_{a}$ = 7 GeV & 2.3 $\pm$ 0.2 & 0.22 $\pm$ 0.06 \\
& $m_{a}$ = 9 GeV & 5.1 $\pm$ 0.3 & 0.9 $\pm$ 0.1 \\
& $m_{a}$ = 11 GeV & 7.0 $\pm$ 0.4 & 1.2 $\pm$ 0.1 \\
& $m_{a}$ = 13 GeV & 6.9 $\pm$ 0.4 & 2.0 $\pm$ 0.2 \\
& $m_{a}$ = 15 GeV & 4.5 $\pm$ 0.3 & 1.3 $\pm$ 0.2 \\
\hline
\multicolumn{2}{|c|}{SM Background} & \begin{tabular}[c]{@{}l@{}}5.41 $\pm$ 1 \stat \\$^{+4.2}_{-4.6}$ \syst\end{tabular} & \begin{tabular}[c]{@{}l@{}}6.08 $\pm$ 1.6 \stat \\$^{+3.7}_{-3.6}$ \syst\end{tabular} \\
\multicolumn{2}{|c|}{Data (observed)} & 7 & 14 \\
\hline
\end{tabular}
\end{center}
\end{table*}

\section{Interpretation\label{sec:results-interpretation}}

\subsection{CLs limit calculation\label{sec:results-limitcalc}}

At the end of the selection sequence, when the final $m_{\mu+\text{had}}$ background prediction (obtained as described in Sec.~\ref{sec:bkgs-jet-fake-unc}) has been normalized according to Sec.~\ref{sec:evtsel-search}, the number of events passing $m_{\mu+\text{had}} >$ 4 GeV are counted. This integral serves as the predicted number of events from background only. The integral of the region A data $m_{\mu+\text{had}}$ distribution above 4 GeV is then compared to the prediction derived from region B. In the context of this analysis, a significant excess in the region A data would be interpreted as coming from the 2HDM signal.  Assuming no excess, limits are set on the cross section times branching ratio for SM Higgs production and decay to lighter Higgses, which subsequently decay to taus.

Methods from the \texttt{HiggsAnalysis/CombinedLimit} package (full code found at~\cite{CombinedGitHub}, documented in~\cite{CombinedTwiki}) are used to evaluate expected and observed limits in this analysis. The signal strengths calculated by  are defined as:

\begin{equation}
r_{\text{prod}} = \frac{\sigma_{\text{prod}}\cdot B(H\rightarrow aa \rightarrow4\tau)}{(\sigma_{\text{prod}})_{\text{expected}}} \\
\label{eq:rDefinition}
\end{equation}

where $\sigma_{\text{prod}}$ is the production cross-section for the channel of interest and the denominators are calculated using the SM Higgs production cross sections given in~\cite{LHCHXSWG} (e.g., $\sigma$(ggH) = 19.27 fb$^{-1}$ and $\sigma$(WH) = 0.7046 fb$^{-1}$ for $m_{H}$ = 125.0 GeV). For the cases of WH and ZH, the production cross-sections are considered to be multiplied by the appropriate SM branching ratio for the decay of the vector boson to leptons.

A 1-dimensional limit calculation with the $\text{CL}_{\text{S}}$ method~\cite{Read} was used to obtain observed and expected upper limits for each $M_{T}$ bin and pseudoscalar mass. For each $M_{T}$ bin, limits were calculated for the signal strength parameter corresponding to the combination of all four signal channels (ggH, WH, VBF, and ZH).

\subsection{Model-independent limits\label{sec:results-limits}}

Model-independent expected and observed limits were calculated using the total expected yield from all four signal channels, with the \texttt{Asymptotic} routine in \texttt{combine}~\cite{springerlink:10.1140/epjc/s10052-011-1554-0} using a pre-fit Asimov dataset to calculate the $\text{CL}_{\text{S}}$ expected limit on the signal strength. Plots of the observed and expected $\text{CL}_{\text{S}}$ limits (median, $\pm1\sigma$, and $\pm2\sigma$) for the low-$M_{T}$ bin, high-$M_{T}$ bin, and combination of the two bins at different $m_{\cmsSymbolFace{a}}$ points are shown in Figure~\ref{fig:lowhighMTCLs}. The limits are reported in terms of the total branching ratio Br($H\rightarrow$$aa\rightarrow4\tau$), assuming SM Higgs production cross-sections.

\begin{figure}[hbtp]
  \begin{center}
    \includegraphics[width=\cmsFigWidth]{figures/expLimits_Br_lowMT_20GeV_ggHVBF}
    \includegraphics[width=\cmsFigWidth]{figures/expLimits_Br_highMT_20GeV_ggHWH}
    \includegraphics[width=\cmsFigWidth]{figures/expLimits_Br_20GeV_4sig}
    \caption{Observed 95\% C.L. limits (solid black curve) on the branching ratio Br($H\rightarrow$$aa\rightarrow4\tau$), compared to expected limits (dotted black curve, with $\pm1\sigma$ bands in green and $\pm2\sigma$ bands in yellow) at pseudoscalar mass points $m_{\cmsSymbolFace{a}}$ = 5 through 15 GeVcc.  (Top \cmsLeft) $M_{\text{T}}$ \textless\xspace 50 GeV. (Top \cmsRight) $M_{\text{T}}$ \textgreater\xspace 50 GeV.  (Bottom) Combined result between the low- and high-$M_{\text{T}}$ bins.}
    \label{fig:lowhighMTCLs}
  \end{center}
\end{figure}

The model-independent observed and expected $\text{CL}_{\text{S}}$ limits, expressed in terms of limits on Br($H\rightarrow$$aa\rightarrow4\tau$) assuming SM Higgs production cross-sections are tabulated in Tables~\ref{tab:CLs-limits-lowMT}-\ref{tab:CLs-limits}.

\begin{table*}
\begin{center}
  \caption{Observed and expected $\text{CL}_{\text{S}}$ limits on Br($H\rightarrow$$aa\rightarrow4\tau$) assuming SM Higgs production cross-sections in the low-$M_{\text{T}}$ bin.}
%\resizebox{\columnwidth}{!}{%
\singlespacing
\begin{tabular}{|m{3cm}|c|c|c|c|c|c|c|}
  \hline
  $m_{\cmsSymbolFace{a}}$ (GeV) & -2$\sigma$ & -1$\sigma$ & Median expected & +1$\sigma$ & +2$\sigma$ & Observed \\
  \hline
  \hline
  5 & 13.6 & 17.0 & 22.3 & 30.4 & 41.7 & 24.3 \\
  \hline
  7 & 0.255 & 0.318 & 0.420 & 0.576 & 0.787 & 0.457 \\
  \hline
  9 & 0.119 & 0.148 & 0.196 & 0.268 & 0.367 & 0.213 \\
  \hline
  11 & 0.0852 & 0.107 & 0.141 & 0.193 & 0.266 & 0.153 \\
  \hline
  13 & 0.0875 & 0.110 & 0.144 & 0.197 & 0.272 & 0.157 \\
  \hline
  15 & 0.130 & 0.163 & 0.214 & 0.294 & 0.404 & 0.234 \\
  \hline
  \end{tabular}%
%}
\label{tab:CLs-limits-lowMT}
\end{center}
\end{table*}

\begin{table*}
\begin{center}
  \caption{Observed and expected $\text{CL}_{\text{S}}$ limits on Br($H\rightarrow$$aa\rightarrow4\tau$) assuming SM Higgs production cross-sections in the high-$M_{\text{T}}$ bin.}
%\resizebox{\columnwidth}{!}{%
\singlespacing
\begin{tabular}{|m{3cm}|c|c|c|c|c|c|c|}
  \hline
  $m_{\cmsSymbolFace{a}}$ (GeV) & -2$\sigma$ & -1$\sigma$ & Median expected & +1$\sigma$ & +2$\sigma$ & Observed \\
  \hline
  \hline
  5 & 49.2 & 64.7 & 89.6 & 132 & 186 & 145 \\
  \hline
  7 & 0.883 & 1.15 & 1.59 & 2.33 & 3.23 & 2.56 \\
  \hline
  9 & 0.355 & 0.465 & 0.643 & 0.942 & 1.31 & 1.03 \\
  \hline
  11 & 0.271 & 0.355 & 0.490 & 0.719 & 1.005 & 0.789 \\
  \hline
  13 & 0.192 & 0.251 & 0.347 & 0.508 & 0.711 & 0.559 \\
  \hline
  15 & 0.262 & 0.345 & 0.475 & 0.696 & 0.973 & 0.765 \\
  \hline
  \end{tabular}%
%}
\label{tab:CLs-limits-highMT}
\end{center}
\end{table*}

\begin{table*}
\begin{center}
  \caption{Observed and expected $\text{CL}_{\text{S}}$ limits on Br($H\rightarrow$$aa\rightarrow4\tau$) assuming SM Higgs production cross-sections for the combination of the low- and high-$M_{\text{T}}$ bins.}
%\resizebox{\columnwidth}{!}{%
\singlespacing
\begin{tabular}{|m{3cm}|c|c|c|c|c|c|c|}
  \hline
  $m_{\cmsSymbolFace{a}}$ (GeV) & -2$\sigma$ & -1$\sigma$ & Median expected & +1$\sigma$ & +2$\sigma$ & Observed \\
  \hline
  \hline
  5 & 13.2 & 16.7 & 21.8 & 29.7 & 40.4 & 26.0 \\
  \hline
  7 & 0.248 & 0.312 & 0.408 & 0.560 & 0.765 & 0.491 \\
  \hline
  9 & 0.114 & 0.144 & 0.189 & 0.259 & 0.352 & 0.230 \\
  \hline
  11 & 0.0825 & 0.103 & 0.137 & 0.187 & 0.258 & 0.165 \\
  \hline
  13 & 0.0817 & 0.103 & 0.136 & 0.185 & 0.252 & 0.171 \\
  \hline
  15 & 0.120 & 0.152 & 0.200 & 0.273 & 0.371 & 0.254 \\
  \hline
  \end{tabular}%
%}
\label{tab:CLs-limits}
\end{center}
\end{table*} % observed data vs expectation, limits (incl. explanation of CLs)
\chapter{Conclusions\label{sec:conclusions}}

A search for the decay $H\rightarrow$$aa/hh\rightarrow4\tau$ was performed in the gluon fusion, $W$ associated, $Z$ associated, and vector boson fusion production modes.  No evidence of this exotic decay was found, and upper limits on the branching ratio to new physics, assuming SM production of the 125 GeV Higgs, were set.  For a 9 GeV pseudoscalar, an upper limit of 19\% was set on $BR(\PH\to4\tau)$. This result is the first of its kind at the LHC.
%Talk about possibilities for future follow-ups to this research.

\begin{appendices}
\chapter{N-subjettiness\label{sec:nsubj}}

Highly boosted particles can result from the decay of high-mass resonances, whose production in the LHC is made possible by the high collision energies achievable. The decay products of a boosted object appear as a collimated spray of tracks in the detector; with a sufficiently high boost factor and thus sufficiently high collimation, these decay products can be reconstructed as a single jet and are thus not identified as distinct objects. Techniques for probing jet substructure are important for identifying and analyzing boosted jets. One method is the use of N-subjettiness ($\tau_{N}$), a parameter that measures the degree to which the energy within a jet is aligned along N candidate subjet axes\newline
The formula for N-subjettiness is as follows:

\begin{equation}
\tau_{N} = \cfrac{1}{d_{0}}\sum_{k}^{}p_{T,k}min(\Delta R_{1,k},\Delta R_{2,k},\dots,\Delta R_{N,k})
\label{eq:Nsubj}
\end{equation}

\noindent The index \emph{k} goes over all the constituent particles in the jet, $p_{T,k}$ is the transverse momentum of the $k^{th}$ particle, and $\Delta$R$_{n,k}$ is the distance in $\eta-\phi$ space between the $k^{th}$ particle and the axis of the $n^{th}$ candidate subjet. The term $d_{0}$ is given by

\begin{equation}
d_{0} = \sum_{k}^{}p_{T,k}R_0
\label{eq:d0}
\end{equation}

\noindent where $R_{0}$ is the radius used in the original jet clustering algorithm \cite{Thaler:2010tr}.\newline
In a jet whose particles are closely aligned with N or fewer subjets, the terms $p_{T,k}$min($\Delta$R$_{1,k}$, $\Delta$R$_{2,k}$, \dots, $\Delta$R$_{N,k}$) in the sum will be very small, and thus $\tau_{N}$ will be closer to zero, while in a jet whose energy is distributed away from the N subjet axes will have a larger value of $\tau_{N}$ and must have at least N + 1 subjets.

N-subjettiness has been used successfully in the identification of boosted objects such as top quarks and $W$ bosons. A new use of N-subjettiness for identifying jets seeded by boosted tau pairs (referred to as boosted ditau jets) was probed in a theoretical study by Englert et al. \cite{Englert:2011iz}, which suggested that the ratio $\tau_{3}$/$\tau_{1}$ could provide discrimination between boosted ditau jets and QCD jets. This study used 14-TeV Monte Carlo signal and background samples under conditions of zero pileup, where the signal process was  $h_{1}\rightarrow 2a_{1}\rightarrow 4\tau$, with all inclusive tau decay modes considered.

In this study, N-subjettiness ratios $\tau_{3}/\tau_{1}$, $\tau_{2}/\tau_{1}$, $\tau_{1}/\tau_{2}$, $\tau_{2}/\tau_{3}$, and $\tau_{3}/\tau_{4}$  were explored for their possible discriminatory power. An important issue that arose was the influence of pileup on the N-subjettiness distribution, which tends to impair the discriminatory power of N-subjettiness ratios for the signal; for instance, the mean of the $\tau_{3}/\tau_{1}$ distribution was observed to increase with increasing pileup for signal Monte Carlo, causing it to become increasingly indistinguishable from the $\tau_{3}/\tau_{1}$ distribution for jets from $W$+NJets events. Jet pruning was used to remove pileup from jets, and although this recovered some of the discriminatory power when comparing unit-normalized signal and $W$+NJets Monte Carlo $\tau_{3}/\tau_{1}$ distribution shapes, further analysis -- comparing signal Monte Carlo and all other background Monte Carlo samples except QCD, after applying pileup reweighting and all the preselection cuts to these events -- has not shown significant discrimination between signal and background.

Also, as a result of the jet pruning, significant number of jets were left with only 3 or fewer constituents, resulting in $\tau_{3}/\tau_{1}$ values of exactly 0. This suggests that less aggressive methods of pileup removal from jets should be explored. Figures~\ref{fig:nsubjettiness-ratios-lowMT} and~\ref{fig:nsubjettiness-ratios-highMT} show the distributions of two N-subjettiness ratios, $\tau_{3}/\tau_{1}$ and $\tau_{1}/\tau_{2}$, for the low and high $M_{T}$ bins, illustrating both the currently insufficient discriminatory power of these variables and the problematic peak at zero (for $\tau_{3}/\tau_{1}$) caused by jet pruning. Further investigation will eventually be required to find a more effective method of pileup removal and potentially an improvement in discriminatory power for N-subjettiness ratios.

\begin{figure}[hbtp]
  \begin{center}
    \includegraphics[width=1.2\cmsFigWidth]{figures/sigVsBkg_t3t1_lowMT}
    \includegraphics[width=1.2\cmsFigWidth]{figures/sigVsBkg_t1t2_lowMT}
    \caption{Examples of N-subjettiness ratio distributions for the low-$M_{T}$ bin, comparing distributions for two signal models and all backgrounds discussed in Sec.~\ref{sec:evtsel} including data-driven QCD, after all the preselection cuts have been applied. (\cmsLeft) $\tau_{3}/\tau_{1}$. (\cmsRight) $\tau_{1}/\tau_{2}$.}
    \label{fig:nsubjettiness-ratios-lowMT}
  \end{center}
\end{figure}

\begin{figure}[hbtp]
  \begin{center}
    \includegraphics[width=1.2\cmsFigWidth]{figures/sigVsBkg_t3t1_highMT}
    \includegraphics[width=1.2\cmsFigWidth]{figures/sigVsBkg_t1t2_highMT}
    \caption{Examples of N-subjettiness ratio distributions for the high-$M_{T}$ bin, comparing distributions for two signal models and all backgrounds discussed in Sec.~\ref{sec:evtsel} including data-driven QCD, after all the preselection cuts have been applied. (\cmsLeft) $\tau_{3}/\tau_{1}$. (\cmsRight) $\tau_{1}/\tau_{2}$.}
    \label{fig:nsubjettiness-ratios-highMT}
  \end{center}
\end{figure}

\chapter{Pixel detector geometry description\label{sec:matbudg}}

The CMS detector simulation uses an XML schema called the Detector Description Language (DDL) to encode the description of the detector geometry and material composition~\cite{CMS_AN_2005-000}. Together with two other auxiliary packages, the Algorithm Description Language (ADL) and Configuration Description Language (CDL), DDL interfaces with GEANT4 to provide the volumes, positions, and material compositions of the simulated detector elements.

All the various components and subcomponents of the CMS detector form a geometrical hierarchy, with each individual object being a subcomponent of some larger whole. In a system of XML files, DDL defines the basic data structures for describing the dimensions and materials of these parts as well as their geometrical hierarchies.

Two different classes of material definitions exist in DDL. Elementary materials correspond to elements of the periodic table and are identified by name, periodic table symbol, density, atomic number, and atomic weight. Definitions of composite materials are built by specifying their fractional composition in terms of elementary materials, or even other composite materials; the data structure allows one to customize the density associated with a particular composite material. The radiation length and hadronic interaction length of composite materials are calculated from the material definitions in the XML files; these parameters are needed ifor the modelling of particle interactions with detector components.

Detector parts are defined by their material, their shape, and their position in the detector. The various types of 3-dimensional shapes (such as rectangular boxes, trapezoids, and cylinders) allowed in this package are based on the GEANT syntax. Various parameters such as angles and Cartesian coordinates encode a detector component's spatial position, often with respect to a larger structure of which it is a component. Algorithms from ADL, referred to as DDAlgorithms, can be used to position multiple copies of a detector component in a specific pattern, to represent symmetrical or repeating structures.

The rest of this chapter treats the projects involving the CMS pixel detector geometry description, in which I have participated.

\section{Pilot system simulation\label{sec:matbudg-pilot}}
%1. Pilot system simulation (mention of Phase I upgrade, LS1, installation of pilot system, how it was added to the description, figures)

From February 2013 to February 2015, the LHC was turned off. During this period of planned off-time, repairs and maintenance were done on the CMS detector. The pixel detector was extracted from the experimental cavern in order for problematic panels and electronics to be fixed or replaced. While it was out, one extra endcap disk was installed on the -z side of the forward pixel detector (FPIX); 8 prototype modules for the planned Phase I upgrade~\cite{Dominguez:1481838} were mounted on the blades of this extra disk. Circuit boards called portcards were also installed, for reading out the prototype modules. All together, the extra disk and its associated cables and electronics make up the pilot system.

The tracker geometry and material description at the time did not include a description of the pilot system. In order to have an accurate representation of the material distribution (often referred to as the material budget) in the detector for generating MC events with this new configuration of the pixel detector, the geometry description was updated accordingly. A disk on the -z endcap was cloned into the position where the new disk was installed; new objects were defined in DDL to represent the shape and material of the new modules and their support structures. The large uniform blocks roughly representing the FPIX portcard electronics in the service cylinder also were updated, since the material composition had changed due to the addition of copper-containing DC-DC converters and other new electronics for the pilot system's readout. A new composite material representing the new average material composition was defined.

\section{Phase I pixel simulation\label{sec:matbudg-phase1}}
%2. Updates to Phase I description (ongoing; mostly done, describe what is being corrected)

During the next long shutdown at the end of the LHC's Run II, the CMS detector will undergo another round of detector upgrades known as the Phase I upgrade. The pixel detector will acquire one new barrel layer and one new endcap on each side of the interaction point. The cooling system, which currently uses C$_{6}$F$_{14}$ as a coolant, will be replaced with a new cooling system that uses cold carbon dioxide instead. The geometry of the forward pixel endcap support structures will be radically changed, in order to decrease the material budget while still providing full coverage with no dead space.

The CMS geometry description currently has XML files that describe the Phase I pixel geometry and material. The barrel pixel part is mostly accurate aside from a few minor changes involving the dimensions of some support structures and the addition of aluminium cabling in some regions. However, the forward pixel part is very inaccurate and needs significant updating in order to represent the actual upgraded components that will be installed. The main challenges are the following:

\begin{itemize}
\item The current description of the FPIX endcap support rings are flat, uniform rings, whereas the actual support rings have a zigzagging shape. This is extremely difficult to render using the shapes allowed in DDL. A tentative solution is being developed and tested, in which a series of ``infinitesimally" thin blocks are positioned in the form of a zigzag-shaped ring using a DDAlgorithm that gives each block an appropriate displacement along z as a function of $\phi$.
\item The zigzag pattern of the blades in the rings is incorrect for some of the rings; comparisons with blueprints of the actual Phase I FPIX rings are currently underway to determine the correct pattern. Also, in simulation, the endcap disks on the +z side are obtained by rotating the -z disks about the y axis, whereas the actual Phase I disks have a mirror symmetry about the xy plane instead.
\item The shapes of the modules and their support structures need to be refined.
\item The composite material used to describe the Phase I portcard objects is the same material used to describe the portcard objects in the Run I geometry before the installation of the pilot system. This is clearly incorrect, and a new composite material needs to be declared that matches the average composition of the actual Phase I FPIX portcards and DC-DC converters.
\end{itemize}

I have been involved in this project since spring 2015, and it is still ongoing.
\end{appendices}

\listoffigures\newpage
\listoftables\newpage

\bibliography{thesisbib}

% The UMI abstract uses square brackets!
\UMIabstract[In this dissertation, I present a search for non-standard decays of a Standard Model-like Higgs boson to pairs of light bosons, as predicted in models with extended Higgs sectors. In two Higgs doublet models, including the next-to-minimal supersymmetric standard model, the Higgs boson can decay into a pair of light pseudoscalars $a$.

In this search, the gluon fusion, $W$ and $Z$ associated Higgs, and vector boson fusion production channels for the Higgs are all considered, and the decay $H\rightarrow$$aa$ with $a\rightarrow\tau\tau$ is reconstructed from the tau decay products. The final state is characterized by one isolated high $p_T$ muon plus at least one highly boosted pair of taus, of which one of the taus is required to decay to a muon.

Using 19.7 fb$^{-1}$ of 8 TeV center of mass $pp$ collision data recorded by the Compact Muon Solenoid experiment at the Large Hadron Collider, a counting experiment is performed in a region of high di-tau invariant mass. We have found no excess of events above the Standard Model backgrounds, and the observed data is used to set upper limits on the branching ratio Br($H\rightarrow$$aa$)Br$^{2}(a\rightarrow\tau\tau)$. These results are equally applicable to decays of the SM-like Higgs boson to a pair of light scalars $h$.]

\end{document}
