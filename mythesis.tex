\documentclass[12pt,draftcls]{ucdavisthesis}

% PLEASE READ THE MANUAL - ucdavisthesis.pdf (in the package installation directory)

%%%%%%%%%%%%%%%%%%%%%%%%%%%%%%%%%%%%%%%%%%%%%%%%%%%%%%%%%%%%%%%%%%%%%%%%
%                                                                      %
%               LATEX COMMANDS FOR DOCUMENT SETUP                      %
%                                                                      %
%%%%%%%%%%%%%%%%%%%%%%%%%%%%%%%%%%%%%%%%%%%%%%%%%%%%%%%%%%%%%%%%%%%%%%%%

%\usepackage{bookmark}
\usepackage[us,nodayofweek,12hr]{datetime}
\usepackage{graphicx}
%\usepackage[square,comma,numbers,sort&compress]{natbib}
%\usepackage{hypernat}
% Other useful packages to try
\usepackage{amsmath}
\usepackage{amssymb}
%
% Different fonts to try (uncomment only fontenc and one font at a time)
% (you may need to install these first)
%\usepackage[T1]{fontenc} %enable fontenc package if using one of the fonts below
%\usepackage[adobe-utopia]{mathdesign}
%\usepackage{tgschola}
%\usepackage{tgbonum}
%\usepackage{tgpagella}
%\usepackage{tgtermes}
%\usepackage{fourier}
%\usepackage{fouriernc}
%\usepackage{kmath,kerkis}
%\usepackage{kpfonts}
%\usepackage[urw-garamond]{mathdesign}
%\usepackage[bitstream-charter]{mathdesign}
%\usepackage[sc]{mathpazo}
%\usepackage{mathptmx}
%\usepackage[varg]{txfonts}
%packages for CMS symbols
\usepackage{xspace}
\usepackage{ptdr-definitions}
\hyphenation{dis-ser-ta-tion blue-print man-u-script pre-par-ing} %add hyphenation rules for words TeX doesn't know


%\renewcommand{\rightmark}{\scriptsize A University of California Davis\ldots \hfill Rev.~\#1.0 \quad Compiled: \currenttime, \today}
% a fancier running header that can be used with draftcls options

%%%%%%%%%%%%%%%%%%%%%%%%%%%%%%%%%%%%%%%%%%%%%%%%%%%%%%%%%%%%%%%%%%%%%%%%
%                                                                      %
%        DOCUMENT SETUP AND INFORMATION FOR PRELIMINARY PAGES          %
%                                                                      %
%%%%%%%%%%%%%%%%%%%%%%%%%%%%%%%%%%%%%%%%%%%%%%%%%%%%%%%%%%%%%%%%%%%%%%%%

%\title          {A University of California Davis\\
%                 Dissertation/Thesis LaTeX Class File}
\title          {Search in Tau Final States for Higgs Decays to New Light Scalars\\
                 or Pseudoscalars in pp Collisions at $\sqrt{s} =$ 8 TeV}
%Exact title of your thesis. Indicate italics where necessary by underlining or using italics. Please capitalize the first letter of each word that would normally be capitalized in a title.

\author         {Francesca Shun-Ning Annarosa Ricci-Tam}
%Your full name as it appears on University records. Do not use initials.

\authordegrees  {B.S. (University of California, Davis) 2010 \\
                 M.S. (University of California, Davis) 2012}
%Indicate your previous degrees conferred.

\officialmajor  {Physics}
%This is your official major as it appears on your University records.

\graduateprogram{Physics}
%This is your official graduate program name. Used for UMI abstract.

\degreeyear     {2016}
% Indicate the year in which your degree will be officially conferred.

\degreemonth    {June}
% Indicate the month in which your degree will be officially conferred. Used for UMI abstract.

\committee{Maxwell Chertok, Chair}{Robin Erbacher}{Michael Mulhearn}{}{}
% These are your committee members. The command accepts up to five committee members so be sure to have five sets of braces, even if there are empties.

%%%%%%%%%%%%%%%%%%%%%%%%%%%%%%%%%%%%%%%%%%%%%%%%%%%%%%%%%%%%%%%%%%%%%%%%

%\copyrightyear{2020}
%\nocopyright

%%%%%%%%%%%%%%%%%%%%%%%%%%%%%%%%%%%%%%%%%%%%%%%%%%%%%%%%%%%%%%%%%%%%%%%%

\dedication{\textsl{To my family.}}

%%%%%%%%%%%%%%%%%%%%%%%%%%%%%%%%%%%%%%%%%%%%%%%%%%%%%%%%%%%%%%%%%%%%%%%%

\abstract{The abstract submitted as part of your dissertation, in the introductory pages, does not have a word limit. It should follow the same format as the rest of your dissertation (1.5 inch left margin, double-spaced, consecutive page numbering, etc.).}

%%%%%%%%%%%%%%%%%%%%%%%%%%%%%%%%%%%%%%%%%%%%%%%%%%%%%%%%%%%%%%%%%%%%%%%%

\acknowledgments{Acknowledgements to those who helped you get to this point. They should be listed by chapter when appropriate.}

%%%%%%%%%%%%%%%%%%%%%%%%%%%%%%%%%%%%%%%%%%%%%%%%%%%%%%%%%%%%%%%%%%%%%%%%

% Each chapter can be in its own file for easier editing and brought in with the \include command.
% Then use the \includeonly command to speed compilation when working on a particular chapter.
%%% \includeonly{ucdavisthesis_example_Chap1}

\begin{document}

\newcommand{\bibfont}{\singlespacing}
% need this command to keep single spacing in the bibliography when using natbib

\bibliographystyle{unsrtnat}
%many other bibliography styles are available (IEEEtran, mla, etc.). Use one appropriate for your field.

\makeintropages %Processes/produces the preliminary pages

\chapter{Introduction\label{sec:intro}}

%Standard Model
% - What does it consist of (particles, interactions, gauge fields)
% - Description of Higgs mechanism and its significance
% - Examples of SM limitations (DM, gravity, Higgs loop corrections...)

%Supersymmetry
% - What is supersymmetry, what SM problems does it address
% - 2HDM, NMSSM: Motivations
% - 

\subsection{The Standard Model\label{sec:SM}}

The Standard Model is a quantum field theory that provides the most successful description to date of all experimentally observed fundamental particles and their interactions. In this framework, the fundamental particles are all treated as excitations of quantum fields, and interactions between particles occur via the exchange of mediating particles~\cite{BettiniPhysics}.
Fundamental particles can be classified into three main categories: leptons, quarks, and gauge bosons. The leptons and quarks are all fermions -- particles with half-integer spin whose dynamics obey the Dirac equation. Leptons have integer electric charge and fall into three categories called ``flavours", each of which can be regarded as a representation of the SU(2)xU(1) group. The negatively charged leptons are the electron, the muon, and the tau (in increasing order of mass), and each is associated with a massless neutrino; for each lepton, there is also an associated antiparticle. Quarks have fractional electric charge and fall into three categories known as ``colour", which can be regarded as representations of the SU(3) group.
Particles interact via four fundamental forces: the strong force, electromagnetism, the weak force, and gravity. These interactions are mediated by the third category of fundamental particle, the gauge bosons. The strong force is mediated by gluons, which are colorless, electrically neutral, and massless; only quarks and gluons can participate in strong interactions. Any charged particle can participate in electromagnetic interactions, which are mediated by colorless, electrically, neutral, massless photons. The weak force, which is responsible for nuclear decays, is mediated by the W bosons, which can have positive or negative charge, and Z bosons, which are electrically neutral.
An illustration of the classification of Standard Model particles is shown in Fig.~\ref{fig:StandardModelTable}.

\begin{figure}
   \begin{center}
      \includegraphics[width=0.4\textwidth]{figures/StandardModelTable}
      \caption{Standard Model particles.}
      \label{fig:StandardModelTable}
   \end{center}
\end{figure}

\subsection{Deficiencies of the Standard Model\label{sec:SMdeficiencies}}


Although the Standard Model has been successful in describing a wide range of experimental results, it also falls short of providing a complete description of nature in many respects. It describes only three out of the four fundamental forces; the way in which gravity, which is $10^{-32}$ times weaker than the weak force, factors into the Standard Model is still unknown. It fails to provide a satisfactory description of neutrino oscillations, an explanation of the matter-antimatter asymmetry in the universe, or a suitable candidate (or candidates) for dark matter.

The Standard Model also predicts that the mass of the Higgs boson receives loop corrections -- corrections of a magnitude close to the Planck scale at which the three fundamental forces are expected to be unified. The fact that the experimentally measured Higgs boson mass is 125 GeV -- orders of magnitude smaller than the corrections -- means that these loop corrections must somehow be cancelled, but it is not known how. This constitutes what is known as the hierarchy problem of particle physics.

\subsection{Supersymmetry}

One theory that has been proposed to address the hierarchy problem is that there exists a certain symmetry, referred to a supersymmetry, that relates fermions to bosons. For each fermion, there would exist a corresponding bosonic parter particle, and likewise for each boson there would exist a fermionic partner; under a supersymmetric transformation, a fermion would turn into its bosonic partner and vice versa. Thus, for each loop correction to the Higgs mass, there would be another loop correction from the supersymmetric partner particle; since fermionic loops have a sign opposite that of bosonic loops, the loop corrections would cancel neatly.

\subsubsection{MSSM}

\subsubsection{NMSSM}

\chapter{Experiment description\label{sec:experiment}}

\section{Large Hadron Collider\label{sec:lhc}}
% - What is the LHC: roperties and purposes

\section{Compact Muon Solenoid\label{sec:cms}}
% - What is CMS

\subsection{Tracker\label{sec:cmstracker}}

\subsection{Electromagnetic calorimeter\label{sec:cmsecal}}

\subsection{Hadronic calorimeter\label{sec:cmshcal}}

\subsection{Magnet\label{sec:cmsmagnet}}

\subsection{Muon system\label{sec:cmsmuon}}

\section{Data acquisition and analysis at CMS\label{sec:cmsdaq}}

\subsection{Triggers\label{sec:cmstriggers}}

\subsection{Event reconstruction\label{sec:cmsreco}}
%\include{}

\bibliography{mythesis}

% The UMI abstract uses square brackets!
\UMIabstract[The abstract that is submitted to UMI must be formatted as shown in the example here. The body of the abstract cannot exceed 350 words. It should be in typewritten form, double-spaced, and on bond paper. It is important to write an abstract that gives a clear description of the content and major divisions of the dissertation, since UMI will publish the abstract exactly as submitted. Students completing their requirements under Plan A should provide extra copies of the typed summary for use by the dissertation committee during the examination.

The abstract that is submitted to UMI must be formatted as shown in the example here. The body of the abstract cannot exceed 350 words. It should be in typewritten form, double-spaced, and on bond paper. It is important to write an abstract that gives a clear description of the content and major divisions of the dissertation, since UMI will publish the abstract exactly as submitted. Students completing their requirements under Plan A should provide extra copies of the typed summary for use by the dissertation committee during the examination.

The abstract that is submitted to UMI must be formatted as shown in the example here. The body of the abstract cannot exceed 350 words. It should be in typewritten form, double-spaced, and on bond paper. It is important to write an abstract that gives a clear description of the content and major divisions of the dissertation, since UMI will publish the abstract exactly as submitted. Students completing their requirements under Plan A should provide extra copies of the typed summary for use by the dissertation committee during the examination.

The abstract that is submitted to UMI must be formatted as shown in the example here. The body of the abstract cannot exceed 350 words. It should be in typewritten form, double-spaced, and on bond paper. It is important to write an abstract that gives a clear description of the content and major divisions of the dissertation, since UMI will publish the abstract exactly as submitted. Students completing their requirements under Plan A should provide extra copies of the typed summary for use by the dissertation committee during the examination.]

\end{document}
