\chapter{Conclusions\label{sec:conclusions}}

We have performed search for the decay $H\rightarrow$$aa/hh\rightarrow4\tau$ in the gluon fusion, $W$ associated, $Z$ associated, and vector boson fusion production modes. The observed data was consistent with Standard Model expectations, and no evidence of this exotic decay was found. Thus, we have set model-independent upper limits on the branching ratio to new physics, assuming SM production of the 125 GeV Higgs.  For a 9 GeV pseudoscalar, an upper limit of 18.9\% was set on $BR(\PH\to4\tau)$. The most stringent limits, 13.7\% and 13.6\%, were set at the 11 and 13 GeV pseudoscalar mass points respectively. These branching ratios can be interpreted in the context of any 2HDM models that allow the decay of $H$ to light scalars or pseudoscalars. This result is the first of its kind at the LHC.

The boosted tau identification techniques developed in this search can find promising use in future searches during the LHC's Run II at 13 TeV. Decays such as $H\rightarrow$$aa\rightarrow2\mu2\tau$ and $H\rightarrow$$aa\rightarrow2\tau b\bar{b}$ remain to be explored. Searches need not be limited to Higgs studies either, as any other event topology involving boosted tau pairs could benefit from these techniques.