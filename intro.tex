\chapter{Introduction\label{sec:intro}}

%Standard Model
% - What does it consist of (particles, interactions, gauge fields)
% - Description of Higgs mechanism and its significance
% - Examples of SM limitations (DM, gravity, Higgs loop corrections...)

%Supersymmetry
% - What is supersymmetry, what SM problems does it address
% - 2HDM, NMSSM: Motivations
% - 

\subsection{The Standard Model\label{sec:SM}}

Elementary particle physics is the study of the most fundamental components of matter: determining what these components are, and also how they behave. The word ``atom" is derived from the ancient Greek word for ``indivisible"; the discovery that atoms are not in fact indivisible but are themselves made up of smaller components -- electrons, protons, and neutrons -- set the stage for the ongoing search for ever-smaller and more fundamental building blocks of matter. This has culminated in the present day with the Standard Model, a quantum field theory that provides the most successful description to date of all experimentally observed fundamental particles and their interactions. In this framework, fundamental particles are interpreted as excitations of quantum fields, and interactions between particles occur via the exchange of mediating particles~\cite{BettiniPhysics}. This chapter will present an overview of the Standard Model, its deficiencies, and the theory of supersymmetry that has been developed to address some of these deficiencies.

\subsubsection{A little background\label{sec:SM-history}}

Classical physics differentiates clearly between matter particles, which behave as pointlike objects, and radiation, which behaves as a wave. The discovery of phenomena such as the photoelectric effect and the Compton effect, which point to the quantization radiation, challenged classical assumptions about the wave behaviour of radiation. Similarly, observations of diffraction behaviour in electron beams revealed that treating particles as pointlike bodies is not sufficient, and that beams of electrons can in fact behave like waves~\cite{MessiahPhysics}. The quest to understand this wave-particle duality led to the development of quantum mechanics to describe physical phenomena at the microscopic scale. The propagation of particles or systems of particles and the evolution of their physical states are described by wavefunctions that are functions of the spatial and time coordinates of the particles.

\subsubsection{Particles of the Standard Model\label{sec:SM-particles}}

Fundamental particles can be classified into three main categories: leptons, quarks, and gauge bosons. An illustration of the classification of Standard Model particles is shown in Fig.~\ref{fig:StandardModelTable}.

\begin{figure}
   \begin{center}
      \includegraphics[width=0.6\textwidth]{figures/StandardModelTable}
      \caption{Standard Model particles.}
      \label{fig:StandardModelTable}
   \end{center}
\end{figure}

The leptons and quarks are all fermions -- particles with half-integer spin whose dynamics obey the Dirac equation. Leptons have integer electric charge and fall into three generations called ``flavours". The negatively charged leptons are the electron, the muon, and the tau (in increasing order of mass), and each is associated with a massless neutrino; for each lepton, there is also an associated antiparticle. Quarks have fractional electric charge and fall into three generations; there are a total of six quark flavours (two per generation), and quarks also possess another quantum property known as ``colour", whose implications will be explained shortly.

Particles interact via four fundamental forces: the strong force, electromagnetism, the weak force, and gravity. These interactions are mediated by gauge bosons, which obey Bose-Einstein statistics and have integer spin. Any particle with electric charge can participate in electromagnetic interactions, which are mediated by electrically neutral, massless photons. The strong force is mediated by gluons, which are colorless, electrically neutral, and massless; only quarks and gluons, which possess nonzero colour charge, can participate in strong interactions. W$^{\pm}$ or Z$^{0}$ bosons are the carriers of the weak force, which is responsible for such processes as nuclear decays; they will hence be referred to as $W$ and $Z$, dropping the charge superscript unless it is necessary to explicitly mention their charges. Unlike the gluon and photon, the $W$ and $Z$ bosons are massive.
%flavour: , each of which can be regarded as a representation of the SU(2)$\times$U(1) group
%colour: , which can be regarded as representations of the SU(3) group

The dynamics of quantum mechanical processes such as transitions between states are described by a function called the Lagrangian . By Noether's theorem, the symmetry of a Lagrangian under some transformation implies the conservation of some quantity. described by a symmetry group called a gauge group. Thus, fundamental interactions are governed by the symmetry of their Lagrangians under gauge transformations. The conservation of electric charge, for instance, is a consequence the invariance of the electromagnetic interaction under U(1) transformations; 
Transitions between quantum states are governed by the conservation of quantum numbers associated with the fundamental interaction mediating the transition.

The Standard Model belongs to the symmetry group SU(3)$\times$SU(2)$\times$U(1). The invariance of the QCD Lagrangian under the SU(3) gauge group requires that gluons, the generators of the SU(3) symmetry, be massless. By itself, the SU(2)$\times$U(1) gauge group of the electroweak theory also predicts that its force carriers be massless, but although the photon is massless, the W and Z bosons are not, as their masses have been experimentally measured. This paradox is resolved by the concept of electroweak symmetry breaking, also known as the Higgs mechanism. This model introduces a weak isospin doublet of complex scalar fields, with a total of four degrees of freedom, which breaks the gauge symmetry of the SU(2)$\times$U(1) Lagrangian of the electroweak potential, thus generating mass terms for the W and Z bosons but leaving the photon massless. It also generates the masses of the fermions via Yukawa interactions in the electroweak Lagrangian.

The SU(2)$\times$U(1) symmetry of the electroweak Lagrangian is broken by the introduction of the Higgs field, an SU(2) doublet of complex scalar fields~\cite{ThomsonPhysics}. Via this mechanism, mass terms are generated for the W and Z bosons 

\subsection{Deficiencies of the Standard Model\label{sec:SMdeficiencies}}

Although the Standard Model has been successful in describing a wide range of experimental results, it also falls short of providing a complete description of nature in many respects. It describes only three out of the four fundamental forces; the way in which gravity, which is $10^{-32}$ times weaker than the weak force, factors into the Standard Model is still unknown. It fails to provide a satisfactory description of neutrino oscillations, an explanation of the matter-antimatter asymmetry in the universe, or a suitable candidate (or candidates) for dark matter.

The Standard Model also predicts that the mass of the Higgs boson receives loop corrections -- corrections of a magnitude close to the Planck scale at which the three fundamental forces are expected to be unified. The fact that the experimentally measured Higgs boson mass is 125 GeV -- orders of magnitude smaller than the corrections -- means that these loop corrections must somehow be cancelled, but it is not known how. This constitutes what is known as the hierarchy problem of particle physics.

\subsection{Supersymmetry}

One theory that has been proposed to address the hierarchy problem is that there exists a certain symmetry, referred to a supersymmetry, that relates fermions to bosons. For each fermion, there would exist a corresponding bosonic parter particle, and likewise for each boson there would exist a fermionic partner; such partner particles are referred to as superpartners. Under a supersymmetric transformation, a fermion turns into its superpartner and vice versa. Thus, for each loop correction to the Higgs mass, there would be another loop correction from the supersymmetric partner particle; since fermionic loops have a sign opposite that of bosonic loops, the loop corrections would cancel neatly.
As supersymmetry posits a superpartner for every Standard Model particle, it is of great interest to look for experimental evidence of these superpartners, as none have yet been observed. The masses of supersymmetric particles are not specified, but the numerous parameters of the theory can be constrained by experimental measurements.

The simplest model incorporating supersymmetry into the Standard Model is the Minimal Supersymmetric Standard Model (MSSM), whose Higgs sector consists of two scalar Higgs doublets and a total of five Higgs bosons -- a pair of charged ones, two neutral scalars, and a neutral pseudoscalar. While it does address the hierarchy problem, it comes with deficiencies of its own; one problem, called the mu-problem, stems from the fact that the $\mu$ coupling constant in the Higgsino mass parameter of the MSSM Lagrangian should be of the order of magnitude of the electroweak scale in order to provide the Higgs doublets with nonzero vacuum expectation values after electroweak symmetry breaking, but its magnitude would be expected to be more naturally close to the Planck scale, which is significantly larger; fine-tuning its magnitude to that of the electroweak scale thus seems arbitrary and without physical motivation.

The $\mu$-problem is circumvented in the Next-to-Minimal Supersymmetric Standard Model (NMSSM), which adds a scalar Higgs singlet to the Higgs sector of the MSSM and thus predicts a total of seven Higgs bosons -- a pair of charged ones, three neutral scalars, and two neutral pseudoscalars. The coupling of the singlet field to the doublet fields naturally generates an effective $\mu$ term of the desired magnitude near the electroweak scale via the vacuum expectation value of the singlet field.